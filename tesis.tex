\documentclass{report}

\usepackage{graphicx}
\usepackage{amsthm}
\usepackage{amsfonts}
\usepackage{enumerate}
\usepackage{amsmath}
\usepackage[utf8]{inputenc}
\usepackage{mathtools}
\usepackage[utf8]{inputenc}
\usepackage[spanish]{babel}
\usepackage{fixltx2e} 
\usepackage{amssymb}
 \usepackage{amsmath,amscd}
\usepackage{mathabx} 
 \newcommand\answerbox{%%
    \fbox{\rule{.25in}{0pt}\rule[-0.5ex]{0pt}{4ex}}}
%\pagestyle{empty}
\setlength{\parindent}{1em}
\setlength{\parskip}{1em}

\usepackage{titlesec}
\usepackage{lipsum}
\usepackage{pdfpages}

\newtheorem{theorem}{Teorema}[section]
\newtheorem{prop}[theorem]{Proposici\'on}
\newtheorem{lem}[theorem]{Lema}
\newtheorem{coro}[theorem]{Corolario}

\theoremstyle{definition}
\newtheorem{defi}[theorem]{Definici\'on}


\DeclareMathOperator{\gl}{GL}
\DeclareMathOperator{\dif}{Dif}
\DeclareMathOperator{\inm}{Inm}
\DeclareMathOperator{\rango}{rango}
\DeclareMathOperator{\inte}{int}
\DeclareMathOperator{\sop}{sop}
\DeclareMathOperator{\cod}{codim}
\DeclareMathOperator{\cor}{corango}
\DeclareMathOperator{\dom}{dom}
\DeclareMathOperator{\im}{im}
\DeclareMathOperator{\cok}{coker}
\usepackage[utf8]{inputenc}


\newcommand{\overbar}[1]{\mkern 1.5mu\overline{\mkern-1.5mu#1\mkern-1.5mu}\mkern 1.5mu}

\let\hom\homo

\DeclareMathOperator{\hom}{Hom}

\usepackage[all,cmtip]{xy}

\newcommand{\transv}{\mathrel{\text{\tpitchfork}}}
\makeatletter
\newcommand{\tpitchfork}{%
  \vbox{
    \baselineskip\z@skip
    \lineskip-.52ex
    \lineskiplimit\maxdimen
    \m@th
    \ialign{##\crcr\hidewidth\smash{$-$}\hidewidth\crcr$\pitchfork$\crcr}
  }%
}


\author{Lorenzo Gustavo Reyes N\'uñez}
\title{Estabilidad topol\'ogica y sus equivalencias}

\begin{document}

\includepdf{666.pdf}

\maketitle

\tableofcontents


\chapter{Introducci\'on}


En f\'isica decimos que un fen\'omeno es observable si al perturbarlo un poco sigue siendo ``igual". En nuestro caso, un fen\'omeno va a ser representado por una funci\'on $f:M \to N$ de clase $C^\infty$ con $M$ y $N$ variedades diferenciables. La parte de que un fen\'omeno sea observable se traduce en que si $f^\prime:M \to N$ es suficientemente cercana a $f$, entonces existen $g:M \to M$ y $h:N \to N$ difeomorfismos tales que el siguiente diagrama conmuta,
$$\xymatrix{
M \ar[d]^g \ar[r]^f &N\ar[d]^h\\
M \ar[r]^{f^\prime} &N .}$$
Al pedirle a $g$ y a $h$ que sean difeomorfismos garantizamos que $f$ y $f^\prime$ sean ``iguales". En nuestro caso, a las funciones observables las llamaremos estables. Por ejemplo, nos podemos tomar la funci\'on cuya gr\'afica es:

\includegraphics[scale=.4]{1}.

Podemos hacer una pequeña deformaci\'on de esta funci\'on de modo que obtengamos una funci\'on del siguiente estilo:

\includegraphics[scale=.4]{2}

Nuesta funci\'on original no ser\'a estable, ya que, los difeomorfismos en particular son funciones biyectivas, entonces en la primera imagen los valores cr\'iticos tienen cardinalidad 2 y en la segunda tienen 3. Hasta el momento hemos hablado de que dos funciones est\'en cercanas, esto se refiere a que el espacio de funciones de clase $C^\infty$ entre $M$ y $N$ tiene una topolog\'ia. Esta topolog\'ia se le conoce como topolog\'ia $C^\infty$ de Whitney, debida al matem\'atico Hassler Whitney.

En este trabajo intentamos exponer de una manera clara y simple esta teor\'ia. Aunque nuestros resultados se presenten en contextos m\'as generales nuestro prop\'osito ser\'a demostrarlos en el caso de mapeos entre espacios euclidianos y cuando sea posible demostrarlos en variedades. Iniciamos con un cap\'itulo de preliminares exponiendo lo m\'as esencial de topolog\'ia diferencial y dando ejemplos que nos ser\'an \'utiles en cap\'itulos posteriores.

En el cap\'itulo 3, definiremos la topolog\'ia $C^\infty$ de Whitney adem\'as daremos las propiedades m\'as importantes de este espacio, como que es un espacio de Baire, el teorema de transversalidad de Thom [1956] y el teorema de inmersi\'on de Whitney.

En el siguiente cap\'itulo empezaremos dando la definici\'on formal de estabilidad topol\'ogica, lo primero que notaremos ser\'a que con nuestra definici\'on es muy dif\'icil determinar si un mapeo es estable. Nuestra meta ser\'a encontrar criterios suficientes y necesarios para saber si una funci\'on es estable, para esto demostraremos el teorema de Mather[1968-70] el cual introduce un nuevo tipo de estabilidad, conocido como estabilidad infinitesimal. Para demostrar que estabilidad topol\'ogica es equivalente a estabilidad infinitesimal daremos nuevos tipos de estabilidad equivalentes a estos dos. Estabilidad infinitesimal es una definici\'on mucho m\'as manejable ya que demostraremos que es una condici\'on de orden finito gracias al teorema de preparaci\'on de Malgrange [1962-63]. Finalizamos este cap\'itulo dando ejemplos de funciones estables.

En el \'ultimo cap\'itulo iniciamos resolviendo el problema de estudiar los mapeos estables entre variedades de dimensi\'on 2. Para finalizar este trabajo mencionamos en cuales casos los mapeos estables forman un conjunto denso.

\chapter{Preliminares}

El prop\'osito de este cap\'itulo es recordar conceptos b\'asicos de topolog\'ia diferencial, ponernos de acuerdo en la notaci\'on y desarrollar ejemplos que nos servir\'an en cap\'itulos posteriores.

\section{Variedades diferenciables}

\begin{defi}
Una \textit{variedad topol\'ogica} $M^m$ de dimensi\'on $m$ es un espacio top\'ologico que es Hausdorff, segundo numerable y localmente euclidiano. Esto \'ultimo quiere decir que si $p \in M$ existe un abierto $U \subseteq M^m$, un abierto $U^\prime$ de $\mathbb{R}^m$ y un homeomorfismo de $U$ a $U^\prime$, el cual denotaremos como $\phi_p$. El mapeo $\phi_p$ diremos que es una carta. A partir de este momento omitiremos el super\'indice de la definici\'on, las variedades ser\'an denotadas por letras may\'usculas (normalmente $M$ y $N$) y su dimensi\'on por la misma letra pero en min\'uscula.
\end{defi}

Un punto $\overbar{x}$ de $\mathbb{R}^n$ lo podemos describir por una en\'eada $(x_1, \dots , x_n)$ donde $x_i \in \mathbb{R}$ para toda $i$. Entonces alrededor de cada punto mediante las cartas podemos darle coordenadas a nuestra vecindad $U$. Antes de dar la definici\'on de variedad diferenciable, necesitaremos revisar el concepto de atlas diferenciable.

\begin{defi} Una colecci\'on de cartas sobre una variedad $M$ $ \{ \phi_\alpha \mid \alpha \in A \}$  donde $A$ es un conjunto de \'indices, se dice que es un \textit{atlas} $\mathfrak{A}$ de clase $C^\infty$(diferenciable) si: 

\begin{enumerate}

\item $\bigcup\limits_{\alpha \in A} U_\alpha = M$, donde $U_\alpha$ es el dominio de la carta correspondiente a $\phi_\alpha$. 

\item Si tenemos cartas $(U_1, \phi_1)$ y $(U_2, \phi_2)$ con $U_1 \cap U_2 \neq \emptyset$ entonces $\phi_2 \circ \phi_1^{-1}: \phi_1(U_1 \cap U_2) \to \mathbb{R}^n$ es un difeomorfismo local de clase $C^\infty$. 

\end{enumerate}
\end{defi}

Diremos que dos atlas diferenciables $\mathfrak{A}$ y $\mathfrak{B}$ son compatibles si $\mathfrak{A} \cup \mathfrak{B}$ vuelve a ser un atlas diferenciable. Dado un atlas $\mathfrak{A}$ de $M$, una estructura diferenciable en $M$ es la uni\'on de todos los atlas que son compatibles con $A$. Una variedad es una pareja $(M, \mathfrak{A} )$ donde $M$ es una variedad topol\'ogica y $\mathfrak{A}$ es una estructura diferenciable en $M$.

Ejemplos: 
\begin{enumerate}

\item Sea $U$ un abierto de $\mathbb{R}^n$, entonces $U$ es un abierto con la topolog\'ia inducida y con una \'unica carta $Id: U \to U$.

\item La esfera $S^n = \{ \overbar{x} \in \mathbb{R}^{n + 1} \mid \Vert \overbar{x} \Vert = 1 \}$ es una variedad cuyas cartas est\'an dadas por laa proyecciones estereogr\'aficas.

\item El conjunto de todos los subespacios de dimensi\'on $s$ de $\mathbb{R}^n$, esta variedad se llama el grassmanniano de $s$-planos en $\mathbb{R}^n$ y se denota como $G(s, \mathbb{R}^n)$. Sea $W$ el subconjunto de $V^s$ de $s$ vectores linealmente independientes de $V$, donde $V= \mathbb{R}^n$. Definimos en $W$ la siguiente relaci\'on de equivalencia; $w_1 \sim w_2 \iff \langle w_1 \rangle = \langle w_2 \rangle$. Esto claramente es una relaci\'on de equivalencia. 

Por como definimos $\sim$, tenemos que $G(s, \mathbb{R}^n) = W/\sim$ como conjuntos, entonces podemos darle a $G(s, \mathbb{R}^n)$ la topolog\'ia cociente, lo primero que podemos notar es que con esta topolog\'ia $G(s, \mathbb{R}^n)$ es segundo numerable.

Para dar las cartas nos tomamos $V$ un subespacio de $\mathbb{R}^n$ de dimensi\'on $s$ y sea $\pi_V$ la proyecci\'on ortogonal de $\mathbb{R}^n$ en $V$. Sea $\pi_{U , V}$ la restricci\'on de $\pi_V$ a $U$, donde $U$ es un subespacio de $\mathbb{R}^n$ de dimensi\'on $n-k$. Sea $W_V = \{ U \in G(s, \mathbb{R}^n) \mid \pi_{U, V} \text{ es una biyecci\'on} \}$.

Definimos $\rho_V: W_V \to \hom (V, V^\perp )$, $\rho_V (U) = \pi_{U , V^\perp } \circ \pi_{U, V}^{-1}$. Por como definimos $W_V$, esta funci\'on tiene inversa y por lo tanto es un homeomorfismo.

\item Sean $M$ y $N$ dos variedades, podemos definir el producto topol\'ogico $M \times N$ y darle estructura diferenciable de la siguiente manera. Sean $\mathfrak{A} = (U_\alpha, \phi_\alpha)_{\alpha \in I}$ y $\mathfrak{B} = (V_\beta , \psi_\beta)_{\beta \in J}$ atlas en $M$ y $N$ respectivamente, definimos $\mathfrak{A} \times \mathfrak{B}$ como $(U_\alpha \times V_\beta , \phi_\alpha \times \psi_\beta)_{\alpha \times \beta \in I \times J}$.



\end{enumerate}

Hasta el momento hemos definido nuestros objetos de inter\'es; lo que ahora necesitamos es relacionar estos objetos, esto lo haremos mediante funciones diferenciables. Diremos que $\rho: M \to \mathbb{R}$ es de clase $C^\infty$ si para toda carta $\phi$ el mapeo $\rho \circ \phi^{-1}: \text{imagen}(\phi) \to \mathbb{R}$ es de clase $C^\infty$.

Ahora si tenemos una funci\'on $f:M \to N$ diremos que es de clase $C^\infty$ si para toda funci\'on $\rho: N \to \mathbb{R}$ de clase $C^\infty$ la composici\'on $f \circ \rho$ es de clase $C^\infty$. Diremos que $f$ es un difeomorfismo si tiene inversa suave, es decir, existe $g:N \to M$ tal que $f \circ g = Id_{M}$ y $g \circ f = Id_N$ . A las funciones $C^\infty$ tambi\'en se les llama funciones suaves. A partir de este momento todas las funciones con las que trabajaremos ser\'an suaves y nuestras variedades tambi\'en.

En $\mathbb{R}^n$ los abiertos vuelven a ser variedades con la topolog\'ia inducida, tambi\'en podemos pensar a $\mathbb{R}^k$ ``encajado''  en $\mathbb{R}^n$ (con $k \leq n$) como $i_k: \mathbb{R}^k \to \mathbb{R}^n$ donde mandamos a $\mathbb{R}^k$ en $\mathbb{R}^k \times \{ \overbar{ 0} \}$, para cada punto $\overbar{v}$ en $\mathbb{R}^k \times  \{ \overbar{0} \}$ y para toda vecindad $N_{\overbar{v}}$ abierta de \'el si la intersecamos con la imagen de $i_k$ obtendremos un ``abierto" de $\mathbb{R}^k$, una manera de generalizar este detalle es la siguiente:

\begin{defi} Sea $M^m$ variedad y $N^n \subseteq M$, decimos que $N$ es una \textit{subvariedad} de dimensi\'on $n$ de $M$ si para todo $x \in N$ existe una carta $\phi: \dom (\phi) \to \mathbb{R}^m$ cuyo dominio contenga a $x$ tal que $U \cap N = \phi^{(-1)} (\mathbb{R}^n \times \{ 0 \}) $. Denotamos a la codimensi\'on de $N$ en $M$ como $\cod (N) = \dim M - \dim N$. 
\end{defi}




\section{Espacio tangente}

Sea $M^m$ una variedad diferenciable y $x \in M$, sea $C^\infty (M)$ el anillo de funciones $C^\infty$ de $M$ a $\mathbb{R}$ con la suma y producto puntual. Definimos en este conjunto la siguiente relaci\'on de equivalencia: $$f \sim g \iff \exists U_x \text{ vecindad de } x \text{ tal que }f_{\vert U_x} = g_{\vert U_x} . $$

Sea $C^\infty_x (M) = C^\infty (M) / \sim$, a este conjunto le podemos dar estructura de anillo mediante suma y producto de representantes, y adem\'as de $\mathbb{R}$-\'algebra con el morfismo $\mathbb{R} \to C^\infty_x (M)$ que a cada n\'umero real le asigna la clase de la funci\'on constante, es claro que estas operaciones est\'an bien definidas. Denominaremos a $C^\infty_x (M)$ como el anillo de g\'ermenes alrededor de $x$, a un elemento de este anillo le llamaremos germen y a la clase de $f \in C^\infty(M)$ la denotamos por $\overbar{f}$.

Lo primero que vamos a demostrar de este anillo es que es local (tiene un \'unico ideal maximal), sea  $\mathfrak{m}_x (M) \equiv \{ \overbar{f} \in C^\infty_x (M) \mid \overbar{f} (x) = 0 \}$. Aqu\'i cuando evaluamos al germen $\overbar{f}$ en $x$ en realidad estamos evaluando a un representante, pero es evidente que si tomamos cualquier otro representante de la clase de $\overbar{f}$ \'este tambi\'en se anular\'a en $x$.

 Claramente el germen $\overbar{0} \in \mathfrak{m}_x (M)$, si $\overbar{f}, \: \overbar{g} \in \mathfrak{m}_x$, nos tomamos  $f$ y $g$ representantes de cada clase correspondiente, entonces $(f + g)(x) = 0$ y esto obviamente no depender\'a del representante elegido. Sea $\overbar{f} \in \mathfrak{m}_x (M)$ y $\overbar{h} \in C^\infty_x (M)$, nos tomamos un representante de cada clase $f$ y $h$ respectivamente, entonces $(fh)(x) = f (x) h(x) = 0$, entonces $\overbar{f} \overbar{h} \in \mathfrak{m}_x (M)$. Por lo tanto, $\mathfrak{m}_x (M)$ es ideal.
 
 Ahora si nos tomamos $\overbar{h}$ en el complemento de $\mathfrak{m}_x (M)$ y sea $h$ un representante de esta clase. Por como definimos a nuestro ideal, existe una vecindad $V_x$ de $x$ tal que $h_{\vert V_x} \neq 0$. Entonces $h$ y $h_{\vert V_x}$ pertenecen a la misma clase $\overbar{h}$ en $C^\infty_x (M)$, entonces $h_{\vert V_x}$ es invertible y su inverso est\'a dado por $1 / h_{\vert V_x}: V_x \to \mathbb{R}$. Entonces el complemento de nuestro ideal $\mathfrak{m}_x (M)$ nada m\'as est\'a conformado por unidades, por lo tanto nuestro anillo es local.
 
\begin{defi}
 Una \textit{derivaci\'on} $X$ es un mapeo lineal de $C^\infty_x (M) \to \mathbb{R}$ que cumple la regla de Leibniz, es decir, $$X( \overbar{f} \cdot \overbar{g} ) = X( \overbar{f} ) \cdot \overbar{g}(x) + \overbar{f} (x) X (\overbar{g}) .$$
\end{defi}


Claramente el conjunto de derivaciones es un subespacio vectorial de los mapeos lineales entre $C^\infty_x (M)$ y $\mathbb{R}$, lo llamaremos espacio tangente en $x$, y la notaci\'on que usaremos a lo largo de este trabajo ser\'a $T_x M$. 

La dimensi\'on de $T_x M$ coincide con la de $M$, la base para $T_x M$ esta dada por el conjunto $\{ \partial / \partial x_1, \dots , \partial / \partial x_n \}$, donde cada una de estas derivaciones es derivar parcialmente con respecto a su coordenada correspondiente. 

Esta definici\'on de espacio tangente es muy \'util para hacer c\'alculos pero le hace falta geometr\'ia, si pensamos en un abierto de un espacio euclidiano el espacio tangente en un punto $x$ es el espacio de velocidades con las cuales puede pasar una curva suave $c$ en $x$. Para poder ocupar esta definici\'on en variedades ocuparemos cartas. Sea $W$ el conjunto de curvas suaves de $(-1, 1)$ en $M$ tales que $c(0) =x$ para toda curva $c$. Definimos en $W$ la siguiente relaci\'on de equivalencia, $c_1 \sim c_2 \iff   (\phi \circ c_1)^\prime (0) = (\phi \circ c_2)^\prime (0)$ para alguna carta $\phi$ fija tal que su dominio contenga a $x$. A $W/ \sim$ le podemos dar estructura de espacio vectorial mediante nuestra carta $\phi$, las operaciones estar\'an dadas por: 

\begin{enumerate}

\item $\overbar{c_1} + \overbar{c_2} = \overbar{ \phi^{-1} ((\phi (c_1 ) + \phi (c_2 )^\prime (0))}$. Esta \'ultima clase de equivalencia corresponde a la  l\'inea recta que pasa por el origen con velocidad constante $(\phi (c_1 (0)) + \phi (c_2 (0))^\prime$. 

\item $\lambda \cdot \overbar{c_1} = \overbar{ \phi^{-1} (\lambda ((\phi \circ c_1)^\prime (0))} .$ 
\end{enumerate}

Debido a la linealidad de la derivada estas operaciones no depender\'an del representante. En este caso para definir el espacio tangente ocupamos nada m\'as una carta, $\phi$, ¿si ocup\'aramos otra carta obtendr\'iamos el mismo espacio vectorial? La respuesta es s\'i. Esto se debe a que los cambios de carta son difeomorfismos, entonces nuestra definici\'on de espacio tangente mediante curvas no depende de la carta. 

Estaremos usando las dos definiciones de espacio tangente dependiendo de lo que estemos haciendo, pero no es dif\'icil convencerse de que estas definiciones son equivalentes.

Si tenemos un mapeo $f: M \to N$ suave, con $f(x)=y$, $f$ nos induce un mapeo lineal $(df)_x:T_x M \to T_{y} N$, donde a una clase $\overbar{c}$ la enviamos a $\overbar{f \circ c }$. 

\begin{defi} 
Sea $f:M^m \to N^n$ suave con $f(x) =y$, si $(df)_x$ tiene rango m\'aximo decimos que $f$ es:
\begin{enumerate}
\item Una \textit{sumersi\'on} en $x$ si $m \geq n$,
\item Una \textit{inmersi\'on} en $x$ si $m \leq n$.

\end{enumerate}
\end{defi}
Decimos que $f$ es una inmersi\'on o sumersi\'on si es una inmersi\'on o sumersi\'on respectivamente en cada uno de sus puntos.

\begin{defi}
Sean $M$ y $N$ variedades, $f: M \to N$ suave, $x \in M$ y $y \in N$, entonces:
\begin{enumerate}
\item $\cor (df)_x = \min \{ \dim M, \dim N\} - \rango ((df)_x)$.
\item $x$ es un \textit{punto cr\'itico} si $\cor (df)_x >0$,
\item $y$ es un \textit{valor cr\'itico} si $f^{-1} (y)$ tiene un punto cr\'itico,
\item $x$ es un \textit{punto regular} si no es un punto cr\'itico,
\item $y$ es un \textit{valor regular} si no es un valor cr\'itico. En particular si $y \notin Im(f)$ entonces $y$ es un valor regular.
\end{enumerate}
\end{defi}
\begin{theorem}
Sea $f: M^m \to N^n$ suave con $f(x)=y$, si $f$ es una inmersi\'on en $x$, entonces existen cartas $\phi_x$ y $\psi_y$ alrededor de $x$ y $y$ respectivamente tal que $\phi_x^{-1} \circ f \circ \psi_y: \dom \phi_x^{-1} \to \rango  \psi_y$ es la inclusi\'on can\'onica de $\mathbb{R}^n$ en $\mathbb{R}^m$. (Podemos cambiar que $f$ sea una sumersi\'on en $x$ para que el mapeo $\phi_x^{-1} \circ f \circ \psi_y: \dom \phi_x^{-1} \to \rango  \psi_y$ sea la proyecci\'on can\'onica de $\mathbb{R}^n$ en $\mathbb{R}^m$).
\end{theorem}

Este teorema no lo demostraremos, pero es una consecuencia del teorema de la funci\'on impl\'icita de c\'alculo vectorial. 


\section{Algunas cosas \'utiles}

Antes de seguir daremos algunas definiciones y teoremas que necesitaremos en el futuro, cuyas demostraciones se pueden encontrar en cualquier libro b\'asico de topolog\'ia diferencial \cite{Introduction to Differential Topology, Differential Topology}.

\begin{defi}
Sea $M^m$ una variedad y $A \subset M$. $A$ tiene \textit{medida cero} en $M$ si existe una colecci\'on de cartas $\{ \phi_i \}_{i \in \mathbb{N}}$ con $ A \subset \bigcup\limits_{i=1}^\infty \dom (\phi_i )$ tales que para cada $i \in \mathbb{N}$, $\phi_i (\dom (\phi_i)) \cap A$ tiene medida cero en $\mathbb{R}^m$.
\end{defi}

Notemos que si para una cubierta $A$ tiene medida cero entonces para cualquier cubierta lo tendr\'a ya que los mapeos de transici\'on son suaves y mandan conjuntos de medida cero en medida cero. Tambi\'en le pedimos a la cubierta que sea numerable porque si una cubierta no numerable cubre a $A$ entonces existir\'a una subcubierta numerable ya que las variedades son segundo numerable. 

\begin{theorem}
Sea $M^m$ variedad, $N^n \subset M^m$ subvariedad con $n < m$, entonces $N$ tiene medida cero en $M$.
\end{theorem}

\begin{theorem}
Sea $M$ variedad, $U \subset M$ abierto, entonces $U$ no tiene medida cero.
\end{theorem}

\begin{theorem}{Teorema de Sard.}
Sean $M$ y $N$ variedades, $f: M \to N$ suave, entonces el conjunto de valores cr\'iticos de $f$ tiene medida cero en $N$.
\end{theorem}

\begin{defi}
Sea $M$ una variedad, $\{ U_i \}_{i \in I}$ y $\{ V_j \}_{j \in J} $ cubiertas de $M$, entonces:
\begin{enumerate}
\item $\{ V_j \}_{j \in J} $ es un refinamiento de $\{ U_i \}_{i \in I}$ si para todo $j \in J$ existe $i \in I$ tal que $V_j \subset U_i$.
\item $\{ V_j \}_{j \in J}$ es localmente finita si para todo $x \in M$ existe una vecindad $W$ tal que $W \cap V_j = \emptyset$ para casi toda $j \in J$, es decir, $V \cap U_j \neq \emptyset$ nada m\'as para un n\'umero finito de \'indices.

\end{enumerate}
\end{defi}

\begin{theorem}
Sea $M$ variedad. Toda cubierta de $M$ tiene una subcubierta localmente finita.

\end{theorem}

\begin{defi}
Sea $M$ variedad, entonces
\begin{enumerate}
\item Sea $\rho : M \to \mathbb{R}$, el \textit{soporte} de $\rho$, denotado por $\sop(\rho )$, es la cerradura del conjunto $ \{x \in M \mid \rho (x) \neq 0\} $
\item Una \textit{partici\'on de la unidad} en $M$ es una familia de funciones suaves real valuadas $\{ \rho_i \}_{i \in I}$ tales que  
\begin{enumerate}[a.]
\item $\{ \sop (\rho_i) \}_{i \in I}$ es una cubierta de $M$ localmente finita.
\item $\rho_i (x) \geq 0$ para todo $i \in I$ y para todo $x \in M$.
\item $\sum\limits_{i \in I} \rho_i (x) \equiv 1$ para todo $x \in M$. La suma siempre ser\'a finita por a.
\item Una partici\'on de la unidad $ \{ \rho_i \}_{i \in I}$ en $M$ es subordinada a una cubierta $\{ U_j \}_{j \in J}$ si para todo $i \in I$ existe $j \in J$ tal que $\sop (\rho_i ) \subset U_j$.
\end{enumerate}
\end{enumerate}
\end{defi}

\begin{theorem}
Sea $M$ variedad y $\{ U_i \}_{i \in I}$ cubierta abierta de $M$, entonces existe una partici\'on de la unidad $\{ \rho_i \}_{i \in I} $ subordinada a esta cubierta.
\end{theorem}
\section{Haces vectoriales}

\begin{defi}
 Sean $E^{m+k}$ y $M^m$ variedades y $\pi: E \to M$ una sumersi\'on. Decimos que $E$ es un \textit{haz vectorial} sobre $M$ de rango $k$ si:

\begin{enumerate}
\item Para toda $x \in M$, $\pi^{-1} (x)$ es un espacio vectorial sobre $\mathbb{R}$ de dimensi\'on $k$, cuyas operaciones son funciones suaves en $\pi^{-1} (x)$. A $\pi^{-1} (x)$ le llamaremos la fibra de $x$ en $E$.
\item Para todo $x \in M$ existe una vecindad $U_x$, tal que existe un difeomorfismo de $\phi_x: \pi^{-1} (U_x) \to \mathbb{R}^m \times \mathbb{R}^k$, 


\end{enumerate}
\end{defi}
Normalmente a $E$ le pondremos super\'indice la dimensi\'on de la fibra en lugar de la dimensi\'on de variedad diferenciable, entonces en lugar de escribir $E^{n+k}$ escribiremos $E^k$. De 2) podemos concluir que $\pi$, adem\'as de ser una sumersi\'on, es suprayectiva. A $M$ le llamaremos el espacio base de $E$.

\begin{defi}
 Sean $E$ y $F$ haces vectoriales sobre $M$. Una funci\'on diferenciable $f: E \to F$ es un \textit{homomorfismo entre haces vectoriales} si $f \circ \pi_F = \pi_E$ y  para todo $x \in M$ $f \vert_{E_x}: E_x \to F_x$ es una transformaci\'on lineal entre espacios vectoriales.
 \end{defi}

\begin{defi}
 Sea $E$ un haz vectorial sobre $M$. Una \textit{secci\'on} de $E$ es una transformaci\'on diferenciable $X: M \to E$ tal que $ \pi \circ X \equiv Id_M.$
\end{defi}

 Lo que hace una secci\'on es que a cada $x$ en $M$ le asigna un vector en su fibra $\pi^{-1} (x)$. El conjunto de secciones de un haz $E$ sobre $M$ es un conjunto no vac\'io, ya que cada espacio vectorial tiene un punto especial, el neutro aditivo $0$, entonces si a cada $x \in M$ le asignamos el vector $0$ en su fibra obtendremos claramente una secci\'on.

Ejemplos:
\begin{enumerate}

\item Sea $M$ una variedad, el producto $M \times \mathbb{R}^k$ es un haz vectorial sobre $M$ con la topolog\'ia producto, cuya fibra tiene dimensi\'on $k$. A esta clase de haces les llamaremos haces triviales.

\item Sea $M$ una variedad, $\bigcup\limits_{x \in M} T_x M$ con la topolog\'ia de la uni\'on disjunta, adquiere estructura de haz vectorial, en cada $x$ de $M$ tendremos su espacio tangente. A este haz en particular le llamaremos haz tangente. A las secciones de este haz se les llama campos vectoriales.

\item Sean $M$ y $N$ variedades diferenciables, $f:M \to N$ suave y $E$ un haz vectorial sobre $N$ con proyecci\'on $\pi$, el \textit{haz pullback} sobre $M$ se define como $f^\ast E = \{ (v, x) \in E \times M \mid f(x) = \pi(v) \}$ con la topolog\'ia inducida por el producto. La proyecci\'on de este haz es la restricci\'on de la proyecci\'on can\'onica $\rho: E \times M \to M$ a $f^\ast E$. Este haz sobre $M$ tiene fibra en $x$ a $E_{f(x)}$, entonces la dimensi\'on de $f^\ast E$ es la misma que la dimensi\'on de $E$. Si $g,f:M \to N$ son suaves, entonces $f^\ast E$ y $g^\ast E$ no necesariamente son isomorfos como haces sobre $M$.

\item Sea $Vect_\mathbb{R}$ la categor\'ia de espacios vectoriales sobre $\mathbb{R}$ de dimensi\'on finita. Sea $T: Vect_\mathbb{R} \to Vect_\mathbb{R}$ funtor covariante. Para cualesquiera $V$ y $W$ en $Vect_\mathbb{R}$ $T$ nos induce un mapeo  $T: \hom (V,W) \to \hom ( T(V) , T(W))$. Como $\hom (V, W) \cong \mathbb{R}^{\dim V \cdot \dim W}$ y lo mismo para $\hom ( T(V) , T(W))$, tenemos un morfismo entre variedades diferenciables, en caso de que $T$ sea suave para cualesquiera par de espacios vectoriales diremos que el funtor $T$ es un funtor suave.

\begin{theorem}
Sea $M$ una variedad suave, $E$ un haz vectorial sobre $M$ y $T: Vect_k \to Vect_k$ funtor covariante (contravariante) suave. Entonces $T(E) = \bigcup\limits_{x \in M} T (E_x)$ es un haz vectorial sobre $M$ con la topolog\'ia de la uni\'on disjunta.


\end{theorem}
\underline{Demostraci\'on:}
 Lo primero que podemos notar es que ya tenemos un mapeo $\pi: X \to T(E)$ suprayectivo que para toda $x \in M$ cumple que $\pi^{-1} (x) = T (E_x)$, es decir, las fibras de este mapeo son espacios vectoriales. Entonces si a $T(E)$ le damos estructura de variedad diferenciable tal que con $\pi$, $T(E)$ se convierta en un haz sobre $M$ nuestro problema estar\'ia resuelto. Si existiera $F$ un haz vectorial sobre $M$ con proyecci\'on $\pi_F$ y una funci\'on biyectiva $\phi: T(E) \to F$ que es lineal en fibras tal que $\pi = \pi_F \circ \phi$ entonces le podemos dar una estructura diferenciable a $T(E)$ inducida por $\phi$ tal que $T(E)$ sea un haz sobre $M$.
\begin{enumerate}[a.]


\item Si tenemos dos haces $F$ y $G$ sobre $M$ y $\phi: F \to G$ un morfismo de haces, entonces hay un mapeo $T(\phi) : T(E) \to T(F)$ que es lineal en fibras, es decir, $\phi \vert_{T (E_x)} : T(E_x) \to T(F_x)$ es lineal.

\item Supongamos que $E = M \times \mathbb{R}^k$ para alguna $k \in \mathbb{N}$. Entonces $T(E) = M \times T(\mathbb{R}^k)$, entonces $T(E)$ con esta identificaci\'on obtiene estructura de variedad diferenciable y de haz vectorial sobre $M$.

\item Ahora supongamos que existe un isomorfismo $\phi: E \to M \times V := F$, por $a$ tenemos un mapeo biyectivo lineal en fibras $T(\phi): T(E) \to T(F)$, por el inciso $b$ podemos darle estructura de haz vectorial a $T(E)$ tal que $T(\phi)$ sea un isomorfismo de haces vectoriales.

Tenemos que demostrar que esta definici\'on no depende de $\phi$, es decir, si existe un isomorfismo de haces $\psi: E \to M \times \mathbb{R}^k := G$ entonces $T(\psi \circ \phi^{-1})$ es un isomorfismo. Como $\phi$ y $\psi$ son isomorfismos entonces $\psi \circ \phi^{-1}: M \times \mathbb{R}^k \to M \times \mathbb{R}^k$ es un isomorfismo, al cual lo podemos pensar como un mapeo $\lambda: M \to \hom(\mathbb{R}^k , \mathbb{R}^k)$, donde $\lambda (x) = \psi \circ \phi^{-1} \vert_{x \times \mathbb{R}^k} : \mathbb{R}^k \to \mathbb{R}^k$. Entonces $ T( \psi \circ \phi^{-1}): M \times T ( \mathbb{R}^k ) \to M \times \mathbb{R}^k$, la podemos pensar como $T \circ \lambda$. Como $\psi$ y $\phi$ son suaves $\psi \circ \phi^{-1}$ es suave, $\lambda$ es suave al igual que $T$, entonces $T \circ \lambda$ es suave. Entonces $ T (\psi \circ \phi^{-1} )$ es suave y es un isomorfismo (es evidente qui\'en es su inverso). Entonces el siguiente diagrama conmuta    $$\xymatrix{
T(E) \ar[d]^{id} \ar[r]^{T(\phi)} &T(F)\ar[d]^{T(\psi \circ \phi^{-1})}\\
T(E) \ar[r]^{T(\psi)} &T(G)} $$

e implica que las estructuras son equivalentes.

\item Sea $E$ un haz vectorial sobre $M$ (no necesariamente trivial) con proyecci\'on $\pi$, entonces para todo $x$ en $M$ existe una vecindad $U_x$, tal que $\pi^{-1} (U_x)$ es un haz trivial sobre $U_x$. Por $c$, $T(U_x)$ tiene una \'unica estructura de haz vectorial. Si tenemos dos abiertos trivializadores $U_x$ y $U_y$ tal que $U_x \cap U_y \neq \emptyset$, entonces $T(U_x \cap U_y)$ tiene dos estructuras de haz vectorial heredadas por $T(U_x)$ y $T(U_y)$, pero por $c$ las estructuras heredadas son equivalentes, es decir, los mapeos de transici\'on de $T(E)$ ser\'an suaves, entonces $T(E)$ es un haz vectorial sobre $M$.

    
\end{enumerate}

El caso de funtor contravariante se hace de manera an\'aloga. $\blacksquare$

Como ejemplo tenemos al funtor $\hom ( \text{\_} , \mathbb{R} )$, que a cada espacio vectorial le asigna su espacio dual. Si $M$ es una variedad suave y $E$ es su haz tangente, a $\hom (E, \mathbb{R})$ le llamaremos haz cotangente, a una secci\'on de este haz le llamaremos 1-forma.

Tambi\'en si nos tomamos un espacio vectorial $\mathbb{R}^k$ nos induce un funtor suave dado por $\text{\_} \bigoplus \mathbb{R}^k$ que a cada espacio vectorial $V$ le asigna el espacio vectorial $V \bigoplus \mathbb{R}^k$.


\item Sea $G(s, \mathbb{R}^n)$ el grasmanniano de s planos en $\mathbb{R}^n$. Para cada $x \in G( s , \mathbb{R}^n )$ sea $E_x$ el subespacio vectorial de dimensi\'on $s$ asociado a $x$. Entonces $E := \bigcup\limits_{x \in G(s , \mathbb{R}^n )} E_x$ es un haz vectorial sobre $G(s, \mathbb{R}^n)$, a este haz le llamaremos el haz tautol\'ogico de $G(s, \mathbb{R}^n)$. 
\end{enumerate}

Al igual que las variedades tienen subvariedades, las haces vectoriales tendr\'an subhaces definidos de una manera similar.

 \begin{defi}
  Sea $M$ una variedad suave y $E$ un haz vectorial sobre $M$, decimos que $F \subseteq E$ es un \textit{subhaz vectorial} de $E$ si $F$ es una subvariedad de $E$, $\pi \vert_F: F \to M$ cumple la definici\'on de haz vectorial y sus cartas son las restricciones can\'onicas.
\end{defi}

\begin{defi}
 Sean $\pi_E:E \to M$ y $\pi_F: F \to N$ haces vectoriales sobre $M$ y $N$, decimos que $\phi: E \to F$ es un \textit{homomorfismo de haces} si: 
\end{defi}

\begin{enumerate}
\item Existe una funci\'on suave $f:M \to N$ llamada funci\'on base tal que el siguiente diagrama conmuta   

$$\xymatrix{
E \ar[d]^{\pi_E} \ar[r]^\phi &F\ar[d]^{\pi_F}\\
M \ar[r]^f &N.} $$ 

\item $\phi$ es suave.

\item Es lineal en fibras, es decir, $\phi \vert_{E_x}: E_x \to F_{f(x)}$ es lineal para todo $x \in M$. Sabemos que este mapeo esta bien definido por $1$.
\end{enumerate}

Sea $f: M \to N$ funci\'on suave, entonces para cada $x \in M$ tenemos la transformaci\'on lineal $(df)_x : T_x M \to T_{f(y)}N$, entonces podemos definir $(df): TM \to TN$ el cual es claramento un homomorfismo con funci\'on base $f$.

\begin{prop}
Sean $\pi_E: E \to M$ y $\pi_F: F \to N$ haces vectoriales sobre $M$ y $N$, $\phi: E \to F$ un homomorfismo con funci\'on base $f$. Supongamos que $\phi_x$ tiene el mismo rango para toda $x \in M$, entonces $Ker \phi = \bigcup\limits_{x \in M} Ker \phi_x$ es un subhaz vectorial de $E$.
\end{prop}

Esta proposici\'on no la demostraremos pero nos ser\'a \'util en el futuro. Para finalizar esta secci\'on daremos la definici\'on  de vecindad tubular y enunciaremos un teorema de existencia.

\begin{defi}
Sea $M$ variedad con $N \subset M$ subvariedad. Una \textit{vecindad tubular} de $N$ en $M$ es un abierto $L$ de $M$ con una sumersi\'on $\pi: L \to N$ tal que
\begin{enumerate}
\item $L$ es un haz vectorial sobre $N$ con proyecci\'on $\pi$.
\item $N \subset L$ es la imagen de la secci\'on cero de este haz.
\end{enumerate}
\end{defi}

\begin{theorem}
Sea $M$ variedad y $N \subset M$ subvariedad, entonces existe una vecindad tubular de $N$.
\end{theorem}

Como le pedimos a la vecindad tubular que sea un abierto de $M$ podemos notar que la dimensi\'on de las fibras de este haz ser\'a la codimensi\'on de la subvariedad.

\section{Haces fibrados}

Cuando tenemos $\pi: E \to M$ un haz vectorial sobre $M$, cada fibra $\pi^{-1} (x)$ es un espacio vectorial de dimensi\'on finita, pero un espacio vectorial de dimensi\'on finita al mismo tiempo es una variedad suave, entonces podemos cambiar nuestra definici\'on de haz vectorial y en un lugar de poner espacio vectorial lo podemos cambiar por variedad.

\begin{defi}
Sean $E$, $F$ y $M$ variedades, $\pi: E \to M$ una sumersi\'on. Entonces $E$ es un \textit{haz fibrado} sobre $M$ si para todo $x \in M$, existe una vecindad $U_x$ y un difeomorfismo $\phi_{U_x}: \pi^{-1}(U_x) \to U_x \times F$ tal que el siguiente diagrama conmuta 

$$\xymatrix{
\pi^{-1} (U_x)  \ar[r]^{\phi_{U_x}} \ar[rd]^\pi &U_x \times F \ar[d]^{\pi_{U_x}}\\
 &U_x}$$ donde $\pi_{U_x}$ es la proyecci\'on can\'onica en el primer factor.
\end{defi}

Podemos notar que cada fibra de $\pi$ es difeomorfa a $F$, si restringimos $\phi_{U_x}$. A diferencia de los haces vectoriales aqu\'i no tenemos necesariamente un punto distinguido en $F$, entonces, no necesariamente van a existir secciones de $E$.

Como ejemplos tenemos a los haces vectoriales y a los haces fibrados producto, es decir, nos tomamos dos variedades $M$ y $F$, entonces el producto $M \times F$ con la proyecci\'on $\pi: M \times F \to M$ es un haz fibrado sobre $M$. En este trabajo estaremos trabajando con un haz fibrado muy particular que definiremos a continuaci\'on.

\begin{defi}
Sean $M$ y $N$ variedades suaves, $x \in M$. Supongamos que $f,g: M \to N$ son funciones suaves, con $f(x) = g(x) = y$.
\begin{enumerate}
\item Decimos que $f$ tiene \textit{contacto de primer orden} con $g$ en $x$ si $(df)_x = (dg)_x$ como transformaciones lineales de $T_x M \to T_y N$.
\item Decimos que $f$ tiene \textit{contacto de orden $k$} con $g$ en $x$ si $(df): TM \to TN$ tiene contacto de orden $(k-1)$ con $(dg)$ en todo punto de $T_x M$. Esto lo vamos a denotar como $f \sim_k g$ en $x$, ocupamos esta notaci\'on ya que \'esta es, obviamente, una relaci\'on de equivalencia sobre el conjunto de funciones suaves de $M$ en $N$ tales que $f(x) =y$.

\item Sea $J^k (M,N)_{x,y}$ el conjunto de clases de equivalencia bajo la relaci\'on $\sim_k$ en $x$. 

\item Sea $J^k (M,N) := \bigcup\limits_{ (x,y) \in M \times N} J(M,N)_{x,y}$. A un elemento $\sigma$ de este conjunto le llamaremos $k$-jet de mapeos (o $k$-jet) de $M$ en $N$.

\end{enumerate}


\end{defi}

Nuestra intenci\'on es demostrar que $J^k (M,N)$ es una variedad diferenciable y un haz fibrado sobre $M$ para toda $k \in \mathbb{N}$, pero aplazaremos esta demostraci\'on para notar algunos hechos y demostrar unos lemas que nos ayudar\'an con esto. 


Sea $\sigma$ un $k$-jet, entonces existen $x \in M$ y $y \in N$ tales que $\sigma \in J^k (M,N)_{x,y}$, a $x$ lo llamaremos la fuente de $\sigma$ y a $y$ el blanco de $\sigma$. Entonces tenemos dos mapeos, $\alpha: J^k (M,N) \to M$ que a cada $k$-jet le asignamos su fuente y $\beta:J^k (M,N) \to N$ que a cada $k$-jet le asigna su blanco. Es claro que $\alpha$ y $\beta$ est\'an bien definidas por el inicio de este p\'arrafo. A partir de este momento usaremos las letras $\alpha$ y $\beta$ exclusivamente para estos mapeos.

Dada una funci\'on $f:M \to N$, \'esta nos induce un mapeo $j^k (f) : M \to J^k(M,N)$ llamado el $k$-jet de $f$ que a cada $x \in M$ le asigna la clase de $f$ en $J^k (M , N )_{x , f(x)}$. En caso de que $J^k (M,N)$ fuera una variedad diferenciable, nos gustar\'ia que los mapeos $\alpha$, $\beta$ y $j^k (f)$ fueran diferenciables, ya que definirlos probablemente ser\'ia una p\'erdida de tiempo si no lo fueran. Cuando $k=0$ a $J^0 (M,N)$ lo podemos identificar con $M \times N$, a cada $0$-jet le asignamos su fuente y su blanco en $M \times N$, es decir, $j^0 (f) (x) = (x, f(x))$.


\begin{lem}
Sea $U$ un abierto de $\mathbb{R}^n$ y $y \in U$. Sean $f,g : U \to \mathbb{R}^m$ suaves. Entonces $f \sim_k g$ en $y \iff$ $$ \frac{\partial^{\vert \gamma \vert} f_i }{\partial x^\gamma} (y) = \frac{\partial^{\vert \gamma \vert} g_i }{\partial x^\gamma} (y)$$
para todo multi\'indice $\gamma$ con $\vert \gamma \vert \leq  k$ y $1 \leq i \leq m$. Donde $f_i$ y $g_i$ son las funciones coordenadas de $f$ y $g$ respectivamente y $x_i$ las coordenadas en $U$.
\end{lem}

\underline{Demostraci\'on:} Para $k=1$, $f \sim_1 g$ en $y \iff (df)_y = (df)_y$ como transformaciones lineales, pero esto nos dice que las entradas de las matrices son exactamente las mismas, por lo tanto, las derivadas parciales de primer orden de $f$ y $g$ en $y$ son iguales.

Supongamos que el lema es cierto para $k-1$. Sean $\overbar{x_1}, \dots , \overbar{x_n}$ las coordenadas correspondientes a $\mathbb{R}^n$ en $U \times \mathbb{R}^n = TU$ (los haces tangentes de abiertos de $\mathbb{R}^n$ son siempre triviales). Con estas coordenadas podemos ver a $(df)_y: U \times \mathbb{R}^n \to \mathbb{R}^m \times \mathbb{R}^m$ est\'a dada por $(x, \overbar{x}) \to (f(x) , \overbar{f_1} (x , \overbar{x}) , \dots , \overbar{f_m} (x ,\overbar{x} ))$ donde $$ \overbar{f_i} ( x , \overbar{x}) = \sum\limits_{j=1}^n \frac{\partial f_i}{\partial x_j} (x) \overbar{x_j} .$$

Podemos hacer lo mismo para $(dg)$ (cuando escribimos $\overbar{f_i}$ nos estamos refiriendo a las funciones coordenadas de $\overbar{f_i}$, lo mismo para $\overbar{x}$).

Por hip\'otesis de inducci\'on $(df) \sim_{k-1} (df)$ para todo punto $(y , v) \in  \{ y \} \times \mathbb{R}^n$ lo cual implica que las derivadas de $(df)$ y $(dg)$ son iguales en estos puntos. Sea $\gamma$ un multi\'indice con $\vert \gamma \vert \leq k-1$ entonces
$$\frac{\partial^{\vert \gamma \vert} \overbar{f_i}}{\partial x^\gamma} (y , v) = \frac{\partial^{\vert \gamma \vert} \overbar{g_i}}{\partial x^\gamma} (y,v) .$$
Evaluamos en $v_j = (0 , \dots , 1 , \dots , 0) \in \mathbb{R}^n$ con $1$ en la j-\'esima coordenada, entonces 
$$\frac{\partial^{\vert \gamma \vert} \partial f_i }{\partial x^\gamma \partial x_j} (y ) = \frac{\partial^{\vert \gamma \vert} \partial g_i}{\partial x^\gamma \partial x_j} (y).$$
Este proceso lo podemos hacer para cada $j$ tal que $1 \leq j \leq n$, y obtenemos lo deseado. La otra implicaci\'on es obvia.
$\blacksquare$

De la prueba anterior tenemos el siguiente corolario:
\begin{coro}
$f,g: U \to \mathbb{R}^m$ contacto de orden $k$ en $y \in U$ si y s\'olo si los polinomios  de Taylor de $f$ y $g$ en $y$ de grado k son iguales.
\end{coro} 

Tambi\'en como consecuencia del lema anterior y de la regla de la cadena tenemos:

\begin{coro}
Sean $U \subseteq \mathbb{R}^n$ abierto y $V\subseteq \mathbb{R}^m$ abierto. Sean $f_1 , f_2 : U \to V$ y $g_1, g_2 : V \to \mathbb{R}^l$ suaves. Supongamos que existe  $x \in U$ tal que $f_1 \sim_k f_2 $ en $x$ y $g_1 \sim_k g_2 $ en $f_1(x) =f_2 (x)$, entonces $g_1 \circ f_1 \sim_k g_2 \circ f_2$ en $x$.
\end{coro}

Como consecuencia de estos dos corolarios tenemos:
\begin{prop}
Sean $M_1 , M_2 , M_3 , M_4 $ variedades suaves y $k \in \mathbb{N}$.
\begin{enumerate}
\item Sea $h: M_2 \to M_3$ suave, entonces $h$ induce un mapeo $h_\ast: J^k (M_1,M_2) \to J^k (M_1, M_3)$ definido de la siguiente manera; sea $\sigma \in J^k (M_1, M_2 )_{x , y}$ y $f$ un representante de $\sigma$, entonces $h_\ast (\sigma) = \overbar{ h \circ f (x)} \in J^k (M_1, M_3)_{x , h(y)}$.

\item Sea $\gamma: M_3 \to M_4$ suave. Entonces $\gamma_\ast \circ h_\ast = ( \gamma \circ h )_\ast$ como mapeos de $J^k (M_1 , M_2) \to J^k (M_1 , M_4)$ y $(id_{M_2})_\ast = id_{J^k (M_2, M_2)}$. Entonces si $h$ es un difeomorfismo, $h_\ast$ ser\'a una biyecci\'on.

\item Sea $g: M_3 \to M_1$ difeomorfismo, entonces $g$ induce un mapeo $g^\ast : J^k (M_1, M_2) \to J^k (M_3. M_2)$ definido de la siguiente manera, sea\\ $\tau \in J^k (M_1, M_2)_{x , y}$ y $f$ un representante de $\tau$. Entonces $g^\ast ( \tau ) = $\\$\overbar{f \circ g ( g^{-1} (x))} \in J^k ( M_1,M_3)_{g^{-1} (x) , y}$.

\item Sea $\xi: M_4 \to M_3$ difeomorfismo. Entonces $\xi^\ast \circ g^\ast = ( g \circ \xi )^\ast $ como mapeos de $J^k( M_1 , M_2) \to J^k (M_4, M_2)$ y $(id_{M_1})^\ast = id_{J^k( M_1 ,M_1)}$ tal que $g^\ast$ es una biyecci\'on.
\end{enumerate}
\end{prop}

\underline{Demostraci\'on:}
Para demostrar 1 y 3, ocupamos el \'ultimo corolario para ver que est\'an bien definidas. Para 2 y 4 ocupamos el hecho de que los difeomorfismos tienen inversa suave. $\blacksquare$

Sea $\mathbb{R}_k [x_1, \dots, x_n]$ el espacio vectorial de polinomios en n variables de grado menor o igual que k que tienen t\'ermino constante $0$ sobre $\mathbb{R}$. Este es un espacio vectorial de dimensi\'on finita, entonces le podemos dar estructura de variedad diferenciable. Sea $V_{n,m}^k = \bigoplus\limits_{i=1}^m \mathbb{R}_k [x_1, \dots, x_n]$. Como tenemos un n\'umero finito de sumandos, este vuelve a ser un espacio vectorial de dimensi\'on finita, de la misma manera es una variedad diferenciable. 

Sea $U \subseteq \mathbb{R}^n$ abierto y $f: U \to \mathbb{R}$ suave. Definimos $p_k (f) : U \to \mathbb{R}_k [x_1, \dots, x_n]$, que a cada $x_0$ nos lo manda a los primeros $k$ t\'erminos del polinomio de Taylor de $f$ centrado en $x_0$ sin contar el t\'ermino constante.

\begin{lem}
Sea $V \subseteq \mathbb{R}^m$ abierto, entonces hay una biyecci\'on $\phi_{U,V}^k: \\ J^k(U,V) \to U \times V \times \mathbb{R}_k [x_1, \dots, x_n]$ donde $$\phi_{U,V}^k ( \sigma) = (\alpha (\sigma) , \beta (\sigma) , p_k (f_1) (x_0) , \dots , p_k (f_m) (x_0) )$$
con $f:U \to V$ suave representante de $\sigma$.
\end{lem}

Por lo que demostramos anteriormente la funci\'on est\'a bien definida y es inyectiva, la suprayectividad es a\'un m\'as clara. $\blacksquare$

\begin{lem}
Sean $U$ y $U^\prime$ abiertos de $\mathbb{R}^n$, $V$ y $V^\prime$ abiertos de $\mathbb{R}^m$. Sean $h:V \to V^\prime$ y $g: U \to U^\prime$ suaves con $g$ un difeomorfismo. Entonces $$ \phi_{{U^\prime} , V^\prime}^k \circ (g^{-1})^\ast h_\ast \circ (\phi_{U , V}^k)^{-1} : U \times V \times V_{n, m }^k \to U^\prime \times V^\prime \times V_{n,m}^k$$
es suave.


\end{lem}
\underline{Demostraci\'on:} Sea $p = (x_0 , y_0 , f_1 (x_0) , \dots , f_m (x_0))$ con $f_i \in \mathbb{R}_k [x_1, \dots, x_n]$, $1 \leq i \leq m$. Sea $f: U \to \mathbb{R}^m$, dada por $f(x) = y_0 + (f_1 (x - x_0) ,  \dots , f_m (x - x_0))$. Entonces $f(x_0) = y_0$ ya que cada $f_i \in \mathbb{R}_k [x_1, \dots, x_n]$. Sea $\sigma$ la clase de equivalencia en $J^k (U,V)_{x_0 , y_0}$, entonces $\phi_{U.V}^k (\sigma) = p$. Por definici\'on de  $(g^{-1})^\ast$ y $h_\ast$ tenemos que $(g^{-1})^\ast \circ h_\ast (\sigma) = j^k (h \circ f \circ g^{-1}) (x_0)$.

Entonces $$\phi_{U^\prime , V^\prime}^k \circ (g^{-1})^\ast \circ h_\ast \circ (\phi_{U, V}^k)^{-1} ( p) = \phi^k_{U^\prime , V^\prime} (j^k (h \circ f \circ g^{-1} ) (x_0)) =$$ $$(g(x_0) , h(y_0), p_k (h \circ f \circ g^{-1} )_1 (g(x_0)) , \dots , p_k (h \circ f \circ g^{-1} )_m (g(x_0)),$$
donde $(h \circ f \circ g^{-1})_i$ son las funciones coordenadas de la composici\'on. Para demostrar que el mapeo es suave, nada m\'as tenemos que demostrar que las funciones coordenadas son suaves, es claro que las dos primeras coordenadas del mapeo son suaves ya que la composici\'on de mapeos suave es suave. Entonces para ver que es suave, nada m\'as nos hace falta restringir el codominio a $V_{n,m}^k$. 
Sea $\delta = (h \circ f \circ g^{-1})$. Entonces $$p_k (\delta_i) (g(x_0)) = \sum\limits_{1 \leq \vert \gamma \vert \leq k} \frac{\partial ^{\vert \gamma \vert } \delta_i}{\partial x^{\vert \gamma \vert}} (g(x_0)) (x - x_0)^\gamma .$$

Como la $\delta_i$ depende de las derivadas parciales de $g$, $h$ y $f$, \'esta ser\'a suave para todo multi\'indice $\gamma$ y para cada $i$. $\blacksquare$

Como las variedades son localmente euclidianas podemos notar que lo que estamos intentando hacer es que si las cartas de $J^k (M, N)$ son de la forma $J^k (U,V)$, con $U$ y $V$ cartas de $M$ y $N$ respectivamente, entonces $J^k(M,N)$ ser\'ia localmente euclidiano.

\begin{theorem}
Sean $M^m$, $N^n$ y $L^l$ variedades, entonces
\begin{enumerate}
\item $J^k (M , N)$ es una variedad de dimensi\'on $m + n + \dim (V_{n , m}^k)$.
\item $\alpha: J^k (M, N) \to M$, $\beta: J^k (M , N) \to N$ y $\alpha \times \beta: J^k (M , N) \to M \times N$ son sumersiones (las sumersiones son suaves).
\item Si $h: N \to L$ suave, entonces $h_\ast: J^k (M , N ) \to J^k (M, L)$ es suave. Si $g : M \to N$ es un difeomorfismo entonces $g^\ast : J^k (N, L) \to J^k ( M, L) $ es un difeomorfismo.
\item Si $g: M \to N$ suave, entonces $j^k (g): M \to J^k (M,N)$ es suave.
\end{enumerate}
\end{theorem}
\underline{Demostraci\'on:}
\begin{enumerate}
\item Es claro que $J^k (M, N)$ es segundo numerable y adem\'as es Hausdorff por ser uni\'on ajena de Hausdorff indexados por un espacio Hausdorff. Sea $U_1$ dominio de una carta $\xi_1$ en $M$ y $V_1$ dominio de una carta $\psi_1$ en $N$. Sea $U_1^\prime = \xi_1 (U_1)$ y $V_1^\prime = \psi_1 (V_1)$. Entonces tenemos el mapeo $(\xi^{-1})^\ast \circ \psi_\ast: J^k (U, V) \to J^k( U^\prime , V^\prime)$. Sea $\tau_{U,V} := \phi_{U^\prime , V^\prime}^k \circ ( \xi^{-1})^\ast \circ \psi_\ast: J^k (U, V) \to U^\prime  \times V^\prime \times V_{n , m}^k$, las cartas de $J^k (M, N)$ las definimos como las $\tau_{U,V}$, para ser variedad diferenciable nos faltar\'ia probar que los cambios de cartas son diferenciables, pero esto es una simple consecuencia del lema anterior.

Para calcular la dimensi\'on nada m\'as nos hace falta calcular la dimensi\'on de $V_{n , m}^k$. Los polinomios homog\'eneos de grado 0 tienen dimensi\'on 1, los de grado 1 tienen dimensi\'on $n= \begin{pmatrix} n + 1 - 1 \\ 1 \end{pmatrix}$. Supongamos que la dimensi\'on de los polinomios homog\'eneos de grado $k$ es $\begin{pmatrix} n + k - 1 \\ k \end{pmatrix}$. Ahora calculamos la dimensi\'on de los de grado $k+1$, a $x_n^0$ le corresponde un espacio de dimensi\'on $\begin{pmatrix} n + k -1 \\ k \end{pmatrix}$, a $x_n^1$ uno de dimensi\'on $\begin{pmatrix} n + k - 2 \\ k -1 \end{pmatrix}$, as\'i hasta $x_n^{k+1}$ al cual le corresponde uno de dimensi\'on $\begin{pmatrix} n \\ 0 \end{pmatrix}$. Entonces 
$$\begin{pmatrix} n \\ 0 \end{pmatrix} + \begin{pmatrix} n \\ 1 \end{pmatrix} + \dots + \begin{pmatrix} n + k - 2 \\ k-1 \end{pmatrix} + \begin{pmatrix} n+k-1 \\ k \end{pmatrix} $$
$$=\begin{pmatrix} n + 1 \\ 1 \end{pmatrix} + \begin{pmatrix} n + 1 \\ 2 \end{pmatrix} + \dots + \begin{pmatrix} n + k -1 \\ k \end{pmatrix} $$
$$= \dots = \begin{pmatrix}
n + k \\ k
\end{pmatrix}.$$

Entonces la dimensi\'on de $V_{n,m}^k $ es $$1 - 1 + \begin{pmatrix} n \\ 1 \end{pmatrix} + \begin{pmatrix} n +1 \\ 2 \end{pmatrix} \dots +\begin{pmatrix} n + k -1 \\ k \end{pmatrix} $$
$$= \begin{pmatrix} n + 1 \\ 1 \end{pmatrix} + \begin{pmatrix} n + 1 \\ 2 \end{pmatrix} + \dots + \begin{pmatrix} n + k - 1 \\ k \end{pmatrix} - 1 $$
$$= \dots = \begin{pmatrix} n + k \\ k \end{pmatrix} - 1 .$$

\item En coordenadas locales $\alpha $ y $\beta$ son la proyecci\'on en el primer y segundo factor respectivamente, entonces son sumersiones. De la misma manera $\alpha \times \beta$ tambi\'en lo es.

\item Que una funci\'on sea suave en un punto es una condici\'on local, entonces por como definimos las cartas de $J^k (M,N)$ y por el lema anterior, $h_\ast$ es suave, de la misma manera $g^\ast$ ser\'a suave, y es un difeomorfismo ya que existe $g^\prime$ funci\'on inversa de $g$ que es un difeomorfismo, lo cual implica que $(g^\prime)^\ast$ es suave, por lo tanto $(g^\prime)^\ast$ es la funci\'on inversa de $g^\ast$, por lo tanto $g^\ast$ es un difeomorfismo.

\item Procedemos de la misma manera, como la suavidad de una funci\'on es una condici\'on local, por como definimos nuestras cartas y por el lema anterior $j^k(g)$ es suave. $\blacksquare$

\end{enumerate}

\begin{coro}
Sean $M$ y $N$ variedades, $k, l \in \mathbb{N}$ con $k > l$. Entonces existe una proyecci\'on can\'onica suave $\pi_{k , l}: J^k (M , N) \to J^l (M, N)$ que a cada $k$-jet nos lo manda a su clase de $l$-jet.
\end{coro}
\underline{Demostraci\'on:} Por como definimos nuestras cartas lo que hace $\pi_{k, l}$ es nada m\'as olvidarse de los t\'erminos de orden $> l$ de cada k-jet, es decir, en coordenadas locales es una proyecci\'on, por lo tanto es suave. $\blacksquare$

Por como definimos las cartas es claro que $J^k( M,N)$ es un haz fibrado sobre $M$, en general no ser\'a un haz vectorial, pero si $N \cong \mathbb{R}^n$ para alguna $n \in \mathbb{N}$, entonces $J^k(M,N)$ s\'i ser\'a un haz vectorial sobre $M$. Tambi\'en es claro que $J^1 (M, N) \cong \hom (TM,TN)$. Finalmente, por nuestro \'ultimo corolario, si $k >l $ entonces $J^k (M,N)$ es un haz fibrado sobre $J^l (M,N)$ con proyecci\'on $\pi_{k, l}$.

\chapter{Topolog\'ia de Whitney}

Cuando definimos una topolog\'ia en un conjunto $X$, estamos definiendo un concepto de ``cercan\'ia'' , por ejemplo, estamos diciendo cuales sucesiones convergen o no. En el \'area de an\'alisis, al espacio de funciones entre dos espacios m\'etricos le damos la topolog\'ia del supremo, en la cual dos funciones est\'an cerca si y s\'olo si sus im\'agenes est\'an cerca. En nuestro caso de funciones entre variedades diferenciables tenemos m\'as estructura, ya que no s\'olo vamos a querer que las im\'agenes de dos funciones est\'en cerca, sino tambi\'en que sus derivadas est\'en cerca. En este cap\'itulo definiremos la topolog\'ia de Whitney en $C^\infty (M,N)$ con la ayuda de los haces de $k$-jets y desarrollaremos las propiedades m\'as elementales de este espacio.
\section{Definiciones}

\begin{defi}
Sean $M$ y $N$ variedades suaves.
\begin{enumerate}
\item Sea $k \in \mathbb{N}$. Sea $U \subseteq J^k (M,N)$, definimos $P^k(U) \subseteq C^\infty (M,N)$ de la siguiente manera: $$P^k (U) := \{ f \in C^\infty (M, N) \mid j^k (f) (M) \subset U \} .$$

Es claro que $P^k(U) \cap P^k (V) = P^k (U \cap V)$. Si el contexto nos lo permite omitiremos la $k$ de $P^k(U)$.

\item La familia de subconjuntos $\{ P^k (U) \}$ con $U \in J^k (M,N)$ abierto forman una base de una \'unica topolog\'ia en $C^\infty$. Esta topolog\'ia se llama la \textit{topolog\'ia $C^k$ de Whitney}. A los abiertos de esta topolog\'ia los denotamos como $W_k$. 
\item La \textit{topolog\'ia $C^\infty$ de Whitney (o simplemente topolog\'ia de Whitney}) en $C^\infty (M, N)$ es la topolog\'ia cuya base es $W= \bigcup\limits_{k=0}^\infty W_k$. \'Esta es una base bien definida ya que $W_l \subset W_k$ cuando $k > l$, ya que el mapeo $\pi_{k,l}: J^k (M,N) \to J^l (M,N)$ es una sumersi\'on.
\end{enumerate}
\end{defi}
En nuestra definici\'on de esta topolog\'ia nunca le pedimos nada a $M$ y a $N$ (s\'olo que sean variedades diferenciables), pero si $M$ es compacta obtendremos una topolog\'ia muy distinta a cuando $M$ no lo es (para nuestra suerte en el futuro solo trabajaremos cuando $M$ es compacta).

Como $J^k (M, N) $ es una variedad, existe una m\'etrica $d$ compatible con la topolog\'ia. Sea $f \in C^\infty (M, N)$, definimos $$B_\delta (f) := \{ g \in C^\infty (M , N) \mid \forall x \in M, \, d(j^k f(x) , j^k g(x)) < \delta (x) \} , $$ donde $\delta : M \to \mathbb{R}^+$ es una funci\'on continua. Entonces si probamos que $B_\delta (f)$ es abierto en $J^k (M, N)$ para toda $\delta$ de este estilo tendremos que $P(B_\delta (f))$ es una vecindad abierta de $f$ en $C^\infty$. 

Sea $\gamma: J^k (M,N) \to \mathbb{R}$ definida como $\gamma (\sigma ) = \delta ( \alpha (\sigma) )- d(j^k f(\alpha(\sigma), \sigma ))$. Este es un mapeo continuo, ya que $\gamma$ en $M$ es continua y los otros mapeos son diferenciables. Sea $U = \gamma^{-1} (0, \infty)$ abierto en $J^k(M,N)$, y por como definimos $B_\delta (f)$ tenemos que $B_\delta (f)= P(U)$. Ahora, si tenemos $W \subseteq C^\infty (M,N)$ vecindad de $f$ demostraremos que existe $\delta: X \to \mathbb{R}$ tal que $W = B_\delta (f)$.

Como $W$ es vecindad de $f$, existe $V \subseteq C^\infty (M, N)$ abierto tal que $ f \in P(V) \subseteq W$, sea $m(x) = \inf \{ d (\sigma , j^k f(x)) \mid \sigma \in \alpha^{-1} (x) \cap (J^k (M,N) - V) \}$ (la $x$ est\'a fija). Notamos que $m(x) = 0$ si $\alpha^{-1} (x) \subset V$. Sea $\gamma: X \to \mathbb{R}^+$ funci\'on continua tal que $\delta (x) < m(x) $  $\forall x \in M$. Como $m$ alcanza su \'infimo en cualquier subconjunto compacto de $M$, entonces podemos definir $\delta$ localmente para que cumpla lo que queremos, luego con un argumento de particiones de la unidad la podemos definir globalmente (cualquier variedad es uni\'on numerable de compactos). Entonces $B_\delta (f) \subseteq W$. De cierta manera estamos ``metrizando'' las vecindades de $f$, aunque $C^\infty (M,N)$ no sea necesariamente  un espacio m\'etrico.

Si tenemos $\delta , \eta : M \to \mathbb{R}^+$ continuas, sea $\gamma: M \to \mathbb{R}^+$ definida como $\gamma (x) = \inf \{ \delta (x) , \eta (x) \}$, como la funci\'on $\inf$ es continua entonces $\gamma$ es continua, y adem\'as $B_\gamma (f) = B_\delta (f) \cap B_\eta (x)$. Entonces $ \{ B_\delta (f) \}$ son una base de vecindades de $f$ en la topolog\'ia $C^k$ de Whitney en $C^\infty$.

Toda funci\'on de un compacto en $\mathbb{R}^+$ est\'a acotada por abajo por la funci\'on constante $\frac{1}{n}$ para alguna $n\in \mathbb{N}$, entonces cuando $M$ es compacta tenemos que para toda $\delta$ existe una $n$ que cumple lo anterior, es decir, $B_{\frac{1}{n}} (f) \subset B_\delta (f)$, entonces acabamos de probar el siguiente teorema.

\begin{theorem}
Sean $M$ y $N$ variedades con $M$ compacta, entonces $C^\infty (M,N)$ es primero numerable y la base de vecindades para $f \in C^\infty(M,N)$ estar\'a dada por $ \{ B_{\frac{1}{n} } (f) \}_{n \in \mathbb{N}} $.
\end{theorem}

\begin{coro}
Sean $M$ y $N$ variedades con $M$ compacta, $\{ f_n \}_{n \in \mathbb{N}} \subset \\ C^\infty (M,N)$, entonces $\{ f_n \}_{n \in \mathbb{N}} $ converge en $C\infty(M,N)$ en la topolog\'ia $C^k$ si y s\'olo si $j^k f_n$ converge uniformemente a $j^k$.

\end{coro}

Esto es una consecuencia simple de como definimos nuestras bolas en \\\ $C^\infty (M,N)$. Como todas las vecindades de una funci\'on $f \in C^\infty (M,N)$ son de la forma $B_\delta (f)$ con $\delta : M \to \mathbb{R}^+$ continua, entonces cuando $M$ no es compacta $C^\infty(M,N)$ aparentemente no ser\'a primero numerable ya que ocupamos fuertemente la compacidad de $M$ en la prueba anterior.
\begin{theorem}
Sean $M$ y $N$ variedades, $\{ f_n \}_{n \in \mathbb{N}} \subset C^\infty (M,N)$. La sucesi\'on $ \{ f_n \}_{n \in \mathbb{N}}$ converge a $f$ en la topolog\'ia $C^k$ si y s\'olo si existe  un compacto $K \subset M$ tal que $j^k f_n$ converge uniformemente a $j^k f$ en $K$ y nada m\'as un n\'umero finito de $f_n$ son diferentes a $f$ fuera de $K$.
\end{theorem}
 
\underline{Demostraci\'on:} La suficiencia es obvia, as\'i que nada m\'as tenemos que demostrar la necesidad. Lo haremos por contradicci\'on, supongamos que $\{ f_n \}_{n \in \mathbb{N}}$ converge a $f$ pero no existe $K$ con la propiedad deseada. Sean $ \{ K_i \}_{i \in \mathbb{N}} \subset M$ una sucesi\'on de compactos tales que $K_i \subset \inte K_{i+1}$ y $M = \bigcup\limits_{i=1}^\infty K_i$. Existe una funci\'on $f_{n_1}$ distinta a $f$. Entonces existe $x_{n_1} \in M$ tal que $d(j^k f_{n_1} (x_{n_1}) , j^k f (x_{n_1}) = a_{n_1} >0$, entonces existe $m_1$ tal que $x_{n_1} \in K_{m_1}$. Sea $\delta_1: M \to \mathbb{R}^+$ tal que $\delta_1 (K_{m_1}) \equiv a_{n_1}$. Inductivamente elegimos $ \{ f_{n_i} \}_{i=1}^s$ para alguna $s \in \mathbb{N}$, que para cada $s$ existir\'a una $x_{n_s} \in K_{m_s}$ y una funci\'on continua $\delta_s: X \to \mathbb{R}^+$ tales que $d(j^k f_{n_s} (x_{n_s}) , j^k f (x_{n_s}) > \delta_s (x_{n_s})$.

Ahora elegimos $f_{n_{s+1}}$ tal que sea distinta a $f$ fuera de alg\'un compacto $K_{m_s+1}$. Sea $x_{n_{s+1}} \notin K_{m_{s+1}}$ con $d(j^k f_{n_{s+1}} (x_{s+1} ) , j^k f(x_{n_{s+1}})) = a_{n_{s+1}} >0$. Entonces $x_{n_{s+1}} \in K_{m_{s+1}}$ para alg\'un $m_{s+1}$, y elegimos $\delta_{s+1}$ de la misma manera. Elegimos una partici\'on de la unidad asociada a nuestra cubierta de $K_i$ y ``pegamos" las $\delta_s$ en una funci\'on $\delta: M \to \mathbb{R}^+$ continua, que vale $a_{n_{s+1}} \in (K_{m_{s+1}} - K_{m_s +1} )$. Entonces estamos construyendo una subsucesi\'on $\{ f_{n_i} \}_{i \in \mathbb{N}}$ y una funci\'on $\delta$ tal que para toda $i \in \mathbb{N}$, $f_{n_i} \notin B_\delta (f)$, es decir, $ \{ f_{n_i} \}_{i \in \mathbb{N}}$ no converge a $f$. $\blacksquare$

\begin{coro}
Si $M$ no es compacta entonces $C^\infty (M,N)$ no es primero numerable.
\end{coro}
\underline{Demostraci\'on:} Supongamos que si es primero numerable. Sea $f \in C^\infty (M,N)$ y $\{ W_i \}_{i \in \mathbb{N}}$ una base de vecindades de $f$. Para cada $i \in \mathbb{N}$ elegimos $\delta_i: M \to \mathbb{R}^+$ tal que $B_{\delta_i} (f) \subseteq W_i$ y una sucesi\'on de puntos $\{ x_i \}_{i \in \mathbb{N}}$ que no tenga punto l\'imite. Elegimos una funci\'on $\delta: M \to \mathbb{R}^+$ tal que $\delta (x_i) < \delta_i (x_i)$ para todo $i \in \mathbb{N}$, podemos construir a $\delta$ localmente y luego la podemos extender con una partici\'on de la unidad. Como $\{ W_i \}_{i \in \mathbb{N}}$ es una base de vecindades de $f$ entonces existe $m \in \mathbb{N}$ tal que $W_m \subset B_\delta (f)$ lo cual implica que $B_{\delta_m} (f) \subset B_\delta(f)$. Esto es una contradicci\'on, ya que construimos $\delta$ para que esto no pasara. $\blacksquare$.

\begin{defi}
Sea $X$ un espacio topol\'ogico. Decimos que
\begin{enumerate}
\item $Y \subseteq X$ es \textit{residual} si es la intersecci\'on numerable de abiertos densos de $X$.
\item $X$ es un \textit{espacio de Baire} si todo subconjunto $Y$  residual es denso en $X$.
\end{enumerate}

\end{defi}

\begin{prop}
Sean $M$ y $N$ variedades, entonces $C^\infty(M,N)$ es un espacio de Baire con la topolog\'ia $C^\infty$ de Whitney.
\end{prop}
\underline{Demostraci\'on:} Para $k \in \mathbb{N}$ elegimos una m\'etrica $d_k$ para $J^k (M , N)$. Sean $\{ U_i \}_{i \in \mathbb{N}} \subset C^\infty(M,N)$ abiertos densos y $V \subset C^\infty (M,N)$ abierto. Si $C^\infty (M, N) $ fuera un espacio de Baire, entonces $\bigcap\limits_{i \in \mathbb{N}} U_i$ ser\'ia denso si y s\'olo si $V \cap   \left(\bigcap\limits_{i \in \mathbb{N}} U_i \right) \neq \emptyset$ ya que $V$ es arbitrario.

Como $V$ es abierto entonces existe $k_0$ y $W \subset J^{k_0} (M, N)$ abierto tal que $P (\overbar{W}) \subset V$ con $P(W) \neq \emptyset$. Entonces en realidad vamos a demostrar que $M(\overbar{W}) (\bigcap\limits_{i \in \mathbb{N}} U_i ) \neq \emptyset$, lo cual obviamente implica la afirmaci\'on.

Sean $\{ f_i \}_{i \in \mathbb{N}} \subset C^\infty (M,N)$, para cada $i \in \mathbb{Z}$ elegimos $W_i \subset J^{k_i} (M,N)$ abierto que satisfaga:
\begin{enumerate}
\item $f_i \in P(W) \cap \left( \bigcap\limits_{j=1}^{i-1} M(W_j) \right) \cap U_i$,
\item $P (\overbar{W_i}) \subseteq U_i $ y $f_i \in P (W_i)$ y
\item $(i > 1)$ $d_s (j^s f_i (x), j^s f_{i-1} (x) ) < 2^{-1}$ para todo $x \in M$ y $1 \leq s \leq i$.
\end{enumerate}

Supongamos que esto es cierto. Definimos $g^s (x) = \lim_{i \to \infty} j^s f_i (x)$. Como $d_s$ convierte a $J^s (X,Y)$ en un espacio m\'etrico completo y por 3 $\{ j^s f_i  (x) \}_{i \in \mathbb{N}}$ es una sucesi\'on de Cauchy, por lo tanto $g^s$ est\'a bien definido. Como $j^0 f_i (x) = (x, f_i (x) )$ (localmente), podemos definir $g : M \to N$ tal que $g^0 (x) = (x , g(x))$. Si $g$ queda definida de esta forma, tendr\'iamos que para cada $i \in \mathbb{N}$, $f_i \in P(W)$ por 1, entonces $g = \lim_{i \to \infty} f_i \in P(\overbar{W})$. Por 2, $W_s$ fue elegido tal que $P(\overbar{W_s}) \subset U_s$ y por 1 cada $f_i$ con $i >s$ est\'a en $P(W_s)$. Entonces $g \in P(\overbar{W_s})$, como $s$ es arbitrario $g \in P (\overbar{W}) \cap ( \bigcap\limits_{s=1}^\infty U_s)$, y obtenemos el resultado.

Para ver que $g$ es suave es suficiente ver que es suave en alg\'un $x \in M$ arbitrario. Sean $x \in M$, $K \subset M$ vecindad compacta de $M$ y $L \subset N$ vecindad compacta de $g(x)$ (las variedades son localmente compactas) con $g(K) \subset L$. Podemos suponer que $K$ y $L$ se encuentran en dominios de cartas (en caso de que no lo fueran las podemos hacer m\'as pequeñas para que cumplan esto), entonces podemos suponer $K \subset \mathbb{R}^n $ y $L \subset \mathbb{R}^m$ (hacemos \'enfasis en que la diferenciabilidad de una funci\'on es una propiedad local).

Recordemos que elegimos $d_s$ m\'etrica en $J^s (M,N)$ tal que fuera compatible con la topolog\'ia, por 3, $\{ j^s f_i \}_{i \in \mathbb{N}}$ converge uniformemente a $g$ en $K$. Si le damos coordenadas a $J^s (M,N)$ entonces $j^s f_i (x)$ es simplemente el polinomio de Taylor de orden $s$, es decir, es $ \sum\limits_{b} \partial^{\vert b \vert} f_i  / \partial x^{\vert b \vert}$ con $\vert b \vert \leq s$, por 3 estas derivadas parciales convergen uniformemente en $K$. Entonces $$ \frac{\partial^{\vert b \vert} f_i}{\partial x^b} (y) = \frac{g^{\vert b \vert}}{\partial x^b} (y) \text{ para toda } y \in K \text{ y para todo } \vert b \vert \leq s.$$

Como $s$ fue arbitrario, entonces las derivadas parciales de todos los \'ordenes de $g$ existen en $x$, por lo tanto, $g$ es suave.

Sea $f_1 \in P(W) \cap U_1$, existe ya que $U_1$ es denso y $P(W)$ es abierto. Entonces satisface 1. Como $U_1$ es abierto y $f \in U_1$ elegimos $k_1$ y $W_1 \subset J^{k_1} (M,N)$ abierto, de modo que $f_1 \in P(W_1)$ y $P(\overbar{W_1}) \subset U_1$, entonces 2 se cumple y 3 por vacuidad. Supongamos que esto lo podemos hacer para $j \leq i -1$. Sea $$D_i = \{ g \in C^\infty (M,N) \mid d_s (j^s g(x), j^s f_{i-1} (x)) \leq \frac{1}{2^{i}} \text{ para } 1 \leq s \leq i $$$$\text{ y para todo }\\ x \in M \}. $$
Si $D_i \subset C^\infty (M,N)$ fuera abierto, sea $E_i := P(W) \cap (\bigcap\limits_{j=1}^{i-1} P(W_j) )\cap D_i$ ser\'ia abierto (intersecci\'on finita de abiertos es abierto). Entonces $f_{i-1} \in E_i$ por hip\'otesis y por como definimos $D_i$. Como $E_i$ es abierto (no vac\'io) y $U_i$ es denso, entonces existe $f_i \in E_i \cap U_i$. Entonces $E_i$ cumple 1 y adem\'as por como est\'a definido $D_i$ cumple 3. Entonces tenemos que demostrar que $D_i$ es abierto. Sea $$F_s =\{ g \in C^\infty (M,N) \mid d_s (j^s g(x) , j^s f_{i-1} (x) ) < \frac{1}{2^{i}}, \quad \forall x \in M \}.$$
Entonces $D_i= \bigcap\limits_{j=1}^{i} F_j$, si cada $F_j$ fuera abierto, entonces $D_i$ ser\'a abierto por ser intersecci\'on finita de abiertos. Sea $B_x = \alpha^{-1} (x) \cap B(\frac{1}{2^{i}} , j^s f_{i-1} (x))$ (recordemos que $\alpha$ es el mapeo fuente) con:

$$B( 2{-1} , j^s f_{i-1} (x)) := \{ \sigma \in J^s (M,N) \mid  d_s (\sigma , j^s f_{i-1} (x)) < 2^{-1} \} .$$

Sea $G := \bigcup\limits_{x \in M} B_x$. Por como definimos $B_x$, tenemos que $F_s = P(G)$, entonces nuestro problema se redujo a demostrar que $G$ es abierto en $J^s (M,N)$. Sea $\sigma \in G$ y $x_\sigma = \alpha (\sigma)$. Sea $\psi: M \to \mathbb{R}$ definido como $\psi (q) = d_s (j^s f_{i-1} (q) , j^s f_{i-1} (x) ) $, como la m\'etrica es compatible con la topolog\'ia de $J^s (M,N)$ y $\alpha$ es diferenciable, el mapeo $\psi$ es continuo. Sea $H := \alpha^{-1} ( \psi^{-1} ( \frac{- \delta}{2} , \frac{\delta}{2} ))$, con $\delta = \frac{1}{2^{i}} - d_s ( \sigma , j^s f_{i-1} (x) )$, como $\psi$ y $\alpha$ son continuas, $H$ es abierto en $M$. Entonces $H \cap B( \frac{\delta}{2} , \sigma)$ es abierto y contiene a $\sigma$.

Sea $\tau \in H \cap B(\frac{\delta}{2} , \sigma)$, si $\tau \in G$ cumplir\'ia que $d_s (\tau , j^s f_{i-1} \alpha (\tau )) < 2^{-1} $, pero

$$ d_s (\tau , j^s (f_{i-1} (\alpha (\tau ))) \leq d_s( \tau , \sigma) + d_s (\sigma , j^s f_{i-1} (x)) +$$ $$ d_s (j^s f_{i-1} (x) , j^s f_{i-1} (\alpha (\sigma))) < \frac{\delta}{2} + \left( \frac{1}{2^{i}} - \delta \right) + \frac{\delta}{2} = \frac{1}{2^{i}}$$
Entonces  $H \cap B(\frac{\delta}{2} , \sigma ) \subset G$, por lo tanto $G$ es abierto. $\blacksquare$

Que $C^\infty (M,N)$ sea un espacio de Baire nos dice que los abiertos densos son muy ``grandes", ya que su intersecci\'on vuelve a ser densa. Por el momento seguiremos demostrando otras propiedades de $C^\infty (M,N)$ pero que \'este sea un espacio de Baire ser\'a una de las bases de este trabajo.

\begin{prop}
Sean $M$ y $N$ variedades. Entonces el mapeo $j^k : \\ C^\infty(M,N) \to C^\infty (M, J^k (M,N)$ es continuo en la topolog\'ia $C^\infty$.
\end{prop}
\underline{Demostraci\'on:} Sea $U \subset  J^l (M, J^k (M,N))$ abierto entonces $P(U) \subset \\ C^\infty (M , J^k (M, N))$ es abierto, para que $j^k$ sea continua tenemos que demostrar que $(j^k)^{-1} (P(U))$ es abierto, ya que los conjuntos de la forma $P(U)$ forman una base de la topolog\'ia $C^\infty$. Sea
$$a_{k,l}: J^{k+l} (M,N) \to J^l (M, J^k (M,N)),$$ sea $\sigma \in J^{k+l} (M,N)$ cuya fuente es $x\in M$ y $f:M \to N$ un representante de $\sigma$, definimos $a_{k,l} (\sigma) = j^l (j^k f) (x)$, por lo cual $a_{k,s}$ es suave.

Entonces sabemos que $a_{k,l}^{-1} (U)$ es abierto en $J^{k+l} (M,N)$. Por como definimos a $a_{k,l}$, $a_{k,l} \circ j^{k+l} f \equiv j^l \circ j^k (f) : M \to J^l (J^k (M,N))$, entonces $P (a_{k,l}^{-1} (U)) = (j^k)^{-1} (P(U))$, por lo tanto, $j^k$ es continua. $\blacksquare$

\begin{prop}
Sean $M$, $N$ y $L$ variedades, $\phi:N \to L$ diferenciable. Entonces $\phi_\ast: C^\infty (M,N) \to C^\infty (M, L)$ es continuo en la topolog\'ia $C^\infty$ de Whitney.

\end{prop}
\underline{Demostraci\'on:} Esto es una consecuencia trivial de que $\phi_\ast : J^k (M,N) \to J^k (M,L)$ es diferenciable para toda $k \in \mathbb{N}$. $\blacksquare$

Ahora estudiaremos las propiedades de $C^\infty (M, \mathbb{R}) $ como espacio vectorial sobre $\mathbb{R}$. Nos gustar\'ia que $C^\infty (M,\mathbb{R})$ fuera un espacio vectorial topol\'ogico, es decir, que las operaciones de este espacio vectorial sean compatibles con la topolog\'ia de Whitney. En el caso cuando $M$ no es compacta no lo ser\'a. Por ejemplo, elegimos $f:M \to \mathbb{R}$ tal que tenga soporte no compacto, podemos elegir una funci\'on constante. Entonces si las operaciones fueran continuas tendr\'iamos que $\lim_{n \to \infty} \frac{f}{n} \equiv 0$, pero esto no es posible por c\'omo es la convergencia en este espacio y $f$ tiene soporte no compacto. Es claro que con la suma y multiplicaci\'on de funciones puntualmente no pasar\'a esto.

\begin{lem}
Sean $M$, $N$ y $L$ variedades. Sea $$J^k (M,N) \times_M J^k (M,L) := \{  (x,y) \in J^k (M,N) \times J^k (M,L) \mid \alpha_1 (x) = \alpha_2 (y) \}$$
con $\alpha_1: J^k (M,N) \to M$ y $\alpha_2: J^k(M,L) \to M$ los mapeos fuente, y a este conjunto le damos la topolog\'ia de subespacio en $J^k (M,N) \times J^k (M,L)$. Sean $K \subset J^k (M,N)$ y $L \subset J^k (M,L)$ compactos, tales que $\alpha_1 \vert_L$ y $\alpha_2 \vert_K$ son propias. Sea $U $ vecindad abierta de $K \times_M L$, entonces existen vecindades $V$ de $K$ y $W$ de $L$, tales que $V \times_M W \subseteq U$.
\end{lem}

Aunque este lema lo necesitaremos, no lo vamos a demostrar ya que la demostraci\'on no es de nuestro inter\'es. Este espacio se llama producto fibrado de $J^k (M,N)$ y $J^k (M,L)$ sobre $M$.

\begin{prop}
Sean $M$, $N$ y $L$ variedades. Entonces $\phi: C^\infty (M,N) \times C^\infty (M,L) \to C^\infty (M, N \times L)$ donde $\phi (f(x), g(x)) = (f(x), g(x))$, es un homeomorfismo en la topolog\'ia $C^\infty$.

\end{prop}

\underline{Demostraci\'on:} Sean $\pi_N: N \times L \to N$ y $\pi_L: N \times L \to L$ las proyecciones can\'onicas, \'estas son claramente diferenciables, entonces nos inducen mapeos continuos $$(\pi_N)_\ast : C^\infty (M , N \times L ) \to C^\infty (M, N) \text{ y }(\pi_L)_\ast :C^\infty (M, N \times L) \to C^\infty (M ,L).$$
Es claro que $\phi \equiv \pi_N \times \pi_L$, entonces $\phi$ es continua y adem\'as biyectiva, entonces si $\phi$ es abierta, su inversa (la cual existe por ser biyectiva) ser\'a continua, por lo tanto, $\phi$ ser\'a un homeomorfismo.

Sea $(f,g) \in C^\infty (M, N \times L ) $ (a una funci\'on la podemos identificar con sus funciones coordenadas en $N$ y en $L$). Sea $W \subset J^k (M, N \times L )$ abierto tal que $(f,g ) \in P(W)$. Notemos que $J^k (M, N \times L ) = J^k (M, N) \times_M J^k (M, L)$ (esto es claro por la aclaraci\'on anterior). Ocupando el lema anterior existen abiertos $U \subset J^k (M,N)$ y $V \times J^k(M,L)$ tales que $U \times_M V \subset W$, entonces $P(U) \times P(V) \subset ( (\pi_N)_\ast \times (\pi_L)_\ast )(P(W))$ por lo tanto $\phi$ es abierta. $\blacksquare$

\begin{coro}
Sea $M$ variedad, entonces la adici\'on y multiplicaci\'on (puntual) de funciones son continuas en la topolog\'ia $C^\infty$ de Whitney.

\end{coro}
\underline{Demostraci\'on:}
\begin{enumerate}

\item $+: \mathbb{R} \times \mathbb{R} \to \mathbb{R}$ es continua, entonces $$+_\ast : C^\infty(M,\mathbb{R} \times \mathbb{R} ) \cong C^\infty(M, \mathbb{R}) \times C^\infty (M , \mathbb{R}) \to C^\infty (M , \mathbb{R})$$ es continua, por lo tanto, la adici\'on es continua.
\item $\cdot: \mathbb{R} \times \mathbb{R} \to \mathbb{R}$ es continua, entonces $$\cdot_\ast : C^\infty(M,\mathbb{R} \times \mathbb{R} ) \cong C^\infty(M, \mathbb{R}) \times C^\infty (M , \mathbb{R}) \to C^\infty (M , \mathbb{R})$$ es continua, por lo tanto, la multiplicaci\'on es continua. $\blacksquare$
\end{enumerate}

\begin{prop}
Sean $M$, $N$ y $L$ variedades, con $M$ compacta. Entonces $\circ: C^\infty (M,N) \times C^\infty (N,L) \to C^\infty (M,L)$ dado por la composici\'on es continua en la topolog\'ia $C^\infty $ de Whitney.
\end{prop}
\underline{Demostraci\'on:} Sea $D:= J^k (M,N) \times_N J^k(N,L)$ (ahora, en lugar de tomarnos el producto fibrado con el mapeo fuente dos veces, ocuparemos el mapeo fuente primero y el mapeo blanco respectivamente, el lema sigue siendo cierto, lo hacemos de esta manera para que componer $k$-jets tenga sentido). Sea $\gamma: D \to J^k (M,L)$, dado por $\gamma (\sigma, \tau) = \tau \circ \sigma$, claramente este mapeo est\'a bien definido. 

Si $f \in C^\infty (M,N)$, $g \in C^\infty (N,L)$ y $V \subset J^k (X, Z)$ abierto con $g \circ f \in  P(V)$, entonces existen abiertos $U \subset J^k (M,N)$ y $W \subset J^k (N,L)$ con $\gamma (U \times_N W) \subset P(V)$, entonces si $f^\prime \in P(U)$ y $g^\prime \in P(W)$ tendremos que $g^\prime \circ f^\prime \in P(V)$, si probamos esto tendremos que la composici\'on de $k$-jets es continua para toda $k$, por lo tanto, la composici\'on ser\'a continua en la topolog\'ia $C^\infty$ de Whitney.

Por la regla de la cadena tenemos que $j^k (g \circ f) (M) = \gamma ( j^k f(M) \times_N j^k g(N))$, entonces $j^k f(M) \times_N j^k f(N) \subset \gamma^{-1} (V)$. Entonces si aplicamos el lema anterior con $K= j^k f(M)$, $L= j^k g(N)$ y a $\gamma(V)$ obtendremos la $U$ y $W$ deseada. $\blacksquare$

Esta proposici\'on no ser\'a cierta si $M$ no es compacta, pero como ya hemos mencionado, en el futuro nada m\'as estaremos trabajando con funciones suaves con dominio compacto. 

\begin{prop}
Sean $M$ y $N$ variedades. Sea $l \in \mathbb{N}$, entonces los mapeos $\gamma^l : C^\infty (M,N)^l \to C^\infty (M^l, N^l)$ dados por $\gamma^l ((f_1, \dots , f_l)(x) ) = f_1(x) \times \dots \times f_l(x)$ son continuos en la topolog\'ia $C^\infty$ de Whitney para toda $l \in \mathbb{N}$.
\end{prop}
La proposici\'on se demuestra exactamente igual que las \'ultimas proposiciones, (ocupamos el mismo lema de la misma manera) entonces la omitiremos.

\section{Transversalidad}

En esta secci\'on daremos la definici\'on de transversalidad y daremos algunas proposiciones que necesitaremos, en particular el teorema de transversalidad de Thom, que con junto con que $C^\infty (M,N)$ es un espacio de Baire, ser\'an los mayores protagonistas para el estudio de mapeos estables.

\begin{defi} 
Sean $M$ y $N$ variedades, $L \subset N$ subvariedad y $x \in M$. Entonces $f$ \textit{interseca transversalmente} a $L$ en $x$ ($f \transv L$ en $x$) si ocurre uno de los siguientes casos:
\begin{enumerate}
\item $f(x) \notin L$.
\item $f(x) \in L$ y $T_{f(x)} N = T_{f(x)} L + (df)_x (T_x M)$. Si $f$ interseca transversalmente a $L$ para todo $x \in M$ lo denotaremos simplemente como $f \transv L$.
\end{enumerate}
\end{defi}

La suma de espacios tangentes la estamos pensando como suma de espacios vectoriales, entonces si $f(x) \in L$ es claro que para que $f(x) \transv L$ le tenemos que pedir alguna condici\'on a las dimensiones de las variedades por el teorema de la dimensi\'on. 

Ejemplos: 
\begin{enumerate}
\item $M= \mathbb{R}$, $N= \mathbb{R}^2$, $f(x) = (x , x^2)$ y $L = \mathbb{R} \times \{ 0 \}$, entonces $f \transv L$ para todo $x \in \mathbb{R}$ a excepci\'on del 0.
\item Si $f: M \to N$ es una sumersi\'on entonces $f \transv L$ para cualquier subvariedad $L$ de $N$.
\item Si $f(M) \cap L = \emptyset$ entonces $f \transv L$.

\end{enumerate}

\begin{prop}
Sean $M$ y $N$ variedades, $L \subset N$ subvariedad. Supongamos que $\dim M + \dim L < \dim N$. Si $f: M \to N$ es tal que $f \transv L$, entonces $f(M) \cap L = \emptyset$.
\end{prop}
\underline{Demostraci\'on:}
Supongamos que $f(x) \in L$ para alg\'un $x \in M$. Entonces $$ \dim N= \dim (T_{f(x)} L + (df)_x (T_x M) ) \leq \dim T_{f(x)} L + \dim T_x M = $$ $$\dim L + \dim M < \dim N .$$ Lo cual es una contradicci\'on. $\blacksquare$

\begin{lem}
Sean $M$ y $N$ variedades, $L \subset N$ subvariedad y $f: M \to N$. Sea $x \in M$ tal que $f(x) \in L$. Supongamos que existe $U \subset N$ vecindad de $f(x)$ en $N$ y una sumersi\'on $\phi: U \to \mathbb{R}^k$ con $k= \cod L$ tal que $L \cap U = \phi^{-1} (0)$. Entonces $f \transv L$ en $x$ si y s\'olo si $\phi \circ f$ es una sumersi\'on en $0$.
\end{lem}
\underline{Demostraci\'on:} La vecindad $U$ de $f(x)$ siempre existir\'a, nos podemos tomar una carta de $L$ vista como subvariedad de $N$ y cumplir\'a con lo deseado. Por como definimos $U$ tenemos que $\ker (d\phi)_{f(x)} = T_{f(p)} L$. Entonces $f \transv L$ en $x$ $$ \iff T_{f(x)} + (df)_x (T_x M) = T_{f(x)} N$$ $$\iff \ker (d\phi)_{f(x)} + (d f)_x (T_x M) = T_x N .$$
Al aplicar $(d \phi)_{f(p)}$ en ambas igualdades obtenemos el resultado. $\blacksquare$

\begin{theorem}


Sean $M$ y $N$ variedaes, $L \subset N$ subvariedad. Sea $f: M \to N$ tal que $f \transv L$. Entonces $f^{-1}(L)$ es una subvariedad de $M$.
\end{theorem}

Esto es una consecuencia trivial del lema anterior y el teorema de la sumersi\'on.

\begin{prop}
Sean $M$ y $N$ variedades, $L \subset N$ subvariedad. Sea $$T_L:= \{ f \in C^\infty (M,N) \mid f \transv L \}.$$ Entonces $T_L$ es abierto en $C^\infty(M,N)$ si $L$ es una subvariedad cerrada.
\end{prop}
\underline{Demostraci\'on:} Sea $$U := \{ \sigma \in J^1 (M,N) \mid y \notin L \text{ \'o } y \in L \text{ con } T_y N = T_y L + (df)_x (T_x M) \} ,$$
donde $x$ y $y$ son la fuente y el blanco respectivamente de $\sigma$ y $f:M \to N$ un representante de $\sigma$. Es claro que $P(U) = T_L$, si $U$ fuera abierto entonces $T_L$ ser\'ia abierto.

$U$ es abierto si y s\'olo si $ V := J^1 (M,N) - U$ es cerrado. Sea $\{ \sigma_i \}_{i \in \mathbb{N}} \subset V$ convergente cuyo l\'imite es $\sigma$. Sea $p= \alpha(\sigma)$ y $q= \beta (\sigma)$. Como $\beta (\sigma_i) \in L$ para toda $i \in \mathbb{N}$ y $\beta$ es diferenciable, tenemos que $q \in L$ ya que $L$ es cerrado por hip\'otesis. Sea $g: M \to N$ un representante de $\sigma$. Elegimos coordenadas $U$ alrededor de $p$ y $U^\prime$ alrededor de $q$ tal que $f(U) \subset U^\prime$. Podemos suponer que $U = \mathbb{R}^n$ y $U^\prime = \mathbb{R}^m$ con $p=0$ y $q=0$. Sea $k = \rango(dg)_0$.   

Sea $\phi: \mathbb{R}^m \to \mathbb{R}^m / \mathbb{R}^k \cong \mathbb{R}^{m-k}$ la proyecci\'on can\'onica. Entonces $f \transv L$ en $0$ si y s\'olo si $\phi \circ g$ es una sumersi\'on en $0$ si y s\'olo si $\phi \circ (dg)_0  \notin F$, donde $$F:= \{ A \in \hom (\mathbb{R}^n , \mathbb{R}^{m-k}) \mid \rango A < m -k \} .$$
Sea $\eta: \mathbb{R}^n \times \mathbb{R}^k \times \hom(\mathbb{R}^n , 	\mathbb{R}^m) \subset J^1 (\mathbb{R}^n , \mathbb{R}^m) \to \hom (\mathbb{R}^n , \mathbb{R}^{m-k})$ dado por $\eta ( s , t , B) = \phi \circ B$. $F$ es cerrado ya que proyectar y obtener el determinante son diferenciables, y $F$ lo podemos ver como la imagen inversa del $0$ de una proyecci\'on y el determinante. Como $\eta $ es continua entonces $\eta^{-1}(F)$ es cerrado en  $\mathbb{R}^n \times \mathbb{R}^k \times \hom(\mathbb{R}^n , 	\mathbb{R}^m)$, el cual es cerrado en $J^1 (\mathbb{R}^n , \mathbb{R}^m)$. Entonces $V= \eta^{-1} (F)$, ya que si $\tau = (x , y , (dh)_x)) \in V$ si y s\'olo si $y \in 	\mathbb{R}^k$ y $h$ no interseca transversalmente a $\mathbb{R}^k$ en $0$ si y s\'olo si $\eta(\tau) \in F$, por lo tanto, $\sigma \in V$. $\blacksquare$

\begin{prop}
Sean $M$, $N$ y $Z$ variedades, $W \subset Z$ subvariedad. Sea $j: Z \to C^\infty (M,N)$ una funci\'on (no necesariamente continua) y $\Phi: M \times Z \to N$ dada por $\Phi(x,y) = j(y)(x)$. Si $\Phi$ es suave y $\Phi \transv W$ entonces $\{ z \in Z \mid j (z) \transv W \}$ es denso en $Z$.
\end{prop}
\underline{Demostraci\'on:} Como $\Phi \transv W$, $\Phi^{-1} (W)$ es una subvariedad de $M \times Z$. Sea $\pi: \Phi^{-1} (W) \to Z$, $\pi = \pi_Z \circ i_{\Phi}$, donde $\pi: M \times Z \to Z$ es la proyecci\'on can\'onica en el segundo factor y $i_\Phi: \Phi^{-1} (W) \to M \times Z$ la inclusi\'on can\'onica. Si $\dim(\Phi^{-1} (W)) < \dim Z$ entonces, $\pi (\Phi^{-1} (W))$ tiene medida cero en $Z$, entonces el complemento de $(\Phi^{-1}(W))^c$ es denso en $Z$, por lo tanto, $j(y) \transv W$ para todo $y \notin \Phi^{-1} (W)$. 

Ahora el caso cuando $\dim (\Phi^{-1} (W)) \geq \dim Z$. Si $y \in \Phi^{-1} (W)$ es un valor regular $\pi$ y esto implicar\'a que  $j(y) \transv W$, entonces por el teorema de Sard, esto pasar\'ia para un conjunto denso. Sea $y \in Z$ un valor regular de $\pi$ y $x \in M$. Si $(x,b) \notin \Phi^{-1} (W)$, entonces $j (y) (x) \notin W$ lo cual implica que $j (y) \transv W$. Si $(x,y) \in W$, como supusimos que $y$ es un valor regular de $\pi$ tenemos que $$ T_{x,y} (M \times Z) = T_{x.y} \Phi^{-1} (W) + T_{x,y} (X \times {y}).$$
Aplicamos $(d \Phi)_{x,y}$ de ambos lados de nuestra igualdad y obtenemos $$(d\Phi)_{(x.y)} (T_{x,y} (M \times Z)) = T_{j (y) (x)} W + (dj(y))_x (T_x M) .$$
Por hip\'otesis sabemos que $\Phi \transv W$, entonces combinamos la igualdad anterior con la siguiente igualdad $$T_{\Phi(x.y)} N = T_{\Phi(x,y)} W + (d \Phi)_{(x.y)} (T_{x,y} (M \times Z)$$ $$T_{j(y) (x)} N = T_{j(y)(x)} W + (d j (y))_{x} (T_x M)).$$

Por lo tanto $j(y) \transv W$ en $x$. Y como $y$ fue un valor regular arbitrario de $\pi$, por el teorema de Sard tenemos lo deseado. $\blacksquare$

La notaci\'on que usamos para $j$ en este teorema no fue una simple casualidad, el caso que nos interesa es cuando nuestra familia de mapeos es de la forma $j (f) : M \to J^k (M,N)$, con $f: M \to N$ (ya demostramos que $j(f)$ es suave para cualquier $f$).

\begin{theorem}
(de transversalidad de Thom) Sean $M$ y $N$ variedades, $L \subset J^k (M,N)$ subvariedad. Entonces $T_L := \{ f \in C^\infty(M,N) \mid j^k f \transv L \}$ es un conjunto residual en $C^\infty (M,N)$ con la topolog\'ia $C^\infty$.
\end{theorem}
\underline{Demostraci\'on:} Como toda subvariedad es uni\'on de compactos, por el lema anterior nada m\'as tenemos que probar el caso euclidiano, es decir, cuando $L \subset J^k (\mathbb{R}^m , \mathbb{R}^n)$ es compacto.

Como $j^k: C^\infty (\mathbb{R}^m,\mathbb{R}^n) \to C^\infty (\mathbb{R}^m , J^k (\mathbb{R}^m,\mathbb{R}^n))$ es continuo tenemos que $(j^k)^{-1} (T_L)$ es abierto por la proposici\'on 3.2.5. Entonces nada m\'as tenemos que probar que es denso.

Sean $\rho: \mathbb{R}^n \to [0,1]$ y $\rho^\prime: \mathbb{R}^m \to [0,1]$  suaves tales que 
\[ \rho(x) = 
                \begin{cases} 
                1 \quad \text{ en una vecindad de } \alpha (L)\\
                0 \quad x \notin  \alpha (L).
                \end{cases} 
                \]
\[ \rho^\prime (y) =
               \begin{cases}
               1 \quad \text{ en una vecindad de } \beta(L) \\
               0 \quad y \notin \beta(L),
               \end{cases}
               \]
               
Recordemos que en el cap\'itulo 2 definimos a $V^k_{n,m}$ el espacio vectorial de polinomios en $n$ variables con $m$ entradas, con coeficientes hasta $k$ sin t\'ermino constante. Sea $f \in C^\infty (\mathbb{R}^m,\mathbb{R}^n)$ y $q \in V^k_{n,m}$, definimos $f_q: M \to N$ de la siguiente manera
\[ f_q (x)= 
               \begin{cases}
               f(x) & x \notin \alpha (L) \text{ o } f(x) \notin  \beta(L)\\
                \rho (x) \rho^\prime (f(x))q(x) + f(x)) & \text{ en otro caso. }
               \end{cases}
               \]
                           
Es claro que $f_q$ es diferenciable para todo $q \in V^k_{n,m}$, adem\'as podemos definir el mapeo $F: \mathbb{R}^m \times V^k_{n,m} \to \mathbb{R}^n$ como $F(x, q) = f_q (x)$, el cual es suave por como est\'an definidas las $f_q$. 

Sea $\Phi: \mathbb{R}^m \times V^k_{n,m}\to J^k (\mathbb{R}^m,\mathbb{R}^n)$ definida como $\Phi (x,q) = j^k f_q (x)$, nos gustar\'ia que $\Phi \transv L$, para poder usar la proposici\'on anterior, esto no necesariamente ser\'a cierto pero si restringimos nuestro dominio a una vecindad del $0$ en $V^k_{n,m}$, es decir si existe una vecindad del $0$ en $B_0 \subset V^k_{n,m}$, tal que $\Phi: \mathbb{R}^m \times V_0 \to J^k (\mathbb{R}^m,\mathbb{R}^n)$ sea transversal a $L$ podr\'iamos usar la proposici\'on anterior. Si esto pasara dada $f:\mathbb{R}^m \to \mathbb{R}^n$ existir\'a $\{ q_i \}_{i \in \mathbb{N}} \subset V_0$ que converja a $0$, tal que $j^k f_{q_i} \transv L$ (un conjunto es denso en $V_0$ si y s\'olo si cualquier elemento de $V_0$ se puede aproximar por elementos del denso). Y por como es la convergencia en $C^\infty (\mathbb{R}^m, \mathbb{R}^n)$ tenemos que $f_{q_i} \to f$ cuando $i \to \infty$. En resumen, si encontramos $V_0$ vecindad del $0$ en $V^k_{n,m}$, tal que $\Phi: \mathbb{R}^m \times V_0 \to J^k (\mathbb{R}^m,\mathbb{R}^n)$ sea transversal a $L$ acabar\'iamos esta demostraci\'on.

Sea $\epsilon = \frac{1}{2} \min \{ d ( \sop (\rho^\prime) , \mathbb{R}^m - \alpha(L)), d (\eta_i (\beta (L)), (\rho^\prime)^{-1} [0,1))$. Recordemos que la distancia entre dos conjuntos es el \'infimo de las distancias entre cualesquiera dos elementos de \'estos. Sea $V_0 = \{ q \in V_0 \mid \vert q (x) \vert < \epsilon \text{  } \forall x \in \sop (\rho) \}$, entonces $V_0$ es abierto (distinto del vac\'io). Supongamos que $(x , q ) \in \mathbb{R}^m \times V_0$ y que $\Phi (x, q) \in L$, entonces $x \in \alpha (L)$ y $f_q (x) \in \beta (L)$. Entonces $s = d ( (f(x)) ,(f_q (x)) ) < \epsilon$, ya que $$\eta (f_q (x)) = \rho (x ) \rho^\prime ( f (x) q (x) +  f(x) .$$
 
Entonces $f(x) \in Int (\rho^\prime)^{-1} (1)$ ya que $f_b (x) \in \beta(L)$. Como $\rho=1$ en una vecindad de $\alpha(L)$ entonces $f_b (x) = b(x) + f(x)$ es una propiedad abierta para $x$ y para $b$, entonces $\Phi$ es un difeomorfismo local cerca de $(x,b)$.
$\blacksquare$

\begin{coro}
Sean $M$ y $N$ variedades, con $L \subset J^k (M,N)$ subvariedad, tal que $\alpha (\overbar{L}) \subset U$ abierto de $M$. Sea $f: M \to N$ suave y $V \subset C^\infty (M,N)$ vecindad de $f$, entonces existe $g \in V$ tal que $j^k g \transv L$ y $g \equiv f$ fuera de $U$.
\end{coro}
Esto es una consecuencia inmediata de c\'omo construimos la familia de funciones $f_q$ del teorema de Thom.

\begin{coro}
Sean $M$ y $N$ variedades, $L \subset N$ subvariedad. Entonces:
\begin{enumerate}
\item El conjunto de mapeos de $M$ en $N$ que intersecan transversalmente a $L$ es denso en $C^\infty (M,N)$, si $L$ es abierta entonces tambi\'en ser\'a abierto.
\item Sean $U_1 , U_2 \subset M$ abiertos con $\overbar{U_1} \subset U_2$. Sea $f: M \to N$ y $V$ una vecindad de $f$ en $C^\infty (M,N)$. Entonces existe $g \in V$ tal que $g \equiv f$ en $U_1$ y $g \transv L $ fuera de $U_2$.

\end{enumerate}
\end{coro}
\underline{Demostraci\'on:} 
\begin{enumerate}
\item Recordemos que $J^0 (M,N) = M \times N$ y que el mapeo blanco $\beta: \\ J^0 (M,N) \to N$ es una sumersi\'on. Si $L \subset N$ es una subvariedad, entonces $\beta^{-1} (L)$ es una subvariedad de $M \times N$. Entonces para un conjunto denso de $C^\infty(M,N)$ el mapeo $j^0 f: M \to M \times N$ interseca transversalmente a $\beta^{(-1)} (L)$. Si $j^0 f \transv \beta^{-1} (L)$ implicar\'a que $f \transv L$ obtendr\'iamos lo deseado. Supongamos que $j^0 f \transv \beta^{-1} (L)$, si $j^0 f(x) \notin \beta^{-1} (L)$ entonces $f(x) \notin L$, por lo tanto $f \transv L$ en $x$. Ahora, si $j^0 (x) \in \beta^{-1} (L)$ tenemos $$T_{(x, f(x))} \beta^{-1} (L) + (d j^0 f)_x (T_x M) = T_{(x,f(x))} (M \times N).$$
Aplicamos $(d \beta )_{(x, f(x))}$ en ambos lados de la igualdad y obtenemos $$T_{f(x)} L + (df)_x (T_x M) = T_{f(x)} N .$$

Por lo tanto $f \transv L$ en $x$.
\item Notemos que el conjunto $(M \times N) - \alpha^{-1} (\overbar{U_2})$ es abierto en $M \times N$, entonces es una variedad. Y adem\'as $\beta^{-1} (L) \cap ((M \times N) - \alpha^{-1} (\overbar{U_2})) = L^\prime$ es una subvariedad de \'el, con $\alpha (L^\prime) \subset M - \overbar{U_1}$, entonces ocupamos el corolario anterior para encontrar la $g$ deseada y cumplir\'a lo que deseamos por el inciso anterior. $\blacksquare$

\end{enumerate}

Ahora presentaremos una generalizaci\'on de lo que acabamos de ver pero que nos servir\'a para estudiar la autointersecci\'on de mapeos. Sean $M$ y $N$ variedades. Sea $$M^{(s)} := \{ (x_1 , \dots , x_n) \in M^s = \prod_{i=1}^s M \mid x_i \neq x_j \text{ para } 1 \leq i < j \leq s \} .$$
A $M^{(s)}$ siempre lo pensaremos con la topolog\'ia de subespacio de $M^s$ ( $M^s$ tiene la topolog\'ia producto). Definimos $\alpha^s : (J^k (M,N))^s \to M^s$ como $\alpha^s = \prod_{i=1}^s \alpha$, es diferenciable por ser diferenciable en todas sus entradas. A $J^k_s = (\alpha^s)^{-1} (M^{(s)})$ le llamaremos \textit{$s$-doblez de un haz de $k$-jets}. Un \textit{haz de multijets} es un $s$-doblez de un haz de $k$-jets, si el contexto nos lo permite nada m\'as le llamaremos doblez o s-doblez. Recordemos que $\alpha$ es una sumersi\'on, por como definimos $\alpha^s$ tambi\'en ser\'a una sumersi\'on y como $M^{(s)}$ es un abierto de $X^s$, entonces los dobleces ser\'an variedades, adem\'as de la misma dimensi\'on de $(J^k (M,N))^s$ por ser un abierto de \'este. Si $f: M \to N$ es suave, entonces tenemos el mapeo suave $j^k_s :  X^{(s)} \to J^k_s (M,N)$ definido de manera obvia.

De la misma manera que para los haces de $k$-jets, tendremos teoremas similares para los dobleces.

\begin{theorem}
(de transversalidad de multijets) Sean $M$ y $N$ variedades, $L \subset J^k_s (M,N)$ subvariedad. Entonces $$T_L:= \{ f \in C^\infty (M, N) \mid j^k_s f \transv L \}$$
es un conjunto residual de $C^\infty (M,N)$ en la topolog\'ia $C^\infty$. M\'as a\'un, si $L$ es cerrada, $T_L$ ser\'a abierto.
\end{theorem}

La demostraci\'on procede de igual manera que el teorema de transversalidad de Thom, el detalle importante es modificar un poco los incisos c) y d) de la demostraci\'on, pero si $x \in M^{(s)}$ tenemos que $x=(x_1, \dots , x_s)$ con $x_i \neq x_j$ cuando $i \neq j$, como $M$ es variedad es un espacio Hausdorff y $x$ es una $n$-ada finita, entonces existen vecindades $U_i \in M$ de cada $x_i$ tales que $U_i \cap U_j = \emptyset$ cuando $i \neq j$, estas vecindades son las que nos sirven y una aplicaci\'on adecuada de particiones de la unidad nos permite demostrar lo deseado.

\section{Teorema de inmersi\'on de Whitney}

Antes de comenzar con el estudio de los mapeos estables, estudiaremos los mapeos entre dos variedades agregando ciertas hip\'otesis sobre sus dimensiones. Los mapeos m\'as f\'aciles entre dos variedades son las inmersiones y sumersiones, en esta secci\'on nos dedicaremos al estudio de las inmersiones, en particular si \'estas son densas cuando las dimensiones entre dos variedades son adecuadas. A partir de este momento $M$ y $N$ denotar\'an siempre variedades. En particular en esta secci\'on asumiremos que $\dim (M) \leq \dim (N)$.

\begin{defi}
Sea $\sigma \in J^1 (M,N)$, con $f:M \to N$ un representante de $\sigma$ y $x = \alpha (\sigma)$.\begin{enumerate}

 \item Definimos el \textit{corango de $\sigma$} como $\cor (\sigma) = \cor ( (df)_x)$. Es claro que esta definici\'on no depende del representante (de igual manera se define el rango de $\sigma$).
 \item $ S_r(M,N) := \{ \sigma \in J^1 (M,N) \mid \cor (\sigma) = r \}$, si el contexto nos lo permite nada m\'as escribiremos $S_r$.
 \end{enumerate}
\end{defi}

Como las transformaciones lineales entre espacios vectoriales de dimensi\'on finita tienen un rango m\'aximo, entonces $S_r$ ser\'a vac\'io para casi toda $r \in \mathbb{N}$. Cuando $S_r$ tenga sentido nos gustar\'ia que fuera una subvariedad (subhaz fibrado) de $J^1 (M,N)$, todav\'ia nos hace falta un poco para demostrar eso, pero lo primero que podemos notar es la siguiente afirmaci\'on.

\begin{lem}
 $f: M \to N$ es una inmersi\'on si y s\'olo si $j^1 f(M) \cap \left( \bigcup\limits_{r \neq 0} S_r \right) = \emptyset$.
\end{lem}

\begin{lem}
Sea $S $ una matriz de $m \times n$ donde \[
S =
  \begin{bmatrix}
   A & B\\
   C & D
  \end{bmatrix}
\]
con $A$ una matriz de $k \times k$ invertible. Entonces $ \rango (S) = k$ si y s\'olo si $D- C A^{-1} B = 0$.
\end{lem}
 
\underline{Demostraci\'on:} Sea \[
T=
  \begin{bmatrix}
    Id_{k \times k} & 0 \\
    -C A^{-1} & Id_{ m-k \times m-k}
  \end{bmatrix}
\]
una matriz de $m \times m$ invertible. Entonces $\rango (S) = \rango (TS)$, entonces multiplicamos las matrices y obtenemos \[
TS=
  \begin{bmatrix}
    A & B \\
    0 & D - C A^{-1} B
  \end{bmatrix}
\]
y esta matriz tiene rango $k$ si y s\'olo si $D- CA^{-1}B = 0$. $\blacksquare$

Sean $V^n$ y $W^m$ espacios vectoriales sobre $\mathbb{R}$. Recordemos que $\hom(V,W) \cong \mathbb{R}^{mn}$, es decir es una variedad de dimensi\'on $mn$, definimos $L_r (V,W) = \{ S \in \hom(V,W) \mid \cor (S) = r \}$.

\begin{lem}
$L_r (V,W)$ es una subvariedad de $\hom (V,W)$ con $\cod (L_r (V,W)) \\ = (m - q + r) (n - q +r)$ donde $q = \min \{ n , m \}$.
\end{lem}

\underline{Demostraci\'on:} Sea $S \in L_r (V,W)$ y $k = \rango (S)$. Elegimos bases en $V$ y $W$ tales que \[
S=
  \begin{bmatrix}
    A & B \\
    C & D
  \end{bmatrix} ,
\]
donde $A$ es una matriz invertible de $k \times k$. Sea $U \in \hom(V,W)$ vecindad de $S$, tal que si $S^\prime \in U$ entonces  $
S^\prime=
  \begin{bmatrix}
    A^\prime & B^\prime\\
    C^\prime & D^\prime
  \end{bmatrix}
$ con $A^\prime$ matriz invertible de $k \times k$ (recordemos que proyectar es continuo y el determinante tambi\'en). Sea $f: U \to \hom (\mathbb{R}^{n-k} , \mathbb{R}^{m-k})$ dada por $f(S^\prime) = D^\prime - C^\prime (A^\prime)^{-1} B^\prime$. Es claro que $f$ es una sumersi\'on (nada m\'as fijamos las coordenadas $A$, $B$ y $C$). Entonces por como definimos $f$ tenemos que $f^{-1} (0) = L_r (V,W) \cap U$, la cual es una subvariedad por $f $ ser sumersi\'on. 

Adem\'as $\cod (L_r (V,W)) = \dim (\hom (\mathbb{R}^{n-k} , \mathbb{R}^{m-k}) = (n - k ) (m-k)$. $\blacksquare$

\begin{theorem}
$S_r (M,N)$ es una subvariedad de $J^1 (M,N)$.
\end{theorem}

El lema anterior es el que nos da las cartas de subvariedad claramente. Adem\'as tenemos que $S_0 (M,N)$ es un abierto de $J^1 (M,N)$, esto es claro ya que el determinante es continuo y aplicamos el lema anterior (en este caso las fibras ser\'an abiertos de las fibras originales). Sea $\inm (M,N) \subset C^\infty (M,N)$ las inmersiones de $M$ a $N$, por esta \'ultima aclaraci\'on tenemos el siguiente lema:
\begin{lem}
$Inm (M,N) $ es abierto en $C^\infty(M,N)$ en la topolog\'ia $C^\infty$.
\end{lem}

\begin{theorem}
(de inmersi\'on de Whitney) Supongamos que $dim (N)  \geq \\  2 dim (M)$ entonces $Inm (M,N)$ es un abierto denso de $C^\infty (M,N)$.
\end{theorem}

\underline{Demostraci\'on:} Sabemos que $\cod (S_r) = (n - q + r) (m - q + r)$ con $q= \inf \{ n ,m \}$, entonces si $r \geq 1$ tenemos $$\cod (S_r) = r (m - n + r) \geq r ( n+ r) \geq n + 1 .$$
En este caso, si $f : M \to N$, $j^k f \transv S_r$ si y s\'olo si $j^k f(M) \cap S_r = \emptyset$, por el teorema de transversalidad de Thom y el primer lema de esta secci\'on finaliza nuestra demostraci\'on. $\blacksquare$

\begin{theorem}
Supongamos que $\dim (N) \geq 2 \dim (M) +1$. Entonces el conjunto de inmersiones 1 a 1 es un conjunto residual de $C^\infty (M,N)$.
\end{theorem}

\underline{Demostraci\'on:} Sea $W = (\beta^2)^{-1} ( \Delta N) \subset J^0_2 (M,N)$ subvariedad ($\beta$ es sumersi\'on). Si $f: M \to N$ es una inmersi\'on 1 a 1 cumple que $j^0_2 f: M \to J^0_2 (M,N)$ no intersecta a $W$. Como $\cod (W) = \cod (\Delta N) = \dim (N) > 2 \dim (M) = \dim (M^{(2)})$ (recordemos que $M^{(2)}$ es abierto en $M^2$), entonces $j^0_2 f \transv W$ si y s\'olo si $j^0_2 (M^{(2)}) \cap W = \emptyset$, aplicamos el teorema de multitransversalidad. $\blacksquare$

\section{Funciones de Morse}

Ahora estudiaremos el caso cuando la dimensi\'on del codominio es menor a la del dominio, nada m\'as que trabajaremos el caso m\'as simple (no trivial), cuando el codominio es exactamente $\mathbb{R}$, en este caso las funciones siempre ser\'an ``iguales" (recordemos que en topolog\'ia diferencial luego nos referimos que son iguales cuando en realidad lo son bajo un cambio de coordenadas). Como $\mathbb{R}$ es de dimensi\'on $1$, las \'unicas subvariedades de la forma $S_r$ no vac\'ias son $S_0$ y $S_1$. Entonces $x \in M$ es un punto cr\'itico (tambi\'en llamado una singularidad) de $f: M \to \mathbb{R}$ si $j^1 f(x) \in S_1$. Si est\'a en $S_0$ entonces es una sumersi\'on en $x$, por lo tanto, es un punto regular.

\begin{defi}
\begin{enumerate}
\item Sea $x \in M$ un punto cr\'itico de $f: M \to \mathbb{R}$, decimos que $x$ es \textit{no degenerado} si $j^1 f(x) \transv S_1$.
\item $f$ es una \text{funci\'on de Morse} si todos sus puntos cr\'iticos son no degenerados.
\end{enumerate}
\end{defi}

Por como definimos a las funciones de Morse (una condici\'on de transversalidad) en conjunto con el teorema de transversalidad tenemos el siguiente teorema.

\begin{theorem}
El conjunto de las funciones de Morse es denso y abierto en $C^\infty (M,N)$.
\end{theorem}

De nuestra definici\'on no podemos decir a primera vista c\'omo son las funciones de Morse, para nuestra buena suerte las funciones de Morse son de las m\'as simples que estudiaremos, debido a un teorema de forma normal, es decir, localmente todas las funciones de Morse son iguales m\'odulo un difeomorfismo.

\begin{theorem}
Sea $U \subseteq \mathbb{R}^n$ abierto y $f: U \to \mathbb{R}$ suave, sea $x \in U$ un punto cr\'itico. $x$ es no degenarado si y s\'olo si el hessiano de $f$ en $x$ es no singular.
\end{theorem}
\underline{Demostraci\'on:} Recordemos que $J^1 (U, \mathbb{R}) \cong U \times \mathbb{R} \times \hom (\mathbb{R}^n , \mathbb{R})$. Entonces la proyecci\'on can\'onica $\pi : J^1 (U, \mathbb{R}) \to \hom (\mathbb{R}^n , \mathbb{R})$ es una sumersi\'on y adem\'as $\pi^{-1} (0) = S_1$.

Entonces  $j^1 f \transv S_1$ en $x$ si y s\'olo si $\pi \circ j^1 f$ es una sumersi\'on en $x$. Pero $$\pi \circ j^1 f(x) = (df)_x = \left( \frac{\partial f}{\partial x_1} (x) , \dots , \frac{\partial f}{\partial x_n} (x) \right) .$$
Para que este mapeo ($\pi \circ j^1 f$) sea una sumersi\'on en $x$ nada m\'as tenemos que calcular la matriz jacobiana, entonces derivamos las funciones coordenadas y obtenemos

$$
(d (\pi \circ j^1 f))_x=
  \begin{bmatrix}
    \frac{\partial^2 f}{\partial x_1 ^2 } (x) & \dots & \frac{\partial ^2 f}{\partial x_1 x_n} (x)\\
    \vdots & \ddots &\vdots \\
    \frac{\partial^2 f }{\partial x_n x_1} (x) & \dots & \frac{\partial^2 f}{\partial x_n^2} (x)
  \end{bmatrix}
$$
y esta matriz es no singular en x si y s\'olo si su determinante es no cero. Adem\'as podemos observar que los puntos de Morse son aislados. $\blacksquare$

\underline{Ejemplo:} Sea $f: \mathbb{R}^n \to \mathbb{R}$ dada por $f(x) = x_1^2 + \dots + x_n^2$. Es una funci\'on de Morse ya que, si $x \neq 0$ entonces $f$ es una sumersi\'on en $x$. Si $x = 0$ tenemos $(df)_{0} = 0$, pero el hessiano en $0$ es $2^n$. 

En nuestra demostraci\'on anterior vimos que $(d(\pi \circ j^1 f))_x$ es un difeomorfismo local cuando $x$ es un punto cr\'itico no degenarado, entonces bajo un cambio de coordenadas adecuado podemos suponer que es la matriz identidad en nuestro punto cr\'itico. 

\begin{lem}
(de Hadamard) Sea $f: U \subseteq \mathbb{R}^n \to \mathbb{R}$ suave con $U$ convexo y $f(0)=0$. Entonces $$f (x_1, \dots , x_n) =\sum\limits_{i=1}^n x_i g_i (x_1 , \dots , x_n)$$ para algunas funciones suaves $g_i : U \to \mathbb{R}$ que cumplen $g_i (0) = \frac{\partial f}{\partial x_i} (0)$.
\end{lem}
\underline{Demostraci\'on:} $$f(x) = \int\limits_0^1 \frac{df (t x_1 , \dots , t x_n)}{\partial t} dt = \int\limits_0^1 \sum\limits_{i=1}^n \frac{\partial f}{\partial x_i} (t x_1 , \dots , t x_n) \cdot x_i dt = $$
 $$\sum\limits_{i=1}^n \int\limits_0^1 \frac{\partial f}{\partial x_i} (t x_1 , \dots , t x_n) \cdot x_i dt .$$
Sea $g_i (x_1 , \dots , x_n) = \int\limits_0^1 \frac{\partial f}{\partial x_i} (t x_1 , \dots , t x_n) \cdot x_i dt$. $\blacksquare$

\begin{lem}
(de Morse) Sea $f: U \subseteq \mathbb{R}^n \to \mathbb{R}$ suave con $U$ vecindad de $0$ y $f$ funci\'on de Morse. Si $0$ es un punto cr\'itico de $f$ entonces existe un cambio de coordenadas alrededor del $0$ tal que $f(x_1 , \dots , x_n) =  \pm x_1^2 \dots \pm x_n^2 $.
\end{lem}

No demostraremos este lema ya que necesitamos saber qu\'e es el \'indice de una funci\'on, pero el lector que s\'i est\'e relacionado con estos temas la demostraci\'on nada m\'as se reduce a una aplicaci\'on del lema de Hadamard (en realidad se ocupa dos veces), este lema aunque lo hayamos enunciado en contexto euclidiano tambi\'en aplica a variedades ya que las variedades son localmente euclidianas. Un lema que nos ser\'a \'util en el futuro es el siguiente.


\begin{lem}
Sea $f: U \subseteq \mathbb{R}^n \to \mathbb{R}$ suave con $x \in U$ un punto cr\'itico no degenerado. $\phi: U \to U$ difeomorfismo tal que $\phi (x) = x$, entonces $f  \circ \phi$ es de Morse y adem\'as $0$ es un punto cr\'itico no degenerado de $f \circ \phi$.
\end{lem}

Este es una consecuencia de la regla de la cadena, para finalizar este cap\'itulo tenemos la siguiente proposici\'on. 

\begin{prop}
El conjunto de funciones de Morse cuyos valores cr\'iticos son distintos es residual en $C^\infty (M,\mathbb{R})$.
\end{prop}
\underline{Demostraci\'on:} Sea $S = (S_1 \times S_1 ) \cap J_2^1(M, \mathbb{R}) \cap (\beta^2)^{-1} (\Delta \mathbb{R})$, una subvariedad de $J_2^1 (M,\mathbb{R})$. Aplicamos el teorema de multitransversalidad, entonces el conjunto de transformaciones tales que $j_2^1 f \transv S$ es residual, pero $j_2^1 f \transv S$ si y s\'olo si $j_2^1 (M^{(2)}) \cap S= \emptyset$ por las dimensiones. Si $(x,y) \in M^{(2)}$ son puntos cr\'iticos de $f$ entonces $j^1_2 (x,y) \notin S$ entonces $f(x) \neq f(y)$. $\blacksquare$

\chapter{Estabilidad topol\'ogica}

\section{Lo m\'as b\'asico}

\begin{defi}
\item Sean $f,f^\prime \in C^\infty (M,N)$. Decimos que $f$ es \textit{equivalente} a $f^\prime$ si existen difeomorfismos $g: M \to M$ y $h: N \to N$ tales que el siguiente diagrama conmuta
$$\xymatrix{ M \ar[r]^f \ar[d]^g  & N \ar[d]^h \\
               M \ar[r]^{f^\prime}  & N }$$
              
\item $f \in C^\infty (M,N)$ es estable si existe una vecindad $W_f$ de $f$ en $C^\infty(M,N)$ tal que cualquier $g \in W_f$ es equivalente a $f$.                    
\end{defi}

Que una funci\'on sea estable significa que todas las funciones cercanas son iguales a $f$ salvo un cambio de coordenadas. Aparentemente nuestra definici\'on es muy complicada, es decir, a primera vista es muy dif\'icil decir si una funci\'on es o no estable, pues recordemos que nuestra topolog\'ia es bastante complicada. Nuestro prop\'osito es dar un criterio m\'as sencillo para saber cuando una funci\'on es estable (ni squiera sabemos si existen las funciones estables). En el caso del cap\'itulo anterior, cuando consider\'abamos las funciones de $C^\infty (M,\mathbb{R})$, si una funci\'on $f: M \to \mathbb{R}$ era estable entonces \'esta necesariamente ser\'ia una funci\'on de Morse con valores cr\'iticos distintos dos a dos.

Una acci\'on de grupo $G$ en un conjunto $C$ es una operaci\'on $\cdot: G \times C \to C$ que cumple que $(g g^\prime) \cdot a = g \cdot (g^\prime a)$ y $e_G \cdot a = a$ donde $e_G$ es el neutro de nuestro grupo. Denotemos a $(\text{Dif}(M), \circ)$ al grupo de difeomorfismos de $M$ con la composici\'on usual. Sea $G = \text{Dif}(M) \times  \text{Dif}(N)$, el cual es un grupo con las operaciones puntuales, entonces $G$ act\'ua en $C^\infty (M,N)$  del siguiente modo, sea $(g,h) \in G$ y $f \in C^\infty(M,N)$, $(g,h) \cdot f = h \circ f \circ g^{-1}$. La \'orbita de $f$ bajo la acci\'on de $G$ se denota como $\mathcal{O}_f = \{ f^\prime \in C^\infty (M,N) \mid f^\prime = h \circ f \circ g^{-1} \text{ con } (g,h) \in G \}$.

\begin{lem}
Sea $f \in C^\infty (M,N)$. Entonces $f$ es estable si y s\'olo si la \'orbita de $f$ bajo la acci\'on de $G$ es abierta.
\end{lem}
\underline{Demostraci\'on:} Sea $(g,h): C^\infty (M,N) \to C^\infty (M,N)$ donde $(g,h) = (h_\ast) \circ (g^{-1})^\ast $ el cual es continuo por ser composici\'on de mapeos continuos, adem\'as como $g$ y $h$ son difeomorfismos en $M$ y $N$ respectivamente, $(g,h)$ es un homeomorfismo con inversa $(g^{-1} , h^{-1})$.

Por definici\'on sabemos que $f^\prime$ est\'a en la \'orbita de $f$ si y s\'olo si $f^\prime$ es equivalente a $f$. Como cualquier vecindad de $f$ puede ser trasladada por $(g,h)$ obtenemos lo deseado. $\blacksquare$

\begin{defi}
Sea $f: M \to N$ suave.
\begin{enumerate}
\item Sea $\pi_N : TN \to N$ la proyecci\'on can\'onica, sea $\omega: M \to TN$, decimos que $\omega$ es un \textit{campo vectorial sobre $f$} si el siguiente diagrama conmuta:
$$ \xymatrix{
TN \ar[dr]^{\pi_N}\\\
M \ar[u]^\omega \ar[r]^f &N.} $$

Al conjunto de campos vectoriales sobre $f$ lo denotaremos $C^\infty_f (M, TN)$.
\item $f$ es \textit{infinitesimalmente estable} si y s\'olo si para todo $\omega \in C^\infty_f(M,N)$, existen campos vectoriales $X$ en $M$, y $Y$ en $N$ tales que $$\omega = (df) \circ X + Y \circ f .$$
\end{enumerate}
\end{defi}
Recordemos que en un haz vectorial las operaciones en fibras son suaves, por lo cual sumar vectores que viven en la misma fibra tiene sentido. Si analizamos detenidamente el diagrama de la definici\'on anterior, observamos que $\omega$ a cada $x \in M$ le asigna un vector en la fibra de $f(x)$ en $TN$, entonces por esta observaci\'on podemos identificar a los campos vectoriales sobre $f$ con las secciones del haz pullback $f^\ast (TN)$. A las secciones suaves de un haz E las denotaremos como $C^\infty (E)$. A partir de este momento asumiremos que $M$ es compacta, ya que nuestro prop\'osito es demostrar el siguiente teorema de Mather.

\begin{theorem}
Sea $f:M \to N$. Entonces $f$ es estable si y s\'olo si $f$ es infinitesimalmente estable.
\end{theorem}

El teorema anterior es v\'alido cuando $M$ no es compacta pero tenemos que asumir que $f$ es propia. Esta equivalencia de estabilidad es mucho m\'as manejable ya que aqu\'i tenemos coordenadas (localmente) y tenemos particiones de la unidad para extender el resultado local

\underline{Ejemplo:} 

\begin{enumerate}
\item Sea $n,m \in \mathbb{N}$ con $m \geq n$. Sea $f:M^m \to N^n$ sumersi\'on. Sea $x \in M$, como es una sumersi\'on, el mapeo $(df)_x: T_x M \to T_{f(x)}N$ es una transformaci\'on lineal suprayectiva, esto pasa para todo $x \in M$. Entonces $(df): TM \to TN$ tiene rango constante (en fibras), lo cual es equivalente a que $\ker(df)_x$ tiene el mismo rango para todo $x \in M$. Entonces $\ker (df)$ es un subhaz vectorial de $TM$, cuyas fibras son $\ker(df)_x$ para todo $x \in M$, este subhaz nos induce otro subhaz de $TM$, su haz normal $H$ , cuya fibra en $x$ es el expacio normal a $\ker (df)_x$, entonces, si restringimos el mapeo $(df)_x \vert_{H_x}: H_x \to T_{f(x)} N$ es un isomorfismo, entonces $(df)\vert_{H}: H \to TN$ es un mapeo entre haces que es isomorfismo en fibras, por lo tanto, $(df) \vert_{H}$ nos induce un mapeo biyectivo entre $ (df): C^\infty (H) \to C^\infty_{f} (M,TN)$, lo cual es equivalente a que $f$ sea infinitesimalmente estable.
\end{enumerate}

Nuestra primera intenci\'on es analizar como son los mapeos infinitesimalmente estables localmente. Sea $E$ un haz vectorial sobre $M$, si $x \in M$ denotemos como $C^\infty_x (E)$ al anillo de g\'ermenes de secciones alrededor de $x$.
\begin{defi}
Sea $f:M \to N$, con $x \in M$ y $f(x) = y$.
\begin{enumerate}
\item $\overbar{f} \in C^\infty_x (M)$ es \textit{infinitesimalmente estable} si para todo $\overbar{\omega} \in C^\infty_x (f^\ast (TN))$, existen $\overbar{\tau} \in C^\infty_x (TM)$ y $\overbar{\eta} \in C^\infty_y (TN)$ tal que $$\overbar{\omega} = \overbar{(df)(\tau)} + \overbar{\eta \circ f}.$$
\item $f$ es \textit{localmente infinitesimalmente estable} en $x$ si $\overbar{f}$ es infinitesimalmente estable en $x$.
\end{enumerate}
\end{defi}

Si $f:M \to N$ es localmente infinitesimalmente estable entonces es localmente infinitesimalmente estable para todo $x \in M$. En general el regreso no es cierto ya que no sabemos qu\'e pasa en las autointersecciones, si $x_1, x_2 \in M$ son tales que $f(x_1) = f(x_2) = y$ tenemos que $\eta (f(x_1)) = \eta (f(x_2))$ entonces $\eta$ tiene que resolver el problema alrededor de $x_1$ y $x_2$ simult\'aneamente. En el futuro agregaremos una condici\'on que nos ayudar\'a a resolver este problema. Por el momento como este es un problema local podemos agregarle coordenadas, entonces si pensamos a $f: U \to V$ donde $U \in \mathbb{R}^n$ y $V \subset \mathbb{R}^m$ abiertos, sea $\omega \in C^\infty (TV)$ entonces existen $\tau \in C^n (TU)$ y $\eta \in C^\infty (TV)$ tales que \begin{equation}\omega_i = \sum\limits_{j =1}^\infty \frac{\partial f_i }{\partial x_j} \tau_j + \eta_i (f) \quad 1 \leq i \leq m. \quad \tag{*}\label{xd} \end{equation}

donde $$\tau = \sum\limits_{j=1}^n \tau_j \frac{\partial}{\partial x_j} \quad  \text{     y      } \quad  \eta=           \sum\limits_{i=1}  \eta_i \frac{\partial}{\partial y_j}.$$

Recordemos que la base can\'onica de los espacios tangentes son los operadores derivadas parciales. Podemos resolver las ecuaciones de $\omega_i$ hasta orden $k$ si existen g\'ermenes $\overbar{\tau} \in C^\infty_0 (TU) $ y $\overbar{\eta} \in C^\infty_0 (TV)$ tales que $$\overbar{\omega_i} = \sum\limits_{j=1}^n \frac{\partial f_i}{\partial x_j}\overbar{\tau_j} +\overbar{ \eta_i (f)} + O(\vert x \vert^{k+1}).$$
Esto nos acerca a facilitar el problema localmente, nada m\'as tenemos que ver cu\'al es la $k$ correcta, para \'esto tenemos el siguiente teorema (no lo demostraremos ya que es un resultado meramente algebraico).

\begin{theorem}
(de preparaci\'on de Malgrange) Sea $\phi: M \to N$ suave con $\phi (x) = y$. Sea $A $ un $C^\infty_x (M)$ m\'odulo finitamente generado. Entonces $A$ es un $C^\infty_y (N)$ m\'odulo finitamente generado (con la estructura inducida por $\phi$) si y s\'olo si $A/\mathfrak{m}_y (N) A$ es un espacio vectorial sobre $\mathbb{R}$ finitamente generado.
\end{theorem}

Cuando hacemos el cociente de un anillo de g\'ermenes y su ideal maximal obtenemos $\mathbb{R}$, ya que es un anillo local, por lo tanto tiene sentido considerar a $A/\mathfrak{m}_y (N) A$ como un $\mathbb{R}$ espacio vectorial. 

\begin{coro}
Sea $A$ un $C^\infty_0 (\mathbb{R}^n)$ m\'odulo finitamente generado, $\phi: \mathbb{R}^n \to \mathbb{R}^m$ suave con $\phi (0) = 0$, $\{ e_i \}_{i=1}^k \subset A$. Entonces $\{ e_i \}_{i=1}^k$ generan a $A$ como un $C^\infty_0 (\mathbb{R}^m)$ m\'odulo si y s\'olo si $\{ \pi(e_i) \}_{i=1}^k$ generan a $A/\mathfrak{m}_0 (\mathbb{R}^m)$ como un $C^\infty_0 (\mathbb{R}^m)$ m\'odulo, donde $\pi: A \to A/\mathfrak{m}_0 (\mathbb{R}^m) A$ es la proyecci\'on can\'onica.
\end{coro}

\begin{coro}
Si $\{ \pi (e_i) \}_{i=1}^k$ son una base del espacio vectorial \\$A/(\mathfrak{m}_0^{k+1} (\mathbb{R}^n) A + \mathfrak{m}_0 (C^\infty(\mathbb{R}^m)A)$ entonces $\{ e_i \}_{i=1}^k$ es una base para $A$ visto como $C^\infty_0 (\mathbb{R}^m)$ m\'odulo.
\end{coro}

\begin{theorem}
Sea $f: M^m \to N^n$ suave, $x \in M$ con $f(x) = y$. Entonces $\overbar{f}$ es localmente infinitesimalmente estable en $x$ si y s\'olo si la ecuaci\'on de \ref{xd} puede resolverse hasta orden $m$.
\end{theorem}

\underline{Demostraci\'on:} Notemos que $C^\infty_x (f^\ast (TN)) \cong \bigoplus_{i=1}^n C^\infty_x (M)$, esto es local por lo tanto tenemos el isomorfismo mediante coordenadas, entonces $C^\infty_x (f^\ast(TN))$ es un $C^\infty_x (M)$ m\'odulo finitamente generado. 

Sea $A = \{ \omega \in C^\infty_x (f^\ast (TN)) \mid \omega = (df)(\tau) \text{ con } \tau \in C^\infty_x (TM) \}$. Sea $B = C^\infty_x (f^\ast (TN))/A$, es un $C^\infty_x(M)$ m\'odulo finitamente generado. $B$ es un $C^\infty_y (N)$ m\'odulo mediante $f$. Sea $e_i = \pi (f^\ast (\partial / \partial x_i ))$ con $1 \leq i \leq m$ y $\pi$ la proyecci\'on can\'onica. Es claro que $(\mathfrak{m}_x (M))^{(k+1)}$ son los g\'ermenes de funciones cuya serie de Taylor en $x$ inicia en el t\'ermino $k+1$. Entonces $\overbar{f} \in C^\infty_x (M,N)$ es infinitesimalmente estable si y s\'olo si $\{e_i \}_{i=1}^n$ generan a $B$ como un $C^\infty_y (N)$ m\'odulo si y s\'olo si $B/\mathfrak{m}_x^{n+1} (M) B $ es generado por nuestras $e_i$. Entontes $\omega = \sum\limits_{i=1}^n (\eta_i \circ f) e_i + (df)(\tau) + g$ donde $g \in (\mathfrak{m}_x (M))^{n+1} C^\infty_y (TN)$, es decir, $g$ es de orden $n+1$.

Sea $E$ un haz vectorial sobre $M$ con proyecci\'on $\pi$, entonces $\pi_\ast: J^k (M, E) \to  J^k (M,M)$ es una sumersi\'on. Sea $I = \{ \sigma \in J^k (X,X) \mid \overbar{Id_M} = \sigma \}$ subvariedad de $J^k (X,X)$, entonces $J^k(E) = (\pi_\ast)^{-1} (I)$ es una subvariedad de $J^k (M,E)$, el cual llamaremos haz de $k$-jets de secciones de $E$, es claro tambi\'en que $\alpha: J^k (E) \to X$ es un mapeo de haces fibrados, denotaremos a la fibra en $x \in M$ de este haz como $J^k (E)_x$, en este caso tenemos a\'un m\'as, este ser\'a un haz vectorial sobre $M$, ya que las fibras ser\'an polinomios de grado la dimensi\'on de $E$. En lenguaje de jets el teorema anterior nos dice que $f$ es infinitesimalmente estable en $x$ depende exclusivamente de $j^{n+1} f(x)$, reformulamos el teorema anterior de la siguiente manera.

\begin{theorem}
Sea $f: M^m \to N^n$ suave con $f(x) = y$. $f$ es infinitesimalmente estable en $x$ si y s\'olo si $$J^n(f^\ast (TN))_x = (df) _x J^n (TM)_x  + f^\ast J^n(TN)_y .$$
\end{theorem}

Anteriormente dijimos que aunque podamos resolver el problema de estabilidad infinitesimalmente localmente no podemos generalizar debido a las autointersecciones de $f$.

\begin{defi} 
Sea $f: M \to N$ suave, $y \in N$ y $S = \{x_1, \dots , x_k \}\subseteq f^{-1} (y)$, entonces $f$ es \textit{simult\'aneamente infinitesimalmente estable} en $S$ si para cualesquiera $\overbar{\omega_i} \in C^\infty_{x_i} (f^\ast (TN))$ con $1 \leq i \leq k$, existen $\overbar{\tau_i} \in C^\infty_{x_i} (TM)$ con $1 \leq i \leq k$ y $\overbar{\eta} \in C^\infty_y (TN)$ tales que $$\overbar{\omega_i} = \overbar{(df)(\tau_i)} + \overbar{\eta \circ f} \text{ para toda } i.$$
\end{defi}

Sea $E$ un haz vectorial sobre $M$ y $S= \{ x_i \}_{i=1}^k$ definimos $J^m (E)_S = \\ \bigoplus_{i=1}^k J^m(E)_{x_i}$. De la misma manera tenemos el siguiente teorema.

\begin{theorem}
Sea $f:M \to N$ suave y $S = \{ x_i \}_{i=1}^k \subset f^{-1} (y)$. Entonces $f$ es simult\'aneamente infinitesimalmente estable en $S$ si y s\'olo si $$J^m (f^\ast (TN))_S = (df)J^m(TM)_S + f^\ast J^m (TN)_y .$$
\end{theorem}

Este teorema se demuestra exactamente igual que el teorema anterior. Antes de seguir con esto, introduciremos unos conceptos que nos ayudar\'an posteriormente.

\begin{defi}
Sea $V$ un espacio vectorial real, y $\{ H_i \}_{i=1}^r \subset V$ familia de subespacios de $V$. Decimos que $\{H_i \}_{i=1}^r$ est\'an en \textit{posici\'on general} si para cualquier sucesi\'on de enteros $\{ i_j \}_{j=1}^s$ con $1  \leq i_1 < \dots <i_s \leq r$ tenemos 
$$\cod (\bigcap\limits_{j=1}^s H_{i_j}) = \sum\limits_{j=1}^s \cod (H_{i_j}) .$$
\end{defi}

El caso m\'as simple es cuando $r=2$, entonces $H_1$ y $H_2$ est\'an en posici\'on general si y s\'olo si $H_1 + H_2 = V$, ya que $$\dim (H_1 + H_2) = \dim (H_1) + \dim(H_2) - \dim (H_1 \cap H_2)$$
$$= \dim (V) - (\cod(H_1) + \cod(H_2) - \cod (H_1 \cap H_2),$$
entonces $\dim (H_1 + H_2) = \dim (V)$ si y s\'olo si $H_1$ y $H_2$ est\'an en posici\'on general. Este caso es bastante simple, pero en general no lo ser\'a, as\'i que damos una nueva definici\'on para darnos una idea c\'omo son los espacios que est\'an en posici\'on general.


\begin{theorem}
Sea $V$ un espacio vectorial real, $\{ H_i \}_{i=1}^r$. Entonces $\{ H_i \}_{i=1}^r$ est\'an en posici\'on general si y s\'olo si $H_i \transv (\bigcap\limits_{i \neq j} H_j)$.
\end{theorem}

\begin{lem}
Si $\{ H_i \}_{i=1}^r$ est\'an en posici\'on general entonces cualquier subfamilia de $\{ H_i \}_{i=1}^r$ est\'an en posici\'on general.
\end{lem}

\begin{lem}
Sea $V$ espacio vectorial de dimensi\'on finita. Sean $\{ H_i \}_{i=1}^r \subset V$ familia de subespacios. Entonces $\{ H_i \}_{i=1}^r$ est\'an en posici\'on general si y s\'olo si dados $\{ v_j \}_{j=1}^r$ existen $h_i \in H_i$ y $z \in V$ tales que $v_i = h_i + z$ para toda $1 \leq i \leq r$.
\end{lem}

\underline{Demostraci\'on:} Sea $\pi: V \to \bigoplus\limits_{i=1}^r V/H_i$, con funciones coordenadas las proyecciones can\'onicas. $\ker (\pi) = \bigcap\limits_{i=1}^r H_i$, entonces la siguiente sucesi\'on es exacta $$0 \rightarrow \bigcap\limits_{i=1}^r H_i \rightarrow V \rightarrow \bigoplus\limits_{i=1}^r V/H_i$$

Como estamos trabajando con espacios vectoriales de dimensi\'on finita, $\pi$ es suprayectiva si y s\'olo si $\cod (\bigcap\limits_{i=1}^r H_i ) = \sum\limits_{i=1}^r \dim (V /H_i)$, pero $\pi $ es suprayectiva si y s\'olo si dados $v_i \in V/H_i$ existe $z \in V$ tal que $\pi (z) = v_i$ para todo $i$. $\blacksquare$

Este \'ultimo lema es bastante sugestivo, como los espacios tangentes son espacios vectoriales y un campo vectorial es una funci\'on que a cada punto le asigna un vector en su espacio tangente, entonces queremos aplicar la \'ultima parte de nuestro lema en este contexto. 

\begin{theorem}
Sea $f: M^m \to N^n$ suave. Entonces $f$ es infinitesimalmente estable si y s\'olo si se cumple \ref{lol} para todo $q \in N$ y cualquier $S \subset f^{-1} (y)$ que tiene a lo m\'as $n+1$ puntos \begin{equation}J^m(f^\ast TN)_S = (df) J^m (TM)_S + f^\ast (J^m(TN)_y) .\tag{\#}\label{lol} \end{equation}
\end{theorem}

La necesidad es evidente, antes de demostrar la suficiencia necesitaremos los siguientes lemas.

\begin{lem}
Sea $f: M \to N$ que satisface \ref{lol} y $S = \{ x_i \}_{i=1}^k \subset f^{-1} (y)$. Sean $H_i = (df)_{x_i} (T_{x_i} M)$. Entonces $\{H_i \}_{i=1}^k \subset T_y N$ est\'an en posici\'on general.
\end{lem}

\begin{lem}
Sea $f:M ^m\to N^n$ que satisface \ref{lol}, $y \in N$. Entonces el n\'umero de puntos cr\'iticos en $f^{-1} (y)$ es $\leq n$.
\end{lem}

\underline{Demostraci\'on:} Sea $S= \{ x_i \}_{i=1}^{n+1} \subset f^{-1} (y)$ el conjunto de todos los puntos cr\'itocos de $f^{-1}(y)$. Entonces $\{ H_i \}_{i=1}^{n+1} \subset T_y N$ est\'an en posici\'on general (ocupando la notaci\'on del lema anterior), entonces $$n  > \cod (\bigcap\limits_{i=1}^{n+1} H_i ) = \sum\limits_{i=1}^{n+1} \cod ( H_i) \geq n+1$$

ya que los $x_i$ son puntos cr\'iticos, entonces $\cod H_i \geq 1$. $\blacksquare$

\underline{Demostraci\'on suficiencia:} Es claro que el problema de estabilidad infinitesimal est\'a en los puntos cr\'iticos de $f$, en los puntos regulares es claro cual es la soluci\'on, ya que en estos puntos la diferencial bajo un cambio de coordenadas adecuado es simplemente una proyecci\'on. Sea $\Sigma$ el conjunto de puntos cr\'iticos de $f$, y sea $\Sigma_y = \Sigma \cap f^{-1} (y)$, por el lema anterior $\Sigma_y$ tiene a los m\'as $n$ puntos para cualquier $y \in N$. Sea $\omega \in C^\infty_f (M,N)$, entonces tenemos que encontrar los campos vectoriales $X$ en $M$ y $Y$ en $N$, primero resolveremos el problema en una vecindad de $\Sigma$.

Demostraremos que existen abiertos $\{U_i \}_{i=1}^k \subset M$, $\{ V_i \}_{i=1}^k \subset N$ y $\{ W_i \}_{i=1}^k \subset N$, y campos vectoriales $X_i$ en $U_i$ y $Y_i$ en $V_i$ que satisfacen 
\begin{enumerate}
\item $f(\Sigma) \subset \bigcup\limits_{i=1}^k W_i$.
\item $f(U_i) \subset V_i$.
\item $\omega =  (df)(X_i) + f^\ast Y $ en $U_i$.
\item $f^{-1} (\overbar{W_i}) \cap \Sigma \subset U_i$.
\item $\overbar{W_i} \subset V_i .$
\end{enumerate}

Como $f(\Sigma)$ es compacto (recordemos que $M$ siempre es compacta) nada m\'as tenemos que resolver el problema para una $y \in N$, entonces existir\'a la $k$ por compacidad. Por una proposici\'on anterior podemos encontrar $U$ y $V$ que cumplan lo anterior nada m\'as en $y$, con campos vectoriales $X$ en $U$ y $Y$ en $V$. $U$ lo podemos elegir de tal forma que sea uni\'on de vecindades ajenas de las preim\'agenes  de $y$ que son puntos cr\'iticos de $f$. Elegimos $W$ de tal modo que satisfaga las dos \'ultimas condiciones, debido a la continuidad de $f$ y la compacidad de $M$.

Elegimos una partici\'on de la unidad $\{ \rho_i \}_{i=1}^k$ en $W = \bigcup\limits_{i=1}^k W_i$ tal que $\sop (\rho_i ) \subset W_i$. Elegimos una vecindad $Z$ de $\Sigma$ tal que $f^{-1} (\overbar{W_i}) \cap Z \subset W_i$. Sea $\rho: M \to \mathbb{R}$ tal que $\sop (\rho) \subset Z$ y $\rho \equiv 1$ en una vecindad de $\sigma$. Sea $X = \sum\limits_{i=1}^k \rho f^\ast (\rho_i) X_i$ y $Y = \sum\limits_{i=1}^k \rho_i Y_i$. Por como definimos las funciones $\omega = (df) (X) + f^\ast Y$ en una vecindad de $\sigma$. Como las sumersiones son infintesimalmente estables, podemos resolver el problema en el complemento de $\Sigma$ y pegar nuestras soluciones con otra partici\'on de la unidad. $\blacksquare$

Ahora demostraremos dos proposiciones que nos servir\'an en futuras secciones.

\begin{prop}
Sea $f:M \to N$ infinitesimalmente estable. Entonces existe $W \subset C^\infty (M,N)$ vecindad de $f$ tal que toda $g \in W$ es localmente infnitesimalmente estable.
\end{prop}

\underline{Demostraci\'on:} Sea $x \in M$ con $y = f(x)$. Como $f$ es infinitesimalmente estable $J^m (f\ast TN)_y = (df) J^m (TM)_x + f^\ast J^m (TN)_y$. Consideremos el siguiente mapeo suprayectivo:$$\overbar{f} :J^m (TM)_x \bigoplus J^m (TN)_y =J^m (f^\ast TN)_x ,$$
donde $\overbar{f} = (df) + f^\ast$, en coordenadas adecuadas este es un mapeo de $V^k_{m,m} \bigoplus V^k_{n,n} \\ \to V^k_{n,m}$. Es claro que $\overbar{f}$ depende continuamente de $x$ y de $f$, entonces existe una vecindad $U_x \subset M$ de $x $ y una vecindad $W_x \subset C^\infty (M,N)$ de $f$ tal que si $g \in W_x$ se tiene que  $\overbar{g}$ es suprayectiva, entonces $g$ es localmente infinitesimalmente estable en $x^\prime$ si $x^\prime \in U_x$. Como $M$ es compacta y $U_x$ es una cubierta de $M$, existe una subcubierta finita $\{U_i \}_{i=1}^k$ entonces la vecindad deseada es $\bigcap\limits_{i=1}^k W_i$, la cual es una vecindad de $f$ porque estamos intersecando un n\'umero finito de vecindades. $\blacksquare$

Es claro que nos gustar\'ia demostrar este teorema para estabilidad infitesimal global, pero esto no ser\'a posible debido a que $M^{(s)}$ no es compacta aunque $M$ sea compacta. Por el momento lo mejor que podremos hacer ser\'a agregar una hip\'otesis para forzar esto.

\begin{defi} 
Un mapeo $f:M \to N$ infinitesimalmente estable cumple la propiedad $\mathfrak{O}$ si para todo $x \in M$ existe una vecindad $U_x \in M$ de $x$, y una vecindad $W_x \in C^\infty(M,N)$ de $f$ tal que si $g \in W_x$ y $S= \{ p_i \}_{i =1}^k \subset U_x \cap g^{-1} (y)$, entonces $$J^m (g^\ast TN)_S = (dg) J^m (TM)_S + g^\ast J^m (TN)_y .$$

\end{defi}

\begin{lem}
Si $f: M \to N$ es infinitesimalmente estable que satisface la propiedad $\mathfrak{O}$, entonces existe una vecindad $W_f \in C^\infty (M,N)$ de $f$ tal que todos los mapeos de $W_f$ son infinitesimalmente estables.
\end{lem}

\underline{Demostraci\'on:} Sea $(x_1 , x_2) \in M \times M$. Buscaremos vecindades $U_{x_1}, U_{x_2} \subset M$ de $x_1$ y $x_2$ respectivamente y una vecindad $W_{x_1, x_2} \subset C^\infty (M,N)$ de $f$, tal que si $g \in W_{x_1, x_2}$, $p \in U_{x_1}$ y $q \in U_{x_2}$ con $g(x) = g(y)$ entonces $J^m (g^\ast TN)_{p,q} = (dg) J^m (TM)_{p,q} + g^\ast   J^m (TN)_{g(p)}$.

\begin{enumerate}
\item Si $x_1 = x_2$ obtenemos las vecindades ya que $f$ cumple la propiedad $\mathfrak{O}$.
\item Si $x_1 \neq x_2$ con $f(x_1) \neq f(x_2)$, elegimos $U_{x_1}$, $U_{x_2}$ y $W_{x_1, x_2}$ tal que si $g \in W_{x_1, x_2}$, $g(U_{x_1} ) \cap g(U_{x_2}) = \emptyset$.
\item Si $x_1 \neq x_2$ con $f(x_1) = f(x_2)$ elegimos cartas  ajenas $U_{x_1} $ y $U_{x_2}$ alrededor de $x_1$ y $x_2$ respectivamente y una vecindad de $f$, $W_{x_1, x_2}$ tal que si $g \in W_{x_1, x_2}$ entonces $g(U_{x_1}) \cup g(U_{x_2})$ est\'a contenido en una carta. Procedemos de la misma manera que cuando ten\'iamos un punto, es decir, nos tomaremos $\overbar{f} = (df) + f^\ast$, nada m\'as al dominio de esta funci\'on le agregamos el espacio vectorial correspondiente al nuevo punto.
\end{enumerate}

Como $M \times M$ es compacta, entonces $\{ U_{x_1} \times U_{x_2} \}_{(x_1, x_2) \in M \times M}$ es una cubierta abierta, entonces existe una subcubierta finita $\{U_i , U_j \}_{i,j \in J}$, entonces la vecindad de $f$ deseada es $\bigcap\limits_{i,j \in J} W_{i,j}$. $\blacksquare$





\section{Estabilidad bajo deformaciones}

Todav\'ia no conseguimos nuestro prop\'osito, demostrar el teorema de Thom. Para eso introduciremos un nuevo concepto de estabilidad (al final todas nuestras definiciones de estabilidad ser\'an equivalentes), esta nueva clase de estabilidad es muy similar a homotop\'ia, de hecho ser\'a una generalizaci\'on, esta teor\'ia se debe a Harold Levine y a Ren\'e Thom. Seguiremos asumiendo que $M$ es compacta en esta secci\'on y en todas las posteriores.

\begin{defi}
Sea $f:M \to N$ suave y $I_\epsilon = (- \epsilon , + \epsilon)$, donde $\epsilon > 0$.
\begin{enumerate}
\item Sea $F: M \times I_\epsilon \to N \times I_\epsilon$ suave, $F$ es una \textit{deformaci\'on de $f$} si 
\begin{enumerate}[a.]
\item Para todo $s \in I_\epsilon$, $F$ es de la forma $F(x,s) = (F_s (x) , s)$.
\item $F_0 = f$.
\end{enumerate}
\item Sea $F$ deformaci\'on de $f$. $F$ es una \textit{deformaci\'on trivial de $f$} si existen difeomorfismos $G: M \times I_\delta \to M \times I_\delta$ y $H: N \times I_\delta \to N \times I_\delta$ con $\delta \leq \epsilon$ donde $G$ y $H$ son deformaciones de la identidad de $Id_M$ y $Id_N$ respectivamente, son tales que el siguiente diagrama conmuta
$$\xymatrix{
M \times I_\delta \ar[d]^G \ar[r]^F & N \times I_\delta \ar[d]^H\\
M \times I_\delta \ar[r]^{f \times Id_{I_\delta}} &N \times I_\delta .}$$
\item $f$ es \textit{homot\'opicamente estable} si toda deformaci\'on de $f$ es trivial.


\end{enumerate} 
\end{defi}

Es claro que $f \times Id_{I_\delta}$ es una deformaci\'on de $f$, es la deformaci\'on m\'as simple de $f$. Tambi\'en podemos notar que podemos pensar a $F: I_\delta \to C^\infty (M,N)$ como una curva de funciones, que a cada $t \in I_\delta$ le asignamos $F_t$.

\begin{lem}
Sea $f:M \to N$ homot\'opicamente estable. Si existe una vecindad $W \in C^\infty (M,N)$ de $f$ tal que si $g \in W$, $g$ es homot\'opicamente estable, entonces $f$ es estable.
\end{lem}
 
\underline{Demostraci\'on:} Podemos asumir que $W$ es conectable por trayectorias. Sea $g \in W$, entonces existe $F: M \times [-1 , 1] \to N \times [-1,1]$ con $F_1 \equiv g$ y $F_t \in W$ para toda $t \in I_1$. Consideremos la siguiente relaci\'on de equivalencia $\sim$ en $I_1$, $t \sim s$ si y s\'olo si $F_s$ es equivalente a $F_t$, como en una deformaci\'on de las nuestras (con $W$ conectable por trayectorias) las $F_t$ son equivalentes entonces las clases de equivalencia son abiertas, como $I_1$ es conexo entonces solo hay una clase de equivalencia. $\blacksquare$

Si los mapeos infinitesimalmente estables formaran un conjunto abierto entonces estabilidad infinitesimal implicar\'a estabilidad homot\'opica entonces los mapeos infinitesimalmente ser\'ian estables. Todav\'ia nos hace falta demostrar estas dos casos, a\'un tenemos que deshacernos de la propiedad $\mathfrak{O}$. Por el momento generalizaremos el concepto de estabilidad homot\'opica de la manera m\'as natural, en lugar de deformar las funciones a trav\'es de un intervalo lo haremos por un abierto $U \in \mathbb{R}^k$.

\begin{defi}
Sea $f:M \to N$ suave y $U \in \mathbb{R}^k$ vecindad del $0$.
\begin{enumerate}
\item Sea $F:M \times U \to N \times U$ suave. $F$ es una \textit{$k$-deformaci\'on de $f$} si 
\begin{enumerate}[a.]
\item Para todo $u \in U$ y $x \in M$, $F(x,u) = (F_u (x), u)$.
\item $F_0 = f.$

\end{enumerate}
\item Sea $F: M\times U \to N \times U$, una $k$-deformaci\'on de $f$. $F$ es una \textit{$k$-deformaci\'on trivial de $f$} si existen difeomorfismos $G: M \times V \to M \times V$ y $H: N \times V \to N \times V$ $k$-deformaciones de $Id_M$ y $Id_N$ respectivamente con $V \subset U$ vecindad del $0$, tales que el siguiente diagrama conmuta
$$\xymatrix{
M \times V \ar[d]^G \ar[r]^F &N \times V\ar[d]^H\\
M \times V \ar[r]^{f \times Id_V} &N \times V.}$$
\item $f$ es \textit{estable bajo k-deformaciones} si toda deformaci\'on de $f$ es trivial.
\end{enumerate}
\end{defi}

Es claro que si $l \leq k$ entonces si $f$ es estable bajo k-deformaciones entonces lo ser\'a bajo l-deformaciones, en particular ser\'a homot\'opicamente estable. Antes de seguir con nuestro prop\'osito demostraremos algunos lemas t\'ecnicos.

\begin{lem}
Sea $K \subset \mathbb{R}^n$ compacto convexo, $x \in K$ y $g: \mathbb{R}^n \to \mathbb{R}$ suave. Sea $$g(y) = \sum\limits_{0 \leq \vert \alpha \vert \leq r} p_\alpha (y - x)^\alpha + \sum\limits_{\vert \beta \vert = r+1} g_\beta(y)(y-x)^\beta$$
el polinomio de Taylor de $g$ centrado en $x$ de grado $r$. Si $\Vert g \Vert_s^K < \epsilon $ entonces $\Vert g \Vert_{s-r-1} < \epsilon$ con $r < s$, donde $$\Vert g \Vert_s^K = \sup\limits_{ y \in K, \, 0 \leq  \vert \alpha \vert \leq s } \vert \frac{\partial^{\vert \alpha \vert} g}{\partial y^\alpha} (y) \vert.$$
\end{lem}

\underline{Demostraci\'on:} Podemos suponer que $x=0$, iniciamos con el caso $r=0$, entonces $g(y) = g(0) + \sum\limits_{i=1}^n y_i g_i (y)$ del lema de Hadamard, donde $g_i (y) = \\ \int\limits_0^1 (\partial g / \partial x_i )(t y_i) dt$, entonces $$\vert \frac{\partial g_i^{\vert \alpha \vert}}{\partial y^\alpha} (y) \vert \leq \int\limits_0^1 \vert \frac{\partial^{\vert \alpha \vert}}{\partial y^\alpha} \frac{\partial g}{ \partial y_i} (ty) \vert dt < \epsilon \quad \vert \alpha \vert \leq s-1  $$ ya que $\Vert g \Vert_s^K < \epsilon $ entonces $\Vert g_i \Vert_{s-1}^K < \epsilon$. Para el caso general $$g(y) = \sum\limits_{0 \leq \vert \alpha \vert \leq r-1} p_\alpha (y - x)^\alpha + \sum\limits_{\vert \beta \vert = r} g_\beta(y)(y-x)^\beta ,$$ aplicamos el lema de Hadamard a nuestras $g_\beta$ para despues usar nuestra hip\'otesis de inducci\'on. $\blacksquare$

En el lema anterior aunque aparentemente no ocupamos que $K$ fuera conexo lo necesitamos para aplicar el lema de Hadamard. Sea $\mathbb{R}^l_n$ el espacio vectorial de polinomios en $n$ variables y de grado $\leq l$.

\begin{lem}
Sea $r, s \in \mathbb{Z}$ con $r \geq 0$ y $s >0$ y $K \subset \mathbb{R}^n$. Sea $U \subset \mathbb{R}^l_n$ vecindad del $0$ donde $l = (r+1)^s$. Entonces existe $\epsilon >0$ tal que si $\{ x_i \}_{i=1}^s \subset K$ y $g: \mathbb{R}^n \to \mathbb{R}$ suave es tal que $\Vert g \Vert^K_{s(r+1)} < \epsilon$, entonces existe $q \in U$ tal que $$\frac{\partial^{\vert \alpha \vert} p}{\partial y^\alpha} (x_i) = \frac{\partial^{\vert \alpha \vert }g }{\partial y^\alpha } (x_i) \quad 1 \leq i \leq s \quad 0 \leq \vert \alpha \vert \leq r.$$
\end{lem}
\underline{Demostraci\'on:} Haremos inducci\'on sobre $s$. Sea $x=x_1$, entonces $$g(x) = \sum\limits_{0 \leq \vert \alpha \vert \leq r} p_\alpha (y - x)^\alpha + \sum\limits_{\vert \alpha \vert = r+1} g_\alpha(y)(y-x)^\alpha . $$ 
Sea $p= \sum\limits_{0 \leq \vert \alpha \vert \leq r} p_\alpha (y - x)^\alpha$, como los coeficientes dependen de $\epsilon$ podemos elegir $\epsilon$ tan pequeña como queramos para que $p \in U$.
Supongamos que el lema es cierto para $s-1$, $\{x_i \}_{i=1}^s \subset K$ distintos dos a dos, apliquemos el lema de Taylor alrededor de $x_s$ y obtenemos 
$$g(y) = \sum\limits_{0 \leq \vert \alpha \vert \leq r} p_\alpha (y - x_s)^\alpha + \sum\limits_{\vert \alpha \vert = r+1} g_\alpha(y)(y-x_s)^\alpha . $$ 
Si $\Vert g \Vert_{s(r+1)}^K < \epsilon$ entonces $\Vert g \Vert_{(s-1)(r+1)}^K < \epsilon$, entonces podemos escoger polinomios $ q_\alpha$ cuyo grado es $\leq (r+1)^{k-1}$ tal que $$\frac{\partial^{\vert \alpha \vert q_\alpha}}{ \partial y^{\vert \beta \vert}} (x_i) = \frac{\partial^{\vert \alpha \vert} g_\alpha}{\partial y^\beta} (x_i) \quad 1 \leq i \leq s-1 \quad 0 \leq \vert \beta \vert \leq r .$$
Sea $$q= \sum\limits_{0 \leq \vert \alpha \vert \leq r} p_\alpha (y-x_s)^\alpha + \sum\limits_{ \vert \alpha \vert = r+1} (y-x_s)^\alpha q_\alpha .$$
Por la misma raz\'on que el caso base podemos hacer que $q \in U$. Entonces $$\frac{\partial^\beta g}{\partial y^\beta} (x_i) = \frac{\partial^{\vert \beta \vert }}{\partial y^\beta} \left( \sum\limits_{0 \leq \vert \alpha \vert \leq r} p_\alpha (y - x_s)^\alpha + \sum\limits_{\vert \alpha \vert = r+1} g_\alpha(y)(y-x_s)^\alpha \right) =$$ $$\frac{\partial{\vert \beta \vert }q}{\partial y^\beta} (x_i)$$
para $1 \leq i \leq s$ y $0 \leq \vert \beta \vert \leq r$. $\blacksquare$


\begin{prop} Sea $f:M \to N$ infinitesimalmente estable y estable bajo k-deformaciones para alg\'un $k$. Entonces $f$ satisface la propiedad $\mathfrak{O}$.
\end{prop}

\underline{Demostraci\'on:} Sea $x \in M$. Elegimos cartas de $U \subset M$ y $V \subset N$ tales que $x \in U$ y $f(\overbar{U}) \subset V$. Sea $W \in C^\infty(M,N)$ vecindad de $f$ tal que si $g \in W$ entonces $g(\overbar{U}) \subset V$, sea $U_x$ una vecindad de $x$ convexa con cerradura compacta. Sea $\Vert g \Vert^{\overbar{U_x}} = \sup\limits_{1 \leq i \leq n} \Vert g_i \Vert^{\overbar{U_x}}$. Sea $\rho: M \to \mathbb{R}$ suave que es $1$ en una vecindad de $\overbar{U_x}$ y es $0$ fuera de $U$. Sean $r=s=n+1$ y $l= (m+2)^{m+1}$.

Sea $F: M \times V^l_{m,n} \to N \times V^l_{m,n}$ dada por $F(y,p) = (f(y) + \rho(x)p(x), p)$ deformaci\'on de $f$. Como $f$ es estable bajo k-deformaciones existe una vecindad $Z \in V^l_{m,n}$ del $0$ donde $F$ es trivial.

Sea $W_\epsilon = \{ g \in W \mid \Vert g - f \Vert^{\overbar{U_x}}_{(m+1)(m+2)} < \epsilon \}$ vecindad de $f$. Elegimos $\epsilon >0$ del lema anterior, es tal que si $\{ x_i \}_{i=1}^s \subset U_x$ distintos dos a dos con $s \leq m+1$, si $g \in W_\epsilon$ entonces existe $q \in Z$ vecindad del $0$ tal que $j^{m+1} (g-f)(x_i) = j^{m+1} q(x_i)$ para toda $i$.

Ahora demostraremos que si $g \in W_\epsilon$ entonces satisface la condici\'on \ref{lol} de la secci\'on pasada. Si esto es cierto entonces $f$ satisface la propiedad $\mathfrak{O}$. Sea $y \in N$ y $S= \{ x_i \}_{i=1}^s \subset U \cap g^{-1} (y)$. Como $g \in W_\epsilon$ entonces existe $q \in Z$ tal que $j^{m+1} g(x_i) = j^{m+1} (f+q) (x_i)$ para $1 \leq i \leq s$. Por como definimos $\rho$ tenemos que $j^{m+1}(f+q) (x_i) = j^{m+1} F_q (x_i)$ para $1 \leq i \leq s$. Como $f$ es infinitesimalmente estable entonces $F_q$ tambi\'en lo es. Entonces $F_q$ satisface $(\ast)$, por lo tanto $g$ tambi\'en. $\blacksquare$.

Con la ayuda de esta proposici\'on casi obtenemos lo que quer\'iamos, ahora nuestro prop\'osito ser\'a demostrar que estabilidad bajo k-deformaciones implica estabilidad infinitesimal.

\begin{defi}
Sea $f:M \to N$ suave, $V \subset \mathbb{R}^k$ abierto vecindad del $0$ y $F: M \times V \to N \times V$ k-deformaci\'on de $f$. Definimos el campo vectorial sobre $F$

$$\omega^{i}_F= (dF) \left( \frac{\partial}{\partial t_i} \right) - F^\ast \left( \frac{\partial}{\partial t_i} \right),$$
donde $ \{ t_i \}_{i=1}^k$ son las funciones coordenadas en $V$.
\end{defi}

Recordemos que una deformaci\'on en sus \'ultimas coordenadas es la identidad, en este campo vectorial nos ``olvidamos" de lo que pasa aqu\'i y de cierta manera estamos midiendo c\'omo cambian las $F_v$ con respecto al tiempo. Sea $\pi_M: M \times V \to M$ y $\pi_V: M \times V \to V$ las proyecciones can\'onicas. Entonces $T(M \times V ) = \pi_M^\ast (TM) \bigoplus \pi^\ast_V(TV)$. Entonces tenemos el siguiente lema, no lo demostraremos ya que es evidente.

\begin{lem}
Sea $F$ una $k$-deformaci\'on de $f$. Entonces $F \equiv f \times Id_v$ si y s\'olo si $\omega_F^{i} \equiv 0$ para $1 \leq i \leq k$.
\end{lem}

\begin{theorem}
(Thom-Levine) Sea $f:M \to N$ suave y $F: M \times V \to N \times V$ una $k$-deformaci\'on de $f$. Entonces $F$ es trivial si y s\'olo si existe $U \subset V$ vecindad del $0$ y campos vectoriales $X^{i}$ en $M \times V$ y $Y^{i} $ en $N \times U$ con $1 \leq i \leq k$ que satisfacen 
\begin{enumerate}
\item $\pi_V (X^{i})= \pi_V (Y^{i}) = 0$ y
\item $\omega^{i}_F =(dF)(X^{i}) + F^\ast (Y^{i})$ en $M \times U$.
\end{enumerate}
\end{theorem}

Para demostrar este teorema necesitaremos los siguientes lemas.

\begin{lem}
Sea $X$ un campo vectorial en $M \times \mathbb{R}^k$ con soporte compacto tal que $\pi_V (X)=0$. Entonces existe un difeomorfismo $G: M \times \mathbb{R}^k \to M \times \mathbb{R}^k$ deformaci\'on de $Id_M$, tal que $$(dG)(G^{-1})^\ast \left( \frac{\partial}{\partial t_k}\right) = X + \frac{\partial}{\partial t_k} .$$
\end{lem}

\underline{Demostraci\'on:} Como $X$ tiene soporte compacto y $\partial/\partial t_k$ nos define un flujo completo, es decir, existe $\phi: (M \times \mathbb{R}^k) \times \mathbb{R} \to X \times \mathbb{R}$.

Sea $\{ e_i \}_{i=1}^k \subset \mathbb{R}^k$ la base can\'onica. Aseguramos que la soluci\'on $\phi_s: M \times \{ v \} \to M \times \{ v + s e_k \}$. Sea $(x,v) \in M \times \{ v \}$ entonces $$(d \pi_{\mathbb{R}^k})_{(x,v)} \left( \left( X + \frac{\partial}{\partial t_k} \right) \vert_{(x,v)} \right) = \frac{\partial}{\partial t_k} \vert_{(x,v)}$$
por como est\'a definido $X$. Sabemos que el vector tangente a la curva $\phi_s$ en $(x,v)$ es $( X + \partial / \partial t_k)\vert_{\phi_s(x,v)}$, adem\'as $$\left ( \frac{d}{ds} \right) \pi_{\mathbb{R}^k} (\phi_s (x,v))= e_k .$$
Definimos la siguiente k-deformaci\'on de $Id_M$, $G: M \times \mathbb{R}^k \to M \times \mathbb{R}^k$ como $G(x,v) = (\phi_{v_k}(x, v- v_k e_k) , v)$, es claro que efectivamente es una deformaci\'on (las curvas integrales son difeomorfismos) y adem\'as por como la definimos cumplir\'a lo deseado. $\blacksquare$

\begin{lem}
Con la notaci\'on anterior;
\item $X = \pi_M (dG) (G^{-1})^\ast  (\partial /\partial t_k) .$
\item $X = - (dG) \pi_M (dG^{-1})(\partial / \partial t_k) .$
\end{lem}

\underline{Demostraci\'on:} \begin{enumerate}
\item Como $\pi_M (\partial / \partial t_i) = 0$ aplicamos $\pi_M$ a la igualdad del lema anterior y obtenemos lo deseado.

\item  Aplicamos $((dG)^{-1})_{(x,v)}$ en ambos lados de la igualdad del lema anterior y obtenemos $$(dG)^{-1}_{(x,v)} \left( \left( X + \frac{\partial}{\partial t_k} \right) \vert_{(x.v)} \right) = \frac{\partial}{\partial t_k} \vert_{G^{-1} (x,v)} .$$
Entonces $$0 = \pi_M \left( \frac{\partial}{\partial t_i} \vert_{G^{-1} (x,v)} \right) = \pi_M (dG)^{-1}_{(x,v)} (X_{(x,v)}) + \pi_M (dG)^{-1}_{(x,v)} \left( \frac{\partial}{\partial t_i} \vert_{(x,v)} \right) . $$

Como las \'ultimas k entradas de $X$ son $0$ entonces tambi\'en lo son las de $(dG)^{-1} (X)$, entonces $\pi_M (dG)^{-1}_{(x,v)} (X_{(x,v)} = (dG)^{-1}_{(x,v)} (X_{(x,v)}$. A esta \'ultima igualdad le aplicamos $(dG)_{G^{-1} (x,v)}$ para obtener 2. $\blacksquare$.


\end{enumerate}

\underline{Demostraci\'on del teorema de Thom-Levine:} Necesidad: Supongamos que $F: M \times V \to N \times V$ es trivial, entonces existe $U \subset V$ vecindad del $0$ y difeomorfismos $G: M \times U \to M \times U$ y $H: N \times U \times N \times U$ deformaciones de $Id_M$ y $Id_N$ respectivamente tales que $ H \circ F = (f \times Id_U) \circ G$. Tambi\'en sabemos que $$(dF)_{(x,v)} \left( \frac{\partial}{\partial t_i} \vert_{(x,v)} \right) = \frac{\partial}{\partial t_i} \vert_{F(x,v)}$$
ya que en $U$ la deformaci\'on se comporta como la identidad, en general esto pasa para cualquier $k$-deformaci\'on (sea trivial o no). Entonces $$(dF)_{(x,v)} \left( \frac{\partial}{\partial t_i} \vert_{(x,v)} \right) = \frac{\partial}{\partial t_i} \vert_{(x,v)} \vert_{F(x,v)} + \pi_M (dF)_{(x,v)} \left( \frac{\partial}{\partial t_i} \vert_{(x,v)} \right)$$
esto pasa tambi\'en para cualquier deformaci\'on. Sea $y= (f \times Id_U) \circ G^{-1} (x,v)$. Entonces $$ (\bigast) =(dF)_{(x,v)} \left( \frac{\partial}{\partial t_i} \vert_{(x,v)} \right)  =$$ $$ \frac{\partial}{\partial t_i} \vert_{(x,v)} + \pi_M (dH)_y \left( \frac{\partial}{\partial t_i} \vert_y \right) +$$ $$ (dH)_y (df \times Id_U)_{G^{-1} (x,v)} \pi_M (dG^{-1})_{(x,v)} \left( \frac{\partial}{\partial t_i} \vert_{(x,v)} \right) .$$

Sea $X^{i}_{(x,v)} =  (dG)_{G^{-1} (x,v)} \pi_M (dF)_{(x,v)} ( \partial / \partial t_i  \vert_{(x,v)} )$, entonces $X^{i}$ es un campo vectorial en $M \times U$ que cumple que $\pi_U (X^{i}) = \pi_U \circ \pi_M (dG^{-1})( \partial / \partial t_i )= 0$.

Aplicamos $(dG)^{-1}_{(x,v)} \circ (dG)_{G^{-1} (x,v)}$ antes de $\pi$ en $(\bigast)$ y obtenemos $$(dF)_{(x,v)} \left( \frac{\partial }{\partial t_i} \vert_{(x,v)} \right) = \frac{\partial}{\partial t_i} \vert_{F(x,v)} + \pi_M (dH)_y \left( \frac{\partial}{\partial t_i} \vert_{(x,v)} \right) + (dF)_{(x,v)} (X^{i}_{(x,v)}) .$$

Entonces

$$\omega^{i}_F (x,v) = (dF)_{(x,v)} \left( \frac{\partial}{\partial t_i} \vert_{(x,v)} \right) - \frac{\partial}{\partial t_i} \vert_{F(x,v)}  =$$ $$ \pi_M (dH)_y \left( \frac{\partial}{\partial t_i } \vert_y \right)  + (dF)_{(x,v)} (X^{i}_{(x,v)}) .$$

Definimos $Y^{i}_{(x^\prime , v^\prime )} = \pi_M (dH)_{H^{-1} (x^\prime , v^\prime )} (( \partial / \partial t_i \vert_{H^{-1} (x^\prime , v^\prime )})$ donde $(x^\prime , v^\prime) \in \\ N \times U$, es claro que $Y^{i}$ cumple lo deseado, adem\'as como la $i$ fue arbitraria terminamos.

Suficiencia: Sea $F: M \times V \to N \times V$ $k$-deformaci\'on de $f$, $X^{i}$ y $Y^{i}$ campos vectoriales en $M \times V$ y $N \times V$ respectivamente que cumplan nuestras hip\'otesis. Como $M$ es compacta podemos asumir que $X^{i}$ tiene soporte compacto y adem\'as que $Y^{i}$ tambi\'en lo es. Por los lemas anteriores existen difeomorfismos $G: M \times V \to M \times V $ y $H: N \times V \to N \times V$ $k$-deformaciones de $Id_M$ y $Id_N$ respectivamente, tales que $$-X^k = \pi_M (dG) (G^{-1})^\ast \left( \frac{\partial}{\partial t_k} \right) \quad \text{  y  } \quad  Y^k= -(dH) \pi_M (dH^{-1}) \left( \frac{\partial}{\partial t_k} \right) .$$

Sea $x= (x,v) \in M \times U$, $y= G(x)$ y $z= F \circ G(x)$, sea $\overbar{F} = H^{-1} \circ F \circ G$, entonces tenemos
$$(d \overbar{F})_x \left( \frac{\partial}{\partial t_k} \vert_x \right) = \frac{\partial}{\partial t_k} \vert_{\overbar{F}(x)} + \pi_M (dH^{-1})_z \left( \frac{\partial}{\partial t_k} \vert_z \right) +$$ $$ (dH^{-1})_z \pi_M (dF)_y \left( \frac{\partial}{\partial t_k} \vert_y \right) + (dH^{-1})_z (dF)_y \pi_M (dG)_x \left( \frac{\partial}{\partial t_k} \vert_x \right) =$$ $$\frac{\partial}{\partial t_k} \vert_{\overbar{F} (x)} + (dH^{-1})_z(-Y^k_z + \pi_M (dF)_y \left( \frac{\partial }{\partial t_k} \vert_y \right) - (dF)_y(X^k_x) .$$
Por hip\'otesis sabemos que $\omega^k_F (y) =(dF)_y (X^k_y) + Y^k_z$, entonces  $$\omega_{\overbar{F}}^k (x) = (d \overbar{F})_x (\frac{\partial}{\partial t_k} \vert_x) - \overbar{F}^\ast (\frac{\partial}{\partial t_k})=$$ $$=(dH^{-1})_z (- \omega_F^k (y) + \pi_M (dF)_y (\frac{\partial}{\partial t_k} \vert_y)) = 0 .$$

Entonces nuestra deformaci\'on en su \'ultima coordenada se comporta como la identidad, entonces nada m\'as tenemos que ver que es una deformaci\'on trivial para $i \leq k-1$. Notemos que $\pi_M (d \overbar{F} )(\partial / \partial t_i \vert_G)   = \omega_{\overbar{F}}^{i} \circ G = (dH) \omega_F^{-1} - (d \overbar{F}) \pi_M (dG)(\partial / \partial t )$. Tenemos
$$ (dH)(\omega_F^{i}) = \pi_M (d \overbar{F})(dG) \left( \frac{\partial}{\partial t_i} \right) = \pi_M (d \overbar{F}) (\frac{\partial}{\partial t_i} \vert_G) + (dG) \left( \frac{\partial}{\partial t_i} \right) =$$ $$\pi_M (d \overbar{F}) \left( \frac{\partial}{\partial t_i} \vert_G \right) + (dM) \pi_M (dG) \left( \frac{\partial}{\partial t_i} \right) .$$ 

Definimos $\overbar{X}_{G(x)} = (dG)_x (X^{i}_x)$ y $\overbar{Y}^{i}_x = (dH)_{H^{-1}(y)} (Y^{i}_{H^{-1}(y)})$ campos vectoriales en $M \times U$ y $N \times V$ respectivamente cuyas \'ultimas k-entradas son $0$, y adem\'as $\omega_{\overbar{F}}^{i} (G(y))= (d \overbar{F})_{G(y)} (\overbar{X}_{G(y)}) + \overbar{Y}^{i}_{\overbar{F} (G(y))}$. $\blacksquare$

\begin{prop}
Sea $f: M \to N$ estable bajo $k$-deformaciones entonces $f$ es infinitesimalmente estable.
\end{prop}

\underline{Demostraci\'on:} Como $f$ es estable bajo k-deformaciones entonces en particular es homot\'opicamente estable. Sea $\omega \in C^\infty_f (M , TN)$, entonces tenemos que encontrar $X \in C^\infty (TM)$ y $Y \in C^\infty(TN)$ que cumplan nuestra definici\'on. Sea $M_f$ la gr\'afica de $f$ en $M \times N$, podemos pensar a $\omega$ como un campo vectorial sobre $M_f$ de la siguiente manera, $\overbar{\omega}_{(x,f(x)} = (0, \omega_x) \in T_x M \bigoplus T_{f(x)} N$, como $M$ es compacta entonces $M_f$ es compacta, por lo tanto, $\overbar{\omega}$ tiene soporte compacto, extendemos a $\overbar{\omega}$ a todo $M \times N$ de modo que siga teniendo soporte compacto. Entonces $\overbar{\omega}$ nos define un flujo maximal $\phi: M \times N \times \mathbb{R} \to M \times N$.

Sea $F: M \times \mathbb{R} \to M \times \mathbb{R}$ donde $F(x,t)= (\pi_N \circ \phi_t (x , f(x)), t)$ donde $\pi_N : M \times N \to N$ es la proyecci\'on can\'onica y $\phi_t: M \times N \to M \times N$ es la soluci\'on en el tiempo $t$. $F(x,0)= (\pi_n \circ \phi_0 (x, f(x) , t) = (\pi_N (x. f(x)) , t) = (f(x),t)$ por lo tanto es una deformaci\'on de $f$. Como $f$ es homot\'opicamente estable por el teorema de Thom-Levine existen campos vectoriales $\overbar{X}$ y $\overbar{Y}$ en $M \times I_\delta$ y $N \times I_\delta$ tales que $\omega_F = (dF)(\overbar{X}) + F^\ast (\overbar{Y})$ con $\delta >0$. Restringimos la \'ultima ecuaci\'on a $M \times \{ 0 \}$ y obtenemos $\omega_F = (df)(X) + f^\ast (Y)$ donde $X= \overbar{X} \vert_{M \times \{ 0 \}}$ y $Y= \overbar{Y} \vert_{M \times  \{0 \}}$, adem\'as por como definimos $F$ tenemos que $\omega_F \vert_{(x,0)} = \omega_x$. $\blacksquare$

Nos gustar\'ia probar la suficiencia de este teorema, aunque la necesidad fue bastante f\'acil la suficiencia no lo ser\'a, primero resolveremos el caso local en el cual podemos ocupamos teorema de Malgrange.

\begin{prop}
Sea $f: M \to N$ infinitesimalmente estable en $x \in M$ y $f(x)=y$. Sea $F: M \times V \to N \times V$ deformaci\'on de $f$ y $\overbar{\omega} \in C^\infty_{(x,0)} (F^\ast T(N \times V))$ con $\pi_V (\overbar{\omega}) = 0$. Entonces existen $\overbar{\tau} \in C^\infty_{(x,0)} (T(M \times V)$ y $\overbar{\eta} \in C^\infty_y (T(N \times V))$ tales que $$\overbar{\omega} =  (dF)(\overbar{\tau}) + F^\ast (\overbar{\eta})$$ y $\pi_V = (\overbar{\tau}) = \pi_V( \overbar{\eta}) = 0$.


\end{prop}

\underline{Demostraci\'on:} Sea $ A := \{ \overbar{\omega} \in C^\infty_{(x,0)} (F^\ast (T(N \times V) \mid \pi_V (\overbar{\omega} ) = 0 \}$ y sea $B := \{ (df)(\overbar{\tau}) \mid \tau \in C^\infty_{(x,0)} \mid \overbar{\tau} \in C^\infty_{(x,0)} (T(M \times V) , \text{  } \pi_V(\overbar{\tau}) = 0 \}$, como $F$ es una deformaci\'on de $f$ no hay ambigüedad si ponemos $(df)$ en lugar de $(dF)$. Por como definimos a $A$ es un $C^\infty_{(x,0)} (M \times V)$, si $\overbar{\omega} \in A$, entonces $$\overbar{\omega} = \sum\limits_{i=1}^n \overbar{\omega_i} \frac{\partial}{\partial y_i} ,$$ donde $\overbar{\omega_i} \in C^\infty_{(x,0)}(M \times V)$, entonces los generadores de $A$ visto como \\ $C^\infty_{(x,0)} (M \times V)$ m\'odulo son $\{ F^\ast (\partial / \partial y_i ) \}_{i=1}^n$. Sea $C= A/B$, entonces $C$ es un $C^\infty_{(x,0)} (M \times V)$ m\'odulo finitamente generado por ser cociente de un m\'odulo finitamente generado. V\'ia $F^\ast$ $C$ es un $C^\infty_{(y,0)} (N \times V)$ m\'odulo, queremos demostrar que $C$ es generado por $\{ \pi (F^\ast ( \partial /\partial y_i) \}_{i=1}^n$ donde $\pi:A \to C $ es la proyecci\'on can\'onica.

Elegimos un representante $\omega$ de $\overbar{\omega}$, entonces por el teorema de Taylor tenemos $$\omega (x,t) = \omega(x,0) + \sum\limits_{i=1}^k t_i \omega_i (x,y) ,$$ donde $\omega(x,0) $ es un campo vectorial sobre $f$, entonces como $f$ es infinitesimalmente estable existen campos vectoriales $X$ y $Y$ en $M$ y $N$ respectivamente tales que $\omega (x,0) = (df)(X) + F^\ast (Y)$, extendemos a $X$ a $M \times V$ de modo que $\pi_V (X)=0$, lo mismo para $Y$. Aplicamos el teorema de Taylor $$\omega (x,0) - ((dF) (X) + F^\ast (Y))(x,t) = \sum\limits_{i=1}^k t_i \omega_i^\prime (x,t) .$$

$$\omega(x,t)= ((dF)(X) + F^\ast(Y))(x,t) + \sum\limits_{i=1}^k t_i \omega_i^{\prime \prime} (x,t) .$$

Obtenemos esta igualdad de las dos anteriores. Consideremos el espacio vectorial $C/ \langle t_i \rangle_{i=1}^k C$, la clase de $\overbar{\omega}$ est\'a generado por $F^\ast (Y)$, donde $F^\ast (\overbar{Y}) = \sum\limits_{i=1}^n (\overbar{Y_i} \circ f)( \partial / \partial y_i )$, recordemos que $Y$ cumple que $\pi_V (Y) = 0$, entonces $\{ \pi (F^\ast ( \partial / \partial y_i ))\}_{i=1}^n$ generan a este espacio vectorial.

Consideremos $C /( \mathfrak{m}_{(y,0)} (C^\infty (N \times V) )C$ espacio vectorial, como $\{ t_i \}_{i=1}^k \subset  \mathfrak{m}_{(y,0)} (C^\infty (N \times V)$ entonces podemos proyectar el espacio del p\'arrafo anterior en este, aplicamos el teorema de Malgrange para obtener lo deseado, por lo tanto, $$\overbar{\omega}_F = (dF) (\overbar{X}) + F^\ast ( \sum\limits_{i=1} \left( \overbar{Y }\frac{\partial}{\partial y_i} \right) , $$ aplicamos el teorema de Thom-Levine (en realidad una versi\'on local de este teorema) para obtener que $F$ es trivial, por lo tanto, $f$ es estable bajo k-deformaciones. $\blacksquare$

\begin{coro}
Sea $f$ infinitesimalmente estable y $F: M \times V \to M \times V$ una k-deformaci\'on de $f$. Sea $S= \{ x_i \}_{i=1}s =f^{-1}(y)$, entonces existe una vecindad $U$ de $S \times \{ 0 \}$ en $M \times V$ y campos vectoriales $X$ y $Y$ en $M \times V$ y $N \times V$ respectivamente con $\pi_V(X)=\pi_V(Y)=0$ y $\omega_F =(dF) (X) + F^\ast (Y)$ en $U$.


\end{coro}

Este corolario se resuelve con una pequeña modificaci\'on a la proposici\'on anterior. Ahora procedemos de la misma manera que en la secci\'on anterior y resolveremos el problema en una vecindad de los puntos cr\'iticos de $f$. Recordemos que al conjunto de puntos cr\'iticos de $f$ lo denotamos como $\Sigma$ y al conjunto de puntos cr\'iticos en $f^{-1} (y)$ lo denotamos como $\Sigma_y$, y adem\'as si $f$ es infinitesimalmente estable $f^{-1}(y)$ tiene a lo m\'as $\dim (N)$ puntos.

\begin{theorem}
Sea $f:M \to N$ y $F:M \times V \to N \times V$ deformaci\'on de $f$. Entonces existen campos vectoriales $X$ y $Y$ en $M \times V$ y $N \times V$ respectivamente y una vecindad $Z \subset M \times V$ de $\Sigma \times \{ 0 \}$ tal que $\omega_F = (dF)(X) + F^\ast (Y)$ en $Z$.
\end{theorem}
\underline{Demostraci\'on:} Recordemos que $f(\Sigma)$ es compacto, entonces existen abiertos $\{ U_i \}_{i=1}^j \subset M$, $\{ V_i \}_{i=1}^j \subset N$, $\{ W_i \}_{i=1}^j$ y $\epsilon >0$ tales que: 
\begin{enumerate}
\item $f(\Sigma) \subset \bigcup\limits_{i=1}^j W_i$,
\item $\overbar{W_i} \subset V_i$,
\item Para todo $v \in V$ con $\vert v \vert < \epsilon$, $F_v^{-1}(\overbar{W_i}) \cap \Sigma \subset U_i$,
\item Si $\vert v \vert < \epsilon$, entonces $U_i \subset F_v^{-1} (V_i)$ y
\item Existen campos vectoriales $X_i $ y $Y_i$ en $U_i \times B_\epsilon(0)$ y $V_i \times B_\epsilon (0)$ tales que $\pi_V(X_i) = \pi_V (Y_i) = 0$ tales que $$\omega_F =(dF)(X_i) + F^\ast (Y_i) \quad \text{ en } U_i \times B_\epsilon(0) .$$
\end{enumerate}

Este problema lo resolvemos primero para $f^{-1}(y)$ y ocupamos el corolario anterior, luego por la compacidad existe la $j$. Elegimos una partici\'on de la unidad $\{ \rho_i \}_{i=1}^j$ en $\bigcup\limits_{i=1}^j$ tal que $\sop (\rho_i) \subset W_i$. Sea $U$ vecindad de $\Sigma$ tal que $f^{-1} (\overbar{W_i}) \cap U \subset U_i$ para todo $i$. Tomemos una funci\'on $\rho:M \to \mathbb{R}$ tal que $\sop (\rho) \subset U$ y $\rho \equiv 1$ en una vecindad de $\Sigma$. 

Sea $X = \sum\limits_{i=1}^j \rho F^\ast \rho_i X_i$ y $Y = \sum\limits_{i=1}^j F^\ast (\rho_i Y_i)$ entonces en una vecindad de $\Sigma \times \{0 \}$ obtenemos $$(dF)(X) + F^\ast (Y) = (dF) \left( \sum\limits_{i=1}^j \rho F^\ast \rho_i X_i \right) + \sum\limits_{i=1}^j F^\ast (\rho_i Y_i) =$$ $$\sum\limits_{i=1}^j F^\ast \rho_i ((dF)(X_i) + F^\ast (Y_i)) = \omega_F .$$

Lo cual es lo que deseamos. $\blacksquare$

\begin{theorem}
Sea $f:M^m \to N^n$ infinitesimalmente estable, entonces $f$ es estable bajo $k$-deformaciones.
\end{theorem}

\underline{Demostraci\'on:} Si $m < n$ acabamos de demostrarlo. Podemos suponer que $f$ es una sumersi\'on, ya que si demostramos que las sumersiones son estables bajo k-deformaciones entonces con un argumento de particiones de la unidad podemos resolver el problema global. Sea $F: M \times V \to N \times V$ deformaci\'on de $f$, para $v \in V$ pequeña $F_v :M \to N$ es una sumersi\'on. Sea $H$ el subhaz de $TM$ ortogonal al subhaz $\ker(dF)$, entonces $T(M \times V) = \ker(dF) \bigoplus H \bigoplus TV$.

El mapeo restricci\'on $(dF): H \to TN$ es un isomorfismo, entonces existe $X$ secci\'on de $H$ tal que $(dF)(X) = \omega_F$. Y el teorema de Thom-Levine implica el teorema. $\blacksquare$

Notemos que en el \'ultimo teorema la $k$ fue arbitraria, entonces si $f$ es infinitesimalmente estable es estable bajo k-deformaciones para cualquier $k$. Como corolario de esta secci\'on obtenemos

\begin{coro}
Sea $f:M \to N$ infinitesimalmente estable entonces $f$ es estable.
\end{coro}

\section{Estabilidad transversal}

En esta secci\'on nos interesa demostrar que estabilidad implica estabilidad infinitesimal, para eso introduciremos una nueva noci\'on de estabilidad, que ser\'a equivalente a todas las anteriores. Atacaremos el problema de la misma manera que lo hicimos anteriormente, primero atacaremos el problema de una manera local para luego hacerlo globalmente. Recordemos que $G= \dif (M) \times \dif (N)$ es un grupo que act\'ua en $J^k (M,N)$ donde $(g,h) \cdot \sigma = j^k h(y) \circ \sigma j^k g^{-1} (g(x)) $ donde $\sigma \in J^k (M,N)_{(x,y)}$. Denotemos a $\mathfrak{D}_\sigma$ a la \'orbita de $\sigma$ bajo la acci\'on de $G$. 

Sea $G^k (M)_x \subset J^k(M,M)_{(x,x)}$ y $G^k (N)_y \subset G^k (N)_{(y,y)}$ los grupos de k-jets invertibles, y $G^k = G^k (M)_x \times G^k (N)_{(y,y)}$, por como lo definimos $G^k$ es un grupo de Lie, el cual act\'ua en $J^k (M,N)_{(x,y)}$ como $(\tau, \xi ) \cdot \sigma = \xi \circ \sigma \circ \tau^{-1}$ donde $\sigma \in J^k(M,N)_{(x,y)}$. Denotemos por $\mathfrak{O}_\sigma$ a la \'orbita de $\sigma$ en $J^k(M,N)_{(m,n)}$ bajo esta acci\'on, y $\overbar{\mathfrak{O}}_\sigma$ la componente conexa de $\mathfrak{O}_\sigma$ que contiene a $\sigma$. Como $G^k$ es un grupo de Lie, entonces $\overbar{\mathfrak{O}}_\sigma$ es una subvariedad inmersa de $J^k(M,N)_{(x,y)}$.

\begin{lem}
La componente conexa de la identidad en $GL(n)$ es el conjunto de las matrices de determinante positivo.
\end{lem}

\begin{lem}
Sea $T_a: \mathbb{R}^n \to \mathbb{R}^n$ la traslaci\'on por $a \in \mathbb{R}^n$. Sea $U \subset \mathbb{R}^n$ abierto, entonces existe un difeomorfismo $\eta: \mathbb{R}^n \to \mathbb{R}^n$ tal que $\eta = T_a$ en $U$ y $\eta=Id_{\mathbb{R}^n}$ fuera de alg\'un compacto.
\end{lem}

\begin{lem}
Sea $\phi: \mathbb{R}^n \to \mathbb{R}^n$ difeomorfismo tal que $\phi(0) = 0$. Entonces existe un difeomorfismo $\eta: \mathbb{R}^n \to \mathbb{R}^n$ tal que $\eta= \phi$ en una vecindad del $0$ y $\eta=\phi$ fuera de alg\'un compacto.
\end{lem}

Estos lemas no los demostraremos ya que son simples las demostraciones aunque un poco largas y los m\'etodos usados no son de nuestro inter\'es.

\begin{theorem}
$\mathfrak{D}_\sigma$ es una subvariedad inmersa de $J^k (M,N)$.
\end{theorem}

\underline{Demostraci\'on}: Sean $x \in M $ y $y \in N$, $U$ y $V$ cartas coordenadas alrededor de $x$ y de $y$ respectivamente, podemos suponer que $U= \mathbb{R}^m$ y $V = \mathbb{R}^n$ con $x= 0$ y $y=0$. Sea $T: U \times V \times J^k (U,V)_{(x,y)} \to J^k(U,V)$ dado por $T(x^\prime, y\prime , \tau ) = j^k T_{y^\prime} \circ \sigma j^k T_{x^\prime}$ donde $T_{x^\prime}$ y $T_{y^\prime}$ son las traslaciones por $x^\prime$ y $y^\prime$ respectivamente, el cual es un difeomorfismo por ser una carta de $J^k (M,N)$.

Sea $\mathfrak{D}_\sigma^{U,V}$ la componente conexa de $\mathfrak{D}_\sigma \cap J^k (U,V)$ que contiene a $\sigma$, si demostramos que $\mathfrak{D}_\sigma \cap J^k (U,V)$ tiene un conjunto numerable de componentes y que $T(U \times V \times \overbar{\mathfrak{O}}_\sigma ) = \mathfrak{D}_\sigma^{U,V}$ tendremos el teorema. Sea $\overbar{G}^k$ la componente conexa de $G^k$, entonces es claro que $\overbar{\mathfrak{O}}_\sigma = \overbar{G}^k \cdot \sigma$. Sea $\tau \in \overbar{\mathfrak{O}}_\sigma$ entonces $\tau = \overbar{\beta} \circ \sigma  \circ \overbar{\alpha}^{-1}$ donde $\overbar{\alpha} \in \overbar{G}^k(M)_x $ y $\overbar{\beta} \in G^k(N)_y$, podemos suponer que $\alpha: M \to M$ representante de $\overbar{\alpha}$ es un difeomorfismo global, lo mismo para $\beta:N \to N$ representante de $\overbar{\beta}$.

Entonces $\tau = j^k \beta(y) \circ \sigma \circ j^k (\alpha^{-1})(x)$. Por otro lado $T(x^\prime , y^\prime , \tau) = j^k T_{y^\prime} \circ \tau \circ j^k (T_{x^\prime}^{-1})(x^\prime)$, podemos asumir que $T_{x^\prime} : M \to M$ y $T_{y^\prime}: N \to N$ son difeomorfismos globales. Por lo tanto, $T(x^\prime, y^\prime , \tau ) = (T_{x^\prime} \circ \alpha , T_{y^\prime} \circ \beta ) \cdot \tau \in \mathfrak{D}_\sigma$, entonces tenemos la primera contenci\'on.

Para la otra contenci\'on, sea $\tau \in \mathfrak{D}_\sigma^{U,V}$, $x^\prime = \alpha (\tau)$ y $y^\prime = \beta (\tau)$. Sea $\rho= j^k (T_{q^\prime}^{-1}) (y^\prime) \circ \tau \circ j^k T_{x^\prime} (x)$, como las traslaciones son difeomorfismos tenemos que $\rho \in \mathfrak{D}_\sigma$, entonces existen difeomorfismos $(\gamma , \xi) \in G$ tal que $\rho = j^k \xi (y) \circ \omega \circ j^k (\gamma^{-1})(x)$, entonces $\rho \in \mathfrak{O}_\sigma$, entonces $\mathfrak{O}_\sigma \cap J^k (U,V) \subset T(U \times V \times \overbar{\mathfrak{O}}_\sigma$, como $U$ y $V$ fueron cartas $J^k (U,V) \cap \mathfrak{D}_\sigma$ tiene un conjunto numerable de componentes. $\blacksquare$

\begin{defi}
Sea $f: M^m \to N^n$ y $x \in M$ con $\sigma = j^n f(x)$. Entonces $f$ es \textit{transversalmente} estable en $x$ si $j^n f(x) \transv \mathfrak{D}_\sigma$ en $x$.
\end{defi}

\begin{lem}
Sea $f: M \to N$ estable entonces $f$ es transversalmente estable en $x \in M$ para toda $x$.
\end{lem}


Este lema es una consecuencia trivial del teorema de transversalidad de Thom. Sea $f: M \to N$, $\omega \in C^\infty (f^\ast TN)$ y $\overbar{\omega} \in J^k (f^\ast TN)_x$ un representante de $\omega$. Sea $F: M \times I \to N \times I$ deformaci\'on de $f$ tal que $dF / dt \vert_{t=0} = \omega$. Consederemos la curva $t \to j^k F_t (x)$ en $J^k (M,N)$ basada en $\omega$, sea $\lambda (\omega)$ el vector tangente en $t=0$. Como nuestra definici\'on de estabilidad transversal depende de una condici\'on de transversalidad nos ayudar\'a en el futuro calcular los espacios tangentes.
\begin{prop}
$\lambda: J^k (f^\ast TN) \to T_{\overbar{\omega}} J^k (M,N)$ es un monomorfismo de espacios vectoriales.
\end{prop}
\underline{Demostraci\'on:} Escribimos a $\omega= \sum\limits_{i=1}^n g_i f^\ast (\partial / \partial y_i )$ con $g_i$ funci\'on suave. Para $t$ cercano a $0$ podemos escribir a $F_t^j = f_j + t g_j + O( t^2) $, entonces

$$j^k F_t^j (0) = \sum\limits_{\vert a \vert \leq k}  \frac{x^a}{a !} \frac{\partial^{\vert a \vert}}{\partial x^a} F_t^j (0) = \sum\limits_{\vert a \vert} \frac{x^a}{a !} \left( \frac{\partial^{\vert a \vert}}{\partial x^a} f_j (0)        + t \frac{\partial^{\vert a \vert}}{\partial x^a} g_j (0) + O(t^2)\right) ,$$ derivamos con respecto a $t$ y obtenemos 
$$\frac{\partial}{\partial t} j^k F_t^j (0)\vert_0 = \sum\limits_{\vert \alpha \vert \leq k} \frac{x^a}{\alpha !} \frac{\partial^{\vert a \vert}}{\partial x^a} g_j (0) .$$

Por esta \'ultima expresi\'on $\lambda$ est\'a bien definida y adem\'as lineal. Si $\lambda (\overbar{\omega})= 0$ entonces $\partial^{\vert a \vert} / \partial x^a g_j(0)=0$ para $\vert a \vert \leq k$, por lo tanto $\overbar{\omega} = 0$. $\blacksquare$

\begin{prop}
La sucesi\'on $0 \to J^k(f^\ast TN)_x \overset{\lambda}{\to} T_{\omega}J^k (M,N) \overset{(d \alpha)_\omega}{\to} \\ T_x M \to 0$ es exacta, donde $\alpha: J^k (M,N) \to M$ es el mapeo fuente.
\end{prop}

\underline{Demostraci\'on:} Como $\lambda$ es inyectiva y $\alpha$ es una sumersi\'on, nada m\'as nos hace falta probar que $\im (\lambda) = \ker (d \alpha)_\omega$. Como la curva $t \to \alpha \circ j^k F_t (p)$ es constante entonces $(d\alpha)_\omega \circ \lambda = 0$, lo cual implica que $\im (\lambda) \subseteq \ker((d\alpha)_\omega)$. Como estos vectoriales son espacios de dimensi\'on finita nada m\'as nos hace falta probar que tienen la misma dimensi\'on. $\dim (\im (\lambda)) = \dim (V^k_{n,m}) + \dim (N)$, por otro lado, $\dim (\ker (d\alpha)_\omega) = \dim (J^k(M,N)) - \dim( M) = \dim (V^k_{n,m}) + \dim (N)$, por lo tanto la sucesi\'on es exacta. $\blacksquare$

Con la notaci\'on del teorema anterior tenemos un mapeo $(d \gamma_1 ): C^\infty_y (TN) \to T_\omega \mathfrak{D}_\omega$, dado de la siguiente manera. Sea $\overbar{Y} \in C^\infty_y (TN) $ y $Y \in C^\infty (TN)$ un representante de $\overbar{Y}$, podemos elegir a $Y$ de modo que tenga soporte compacto, entonces su flujo asociado $\phi$ es maximal. Consideremos la curva $c(t) = j^k \phi_t (y) \circ \sigma$, como $\phi_t$ es un difeomorfismo esta curva se encuentra en $\mathfrak{D}_\omega$, sea $(d\gamma_1) (\overbar{Y}) = \partial c / \partial t \vert_{t=0}$. Sea $\pi^k : C^\infty_y (TN) \to J^k (TN)_y$ que a cada germen alrededor de $y$ le asigna su $k$-jet.

\begin{prop}
El siguiente diagrama conmuta$$ \xymatrix{
C^\infty_y (TN) \ar[d]^{\pi^k} \ar[r]^{(d\gamma_1)}&T_\omega \mathfrak{D}_\omega \\\
J^k (TN)_y \ar[r]^{f^\ast} & \ar[u]^\lambda  J^k(f^\ast TN)_y .}  $$
\end{prop}

\underline{Demostraci\'on:} Sean $Y$ y $\phi$ como en el p\'arrafo anterior. $$\lambda \circ f^\ast \circ \pi^k(Y) = \frac{d}{d t} j^k (\phi_t \circ f )(x) \vert_{t=0} =\frac{d}{dt} j^k \phi_t(y) \circ \omega \vert_{t=0} = (d \gamma_1)(Y) .$$

La igualdad es v\'alida ya que el flujo asociado a $Y$ es $\phi$. $\blacksquare$

De la misma manera que acabamos de definir $(d\gamma_1)$ podemos definir un mapeo $(d \gamma_2): C^\infty_x (TM) \to T_\omega \mathfrak{D}_\omega$, dado $\overbar{X} \in C^\infty_x (TM)$ elegimos un representante $ X  \in C^\infty (TM)$, como $X$ es compacta $X$ tiene asociado un flujo maximal $\psi$. Consideramos la curva $c(t)= \omega \circ \psi_t (\psi^{-1} (x))$. De la misma manera podemos definir $(d \gamma_2) (\overbar{X}) = dc / dt \vert_{t=0}$. Sea $\pi^k_0: J^k (TM)_x \to J^0 (TM)_x = T_x M$ la proyecci\'on can\'onica.

\begin{prop}
Sea $\gamma:J^k (TM)_x \to T_\omega \mathfrak{D}_\omega$ dada por $\gamma = - \lambda \circ (df) + (d j^k f)_x \circ \pi_0^k$. Entonces el siguiente diagrama conmuta$$
\xymatrix{
C^\infty_x (TM) \ar[r]^{(d\gamma_2)} \ar[d]^{\pi^k} & T_\omega \mathfrak{D}_\omega\\\
 \ar[ur]^\gamma J^k (TM)_x .}$$
\end{prop} 

\underline{Demostraci\'on:} Ocupando la misma notaci\'on, sea $X^k = \pi^k (\overbar{X})$. Sea $F_t= f \circ \phi_t$ deformaci\'on de $f$ que cumple $(\frac{dF}{dt}) \vert_t=0 = (df)(X)$. Entonces $$
\lambda \circ (df) (X^k) = \frac{d}{dt} (j^k F_t(x)) \vert_{t=0} = \frac{d}{dt} [j^k f(\psi_t(x)) \circ j^k \psi_t (x)] \vert_{t=0}=$$
$$= \frac{d}{dt} j^k f(\psi_t(p))\vert_{t=0} + \frac{d}{dt} \omega \circ j^k \psi_t (x) = (d j^k f)_x \pi^0 (X) - \frac{d}{dt} \omega \circ j^k \psi_{-t} (x) \vert_{t=0}=$$
$$(d j^k f)_x \pi^k_0 (X^k) - (d \gamma_2)(X) .$$

Esta \'ultima igualdad la obtenemos debido a que $\pi^0 = \pi^k_0 \circ \pi^k$ y $\psi_{-t} = \psi_t^{-1}$, despejando la \'ultima igualdad obtenemos lo deseado. $\blacksquare$

\begin{prop}
$(d\gamma_1) + (d \gamma_2) : C^\infty_x (TM) \bigoplus C^\infty_y (TN) \to T_\omega \mathfrak{D}_\sigma \to 0$ es exacta.
\end{prop}

\underline{Demostraci\'on:} Sea $v \in T_\omega  \mathfrak{D}_\sigma$ y $c: I \to \mathfrak{D}_\sigma$ una curva que representa a $v$. Supongamos que existe una curva de difeomorfismos en $\dif (M) \times \dif (N)$ de la forma $t \to (g_t , h_t)$ tal que $c(t) = j^k h_t (y) \circ  \omega \circ  j^k g_t (g^{-1} (t))$, primero demostraremos que si esto pasa entonces $v $ estar\'a en la imagen de $(d \gamma_1) + (d \gamma_2)$, luego demostraremos que es posible encontrar esta curva de difeomorfismos. Sean$$
X(x) = \frac{\partial g_t}{\partial t} (x) \vert_{t=0} \quad Y(y) = \frac{\partial h_t}{\partial t} (y) \vert_{t=0}$$
campos vectoriales en $M$ y $N$ respectivamente, como estamos resolviendo un problema a nivel de g\'ermenes podemos suponer que $Y$ tiene soporte compacto. Sean $\phi$ y $\psi$ los flujos maximales asociados a $X$ y $Y$ respectivamente. Entonces la curva $\overbar{c}(t) = j^k \psi_t (y) \circ \omega \circ j^k \phi_{-t} (\phi_t(x))$ satisface
$$\frac{d \overbar{x}}{dt} \vert_{t=0} = \frac{d c}{d t} \vert_{t=0} = v ,$$
por como la definimos. Entonces $$(d \gamma_1) + (d \gamma_2) (\overbar{X} . \overbar{Y}) = \frac{d \overbar{c}}{dt} =v ,$$
ya que $dg / dt \vert_{t=0} = d \phi_t /dt_{t=0}$, lo mismo para $h_t$ y $\psi_t$. Entonces ahora demostraremos que existe la curva de difeomorfismos. Recordemos que si $U$ y $V$ cartas alrededor de $x$ y $y$ entonces $T(U \times V \times \overbar{\mathfrak{O}}_\omega)$ era la componente conexa que conten\'ia a $\omega$ en $\mathfrak{O}_\omega \cap J^k (U,V)$ donde $\overbar{\mathfrak{O}}_\omega$ era la \'orbita de $\omega$ bajo la acci\'on de $G^k$. Podemos suponer que la curva $c(t)$ se encuentra en el conjunto $T(U \times V \times \overbar{\mathfrak{O}}_\omega )$. Entonces   $T^{-1} (c(t)) = (x(t) , y(t), \omega(t))$. Entonces $c(t) = j^k (T_{y(t)}) \circ \omega (t) \circ j^k (T^{-1}_{(x(t)})$, como $\omega (t) $ es una curva en $\overbar{O}_\omega$, existe una curva $(\overbar{g}_t , \overbar{h}_t)$ en $G^k (M)_x \times G^k (N)_y$ donde estos eran los k-jets invertibles de modo que $\overbar{g}_t \circ \sigma \circ \overbar{h}^{-1}_t = \omega (t)$. Sea $g_t$ el polinomio cuyo k-jet es $\overbar{g}_t$, elegimos $h_t$ de la misma manera, podemos asumir que $g_t$ y $h_t$ son difeomorfismos globales. $\blacksquare$

\begin{theorem}
Sea $f: M^m \to N^n$ suave y $x \in M$. Si $f$ es transversalmente estable en $x$ entonces $f$ es infinitesimalmente estable en $x$.
\end{theorem}

\underline{Demostraci\'on:} Sea $\omega \in J^n (f^\ast (TN))_x$, para demostrar que $f$ es infinitesimalmente estable en $x$ tenemos que encontrar $X \in J^n (TM)$ y $Y \in J^n (TN)$ tales que $\omega = (df)(X) + f^\ast (Y)$. Sea $\lambda(\omega) \in T_\omega J^n (M,N)$, existen $v \in T_\omega \mathfrak{D}_\omega$ y $u \in T_x M$ tales que $\lambda (\omega) = v + (d j^n f)_x (u)$, por el lema anterior existen $\overbar{X} \in C^\infty_x (TM)$ y $\overbar{Y} \in C^\infty_{f(x)} (TN)$ tales que $v = - (d \gamma_2)(\overbar{X}) + (d \gamma_1)(\overbar{Y})$. Sean $X= \pi^n (\overbar{X})$ y $Y= \pi^n (\overbar{Y})$, aunque la notaci\'on sea igual notemos que estamos hablando de $\pi_m$ distintas, tenemos $v= \lambda \circ f^\ast (Y) - \gamma(X)$ entonces
$$\lambda(\omega) = \lambda \circ (df)(X) - (dj^m f)_x \circ \pi^n_0(X) + \lambda \circ  f^\ast  (Y) + (d j^m f)_x(u).$$
Aplicamos $(d\alpha)_\omega$ de ambos lados, $$ 0 = u - \pi_0^n(X).$$
Entonces $\lambda(\omega) =\lambda \circ (df)(X) + \lambda \circ f^\ast (Y)$, pero $\lambda$ es inyectiva, por lo tanto $\omega = (df)(X) + f^\ast (Y)$. $\blacksquare$

Nos gustar\'ia pasar este resultado a un \'ambito global, como siempre nuestro problema son las autointersecciones, para eso introducimos la subvariedad inmersa $\mathfrak{D}_\omega^s \subset  J^k_s(M,N)$ que es la \'orbita de $\omega$ en $J^k_s(M,N)$ bajo la \'orbita de $\dif (M) \times \dif (N)$.

\begin{prop}
$\mathfrak{D}_\omega^s$ es una subvariedad inmersa de $J^k_s(M,N)$.
\end{prop}
\underline{Demostraci\'on:} Sea $\omega= (\omega_1, \dots, \omega_s) \in J^k_s(M,N)$, dividiremos la prueba en dos casos: \begin{enumerate}
\item Sea $\beta (\omega_1) = \dots = \beta (\omega_s)$, a estos elementos los llamaremos diagonales. Elegimos vecindades $\{U_i \}_{i=1}^s$ vecindades disjuntas de $\{ \alpha(\omega_u \}_{i=1}^s$ y $V$ una vecindad coordenada de $\beta(\omega_1)$. Definimos $\overbar{T}: U_1 \times \dots \times U_s \times V \times \overbar{\mathfrak{O}}_{\omega_1} \times \dots \times \overbar{\mathfrak{O}}_{\omega_s} \to J^k_s (U_1, V) \times \dots \times J^k_s(M,N)$ dada por $$\overbar{T} (x_1 , \dots , x_s , y , \sigma_1, \dots , \sigma_s)= (T(x_1 , y , \sigma_1) , \dots , T(x_s , y , \sigma_s)) .$$

De manera similar demostramos que estas son las cartas de $\mathfrak{D}_\omega^s$, el \'unico detalle que vale la pena mencionar es que si tenemos difeomorfismos definidos en $U_i$ no necesariamente iguales si $i\neq j$ lo podemos extender a uno global, como les pedimos a las vecindades que fueran ajenas esto es posible inductivamente.

\item Supongamos que $\omega$ no es diagonal, sin p\'erdida de generalidad podemos suponer que $s=2$ con $\beta (\omega_1) \neq \beta (\omega_2)$. Como ser subvariedad inmersa es algo local podemos suponer que $M= \mathbb{R}^m$ y $N = \mathbb{R}^n$, entonces $\mathfrak{D}_\omega^s = \mathfrak{D}_1 \times \mathfrak{D}_2  \cap Z$, donde $Z= \{ (\sigma_1 , \sigma_2) \in J^k_2 (\mathbb{R}^m,\mathbb{R}^n) \mid \beta (\omega_1) \neq (\omega_2)$. $\blacksquare$
\end{enumerate}

\begin{lem}
Sea $f: M \to N$ estable, entonces $f$ es transversalmente estable.
\end{lem}

Esto es una consecuencia trivial de la proposici\'on anterior y el teorema de multitransversalidad de Thom.

\begin{theorem}
Sea $f: M \to N$ transversalmente estable, entonces $f$ es infinitesimalmente estable.
\end{theorem}

\underline{Demostraci\'on:} Sea $y \in N$, y $S= \{ x_i \}_{i=1}^s \subset f^{-1}(y)$, donde $1 \leq s \leq n+1$, definimos $\lambda^s :J^m(f^\ast TN)_S = \bigoplus\limits_{i=1}^s J^n (f^\ast TN)_{x_i} \to T_\omega J^n_s (M,N)= \bigoplus\limits_{i=1}^s T_{\omega_i} J^n(M,N)$ como la suma entrada a entrada de $\lambda$, definida de esta manera $\lambda^s$ es inyectiva. Entonces la siguiente sucesi\'on es exacta
$$0 \to J^n(f^\ast TN)_S \overset{\lambda^s}{\to} T_\omega J^n_\omega(M,N) \overset{(d\alpha^s)_\omega}{\to} T_{(x_1, \dots , x_s)} M^{(s)} \to 0.$$

Definimos $(d \gamma_1^S) : C^\infty_y (TN) \to T_\omega \mathfrak{D}_\omega^s$ de la siguiente manera, sea $Y$ un campo vectorial en $N$ de soporte compacto y $\overbar{Y}$ su clase en $C^\infty_q (TN)$, sea $\phi$ el flujo maximal asociado a $Y$, sea $c(t) = j^k_s \phi_t (y) \circ \omega$, sea $(d \gamma_1^s) (\overbar{Y}) = (\frac{dc}{dt}) \vert_{t=0}$, entonces el siguiente diagrama conmuta$$
\xymatrix{
C^\infty_y (TN) \ar[d]^{\pi^k} \ar[r]^{(d\gamma_1^s)}&T_\omega \mathfrak{D}_\omega^s \\\
J^n (TN)_y \ar[r]^{f^\ast} & \ar[u]^\lambda^s  J^n(f^\ast TN)_S .} $$

De manera an\'aloga podemos definir $(d \gamma_2^s): C^\infty (TM)_S \to T_\omega \mathfrak{D}_\omega^s$ y $\gamma: J^n (TN)_S \to T_\omega \mathfrak{D}_\omega^s$ dada por $\gamma= - \lambda^s \circ df + (d j^m_s f)_s \circ \pi^n_0$, de modo que el siguiente diagrama conmuta$$
\xymatrix{
C^\infty_S (TM) \ar[r]^{(d\gamma_2^s)} \ar[d]^{\pi^k} & T_\omega \mathfrak{D}_\omega^s\\\
 \ar[ur]^{\gamma^s} J^k (TM)_x .}$$ 
 
Los diagramas anteriores conmutan debido a que en sus funciones coordenadas conmutan. De la misma manera $(d \gamma^s_1) + (d \gamma^s_2): C^\infty (TM)_S \bigoplus C^\infty (TN)_s \to T_\omega \mathfrak{D}_\omega^s$, por lo tanto, $J^n (f^\ast TN)_S = (df)(J^n (TM)_S) + f^\ast (J^n (TN)_y)$. $\blacksquare$

\begin{theorem}
Sea $M$ compacta y $f:M \to N$ suave. Entonces los siguientes son equivalentes:
\begin{enumerate}
\item $f$ es estable.
\item $f$ es infinitesimalmente estable.
\item $f$ es estable bajo $k$-deformaciones.
\item $f$ es transversalmente estable.
\end{enumerate}



\end{theorem}
\underline{Demostraci\'on:} $3$ implica $1$ es el lema 4.2.2. $3$ si y s\'olo si $2$ es la proposici\'on 4.2.12 y el teorema 4.2.16. $2$ implica $1$ es el corolorario 4.2.17. $4$ implica $2$ es el teorema 4.3.15. $1$ implica $4$ es el lema 4.3.14. $\blacksquare$

Lo que podemos preguntarnos ahora es la existencia de mapeos estables, es decir, dadas dos variedades arbitrarias, ¿existen los mapeos estables? o ¿son \'estos un conjunto denso? En el caso cuando la dimensi\'on del dominio es mayor a la del codominio las sumersiones son estables. En la siguiente secci\'on analizaremos los casos m\'as simples de mapeos estables y en el siguiente cap\'itulo analizaremos el caso general, seguiremos asumiendo que el dominio siempre es compacto.

\section{Ejemplos}

\subsection{Funciones de Morse}

Recordemos que una funci\'on $f:M \to \mathbb{R}$ es una funci\'on de Morse si $j^1 f \transv S^1$, por lo tanto, las funciones de Morse forman un conjunto denso, entonces si una funci\'on es estable esta necesariamente tiene que ser una funci\'on de Morse. Lamentablemente el regreso a esta afirmaci\'on es falso, debido al problema de las autointersecciones, ya que en un punto cr\'itico la derivada es el morfismo cero. Por suerte, las funciones de Morse cuyos puntos cr\'iiticos tienen im\'agenes distintas tambi\'en son un conjunto residual y como en este caso no hay autointersecciones formulamos la siguiente proposici\'on.

\begin{prop}
Sea $f \in C^\infty(M,\mathbb{R})$, $f$ es estable si y s\'olo si $f$ es una funci\'on de Morse cuyos puntos cr\'iticos tienen im\'agenes distintas.
\end{prop}

\underline{Demostraci\'on:} La necesidad es clara por el p\'arrafo anterior. Sea $f: M \to \mathbb{R}$ funci\'on de Morse cuyos puntos cr\'iticos tienen im\'agenes distintas. Sea $\omega \in C^\infty (f^\ast T \mathbb{R})$, como $T \mathbb{R}= \mathbb{R} \times \mathbb{R}$, $\omega$ es de la forma $\omega(x) = (f(x) , \omega(x))$, entonces podemos pensar que $\omega \in C^\infty (M , \mathbb{R})$. De la misma manera podemos pensar a un campo vectorial $Y$ en $N$ como $Y \in C^\infty (\mathbb{R} , \mathbb{R})$. 

Como $f$ es una funci\'on de Morse y $M$ es compacta, los puntos cr\'iticos de $f$ son un conjunto finito. Sea $Y \in C^\infty (\mathbb{R}, \mathbb{R})$ tal que $\omega(x) = Y \circ f(x)$ en los puntos cr\'iticos de $f$, esto es posible por lo que acabamos de aclarar. Ahora tenemos que encontrar $X \in C^\infty (TM)$ tal que $\omega = (df) (X)$ cuando $\omega (x) =0$, esto ya que podemos sustituir a $\omega $ por $\omega - Y \circ f$. Alrededor de cada punto $x \in M$ elegimos una vecindad $U_x$ de la siguiente manera:
\begin{enumerate}
\item Si $x$ es un punto regular de $f$, elegimos $U_x$ de tal modo que si $x^\prime \in U_x$ entonces $(df)_{x^\prime} \neq 0$. Elegimos un campo vectorial $X^x$ en $U_x$ tal que $(df)(X^x)  \neq 0$.

\item Si $x$ es un punto cr\'itico de $f$ elegimos una vecindad de tal modo que $f$ tenga la forma $ \sum\limits_{i=1}^m x_i^2$, asumiremos que los coeficientes de las $x_i$ ser\'an positivos, pero la demostraci\'on es igual si no lo son.
\end{enumerate}

Entonces $\{ U_x \}_{x \in M}$ es una cubierta de $M$, entonces existe una subcubierta finita $\{ U_{x_i} \}_{i \in I}$ para alg\'un conjunto de \'indices $I$, sea $\{ \rho_i \}_{i \in I}$ una partici\'on de la unidad asociada a esta subcubierta. Escogemos campos vectoriales $\{ X^{i} \}_{i \in I}$ en $M$ de la siguiente manera:

\begin{enumerate}
\item Si $x_i$ es un punto regular $$X^{i} (x) =
  \begin{cases}
		\frac{\omega(x) \rho_i (x) X^{x^{i}} (x)}{(df)_x (X^{x^{i}}(x)) }  & \mbox{si } x 	\in U_{x_i} \\
		0 & \mbox{si } x \notin U_{x_i} .
	\end{cases}$$
	
\item Si $x_i$ es un punto cr\'itico entonces $\omega(x_i) = 0$ y $\rho_i \omega = \sum\limits_{j=1}^m 	h_j x_j$ por el lema de Hadamard (podemos tomarnos a las vecindades $U_{x_i}$ convexas), adem\'as las $h_j$ tienen soporte compacto debido a que $\rho_i \omega$ lo tiene. Sea
$$X^{i} = \sum\limits_{j=1}^m \frac{h_j}{2} \frac{\partial}{\partial x_j}$$
en $U_{x_i}$ y lo extendemos suavemente fuera de $U_x$ de modo que tenga soporte compacto. Dividimos entre $2$ debido a que la derivada de una funci\'on de Morse tiene un 2 multiplicando.
\end{enumerate}

Sea $X = \sum\limits_{i \in I} X^{i}$, ahora nada m\'as tenemos que ver si efectivamente este es el campo vectorial que necesitamos.

\begin{enumerate}
\item Si $x $ es un punto regular tenemos que $$(df)_x(X(x)) = \frac{\omega (x) \rho_i (x) (df)_x X^{x^{i}} (x)}{(df)_x (X^{x^{i})} (x))} = \omega(x) \rho_i (x) = \omega (x) .$$

\item Si $x$ es un punto cr\'itico
$$(df)_x (X (x)) = \sum\limits_{j=1}^m \frac{h_j (x)}{2} \frac{\partial}{\partial x_j} (x_1^2 + \dots + x_m^2) =$$ $$ \sum\limits_{j=1}^m h_j x_j = \rho_i (x) \omega(x) = \omega(x) .$$

\end{enumerate}
Por lo tanto, $(df)(X) = \omega$. $\blacksquare$

\subsection{Inmersiones 1 a 1}

\begin{prop}
Sea $f:M \to N$ con $\dim(N) \geq 2 \dim (M) + 1$. Entonces $f$ es estable si y s\'olo si $f$ es una inmersi\'on 1 a 1.
\end{prop}

\underline{Demostraci\'on:} Necesidad. Si $f$ es una inmersi\'on 1 a 1 entonces cualquier funci\'on equivalente a $f$ es una inmersi\'on 1 a 1. Entonces si $f$ es estable existe una vecindad $W \in C^\infty (M,N)$ donde todas las funciones son equivalentes a $f$. Por el teorema del encaje de Whitney tenemos que existe $g \in W$ inmersi\'on 1 a 1 equivalente a $f$, por lo tanto, $f$ es una inmersi\'on 1 a 1.

Suficiencia. Sea $\omega$ campo vectorial sobre $f$, entonces elegimos $Y$ campo vectorial definido en $f(M)$ de modo que $\omega = f^\ast Y$, extendemos a $Y$ a todo $N$. $\blacksquare$.




\subsection{Inmersiones con cruces normales}

\begin{defi}
Sea $f: M \to N$ suave y $f^{(s)} : M^{(s)} \to N^s$ la restricci\'on de $ \prod f : M^s \to N^s$ a $M^{(s)}$. Entonces $f$ es un \textit{mapeo con cruces normales} si $f^{(s)} \transv \Delta N^s$ para toda $s >1$.
\end{defi}

Que un mapeo tenga cruces normales es una condici\'on de transversalidad por el teorema de multitransversalidad de Thom tenemos la siguiente proposici\'on.

\begin{prop}
El conjunto de mapeos con cruces normales es denso en $C^\infty (M,N)$.
\end{prop}

\begin{coro}
Las inmersiones con cruces normales son densas en el conjunto de inmersiones.
\end{coro}

\begin{prop}
Si $f:M \to N$ es una inmersi\'on estable entonces $f$ es una inmersi\'on con cruces normales.
\end{prop}

Es claro que las inmersiones 1 a 1 son inmersiones con cruces normales debido a que si $f:M \to N$ es una inmersi\'on 1 a 1, entonces $f^{(s)} (M^{(s)}) \cap \Delta Y^s = \emptyset$ para toda $s >1$. Nos gustar\'ia demostrar la suficiencia de esta proposici\'on, primero demostraremos algunos lemas que nos ayudar\'an para ver este hecho.

\begin{lem}
Sea $f:M \to N$ inmersi\'on con cruces normales, sea $y \in N$. Como $M$ es compacta y $f$ una inmersi\'on, entonces $f^{-1} (y) = \{ x_i \}_{i \in I}$ es un conjunto finito. Entonces $\{ (df)_{x_i} (T_{x_i} (M) \}_{i \in I}$ est\'an en posici\'on general, vistos como subespacios de $T_y N$.
\end{lem}

\underline{Demostraci\'on:} Sea $J \subseteq I$ con cardinalidad $s$, renombramos a $(df)_{x_j} (T_{x_j} M)$ como $H_j$ para toda $j \in J$. Denotemos a $q \in \Delta N$ a $q = (q , \dots , q)$. Por hip\'otesis tenemos que 
$$T_q N^s = \bigoplus\limits_{j \in J} H_j + T_q \Delta N .$$

Por el teorema de la dimensi\'on tenemos que 
$$s \cdot \dim (N) = \dim (\bigoplus\limits_{j=1}^s H_j) + \dim (N)- \dim (\bigoplus\limits_{j=1}^s H_j \cap T_q \Delta N^s) .$$

Despejando $\sum\limits_{j=1}^s \cod (H_j) = \cod (\bigoplus\limits_{j=1}^s H_j \cap T_q \Delta N^s)$, pero $\bigoplus\limits_{j=1}^s H_j \cap T_q \Delta N^s = \bigcap\limits_{j=1}^s H_j$, identificando a $T_q \Delta N^s$ con $T_q N$. $\blacksquare$

\begin{lem}
Sea $V$ un $\mathbb{R}$-espacio vectorial de dimensi\'on finita, $\{ H_i \}_{i=1}^s \subset V$ familia de subespacios de $V$. Sea $W = \bigcap\limits_{i=1}^s H_i$, entonces existen $\{ F_i \}_{i=1}^s \subset V$ familia de subespacios de $V$ tales que 
\begin{enumerate}
\item $V = D \bigoplus (\bigoplus\limits_{i=1}^s F_i)$,
\item $H_i = D \bigoplus( \sum\limits_{i\neq j} F_j)$ y
\item  $V= F_i \bigoplus H_i$.
\end{enumerate}
\end{lem}

\underline{Demostraci\'on:} Sea $D_i = \bigcap\limits_{i \neq j} H_j$. Sea $F_i$ espacio complementario a $D$ en $D_i$ para toda $i$. Tenemos que $$\dim (F_i) = \dim (D_i) - \dim (D) =$$ $$\dim(V) - \cod ( \bigcap\limits_{i \neq j} H_j) - \dim (V) + \cod ( \bigcap\limits_{i=1}^s H_i)= \cod (H_i)$$

Entonces $$\dim(D) + \sum\limits_{i=1}^s \dim (F_i)= \dim (D) + \sum\limits_{i=1}^s \cod (H_i) = $$ $$\dim(D) + \cod ( \bigcap\limits_{i=1}^s H_i) = \dim V.$$

Para ver que la suma de 1 es directa nos tomamos $f_1 \in F_1$, supongamos que $f_1 = d + f_2 + \dots + f_s$ donde $d \in D$ y $f_i \in F_i$ para toda $i \neq 1$. Por como definimos a las $F_i$, $f_i \in D_i - D$, entonces $f_i \in H_j$ para cada $i \neq j$, entonces $d + f_2 + \dots + f_s \in H_1$, pero $F_1 \cap H_1 = 0$ ya que $F_1 \subset D_1 - D$, por lo tanto, $f_1 = 0$. Lo mismo podemos hacer para los dem\'as $F_j$, entonces la suma es directa.

Como $D \subset H_i$ y $F_j \subset H_i$ para cada $i \neq j$, entonces $H_i \supset H_i = D \bigoplus( \sum\limits_{i\neq j} F_j)$, pero $\cod (H_i = D \bigoplus( \sum\limits_{i\neq j} F_j))= \dim (F_i) = \cod H_i$, por lo tanto, por el teorema de la dimensi\'on se da la igualdad. Juntamos las igualdades de 1 y 2 para obtener 3. $\blacksquare$

\begin{defi}
Sea $L \subset M$ subvariedad y $X$ un campo vectorial en $M$, decimos que \textit{$X$ es tangente a $L$} si para toda $x \in L$ se cumple que $X_x \in T_x L$.
\end{defi}

\begin{lem}
Sean $\{ H_i \}_{i=1}^s \subset \mathbb{R}^m$ familia de subespacios vectoriales en posici\'on general. Sean $\{ X_i \}_{i=1}^s$ campos vectoriales en $H_i$ respectivamente para toda $i$. Entonces existe $X $ campo vectorial en $\mathbb{R}^m$ tal que para toda $i$, $X - X_i$ es tangente a $H_i$.
\end{lem}

\underline{Demostraci\'on:} Elegimos $\{ F_i \}_{i=1}^s \subset \mathbb{R}^m$ como en el lema anterior. Sea $\pi_i: \mathbb{R}^n \to H_i$ la proyecci\'on ortogonal para cada $i$. Como el haz tangente a $\mathbb{R}^m$ es trivial, podemos pensar a $X_i$ como una funci\'on de $H_i$ en $\mathbb{R}^m$. Sea $Y_i = X_i \circ \pi_i$ campo vectorial en $\mathbb{R}^m$, esto para toda $i$. Sea $Z_i = Y_i - \pi_i \circ X_i$. Por como nos tomamos a $\pi_i$, $\im(Z_i) \subset F_i$. Sea $X = \sum\limits_{i=1}^s Z_i$. Notemos que por el lema anterior tenemos que $\pi_j (Z_i) = Z_i$ para $i \neq j$. Entonces 
$$\pi_i (X - X_i) = X - Z_i  - \pi_i (X_i) = X - Y_i + \pi_i(Y_i) - \pi_i (X_i) = X - X_i ,$$
por lo tanto, $X -X_i \in H_i$. $\blacksquare$


\begin{defi}
Sean $\{ M_i \}_{i=1}^s \subset M$ familia de subvariedades de $M$. Sea $x \in \bigcap\limits_{i=1}^s M$, decimos que $\{ M_i \}_{i=1}^s$ est\'an en \textit{posici\'on general en $x$} si $\{ T_x M_i \}_{i =1}^s \subset T_x M$ est\'an en posici\'on general.
\end{defi}

Aunque parezca que nos hemos desviado un poco de nuestra meta, el siguiente lema nos ayudar\'a a relacionar lo que hemos hecho.

\begin{lem}
Sean $\{ M_i \}_{i=1}^s \subset M^m$ subvariedades en posici\'on general en $x$. Entonces existen una vecindad $U$ de $x$ en $M$, una carta $\phi: U \to \mathbb{R}^m$ y $\{ H_i \}_{i=1}^s \subset \mathbb{R}^n$ subespacios tales que $M_i \cap U = \phi^{-1} (H_i)$ para toda $1 \leq i \leq s$.
\end{lem}

\underline{Demostraci\'on:} Sea $m_i = \cod(M_i)$ en $M$. Para toda $i$ existe una vecindad $U_i$ de $x$ en $M_i$, y funciones $\{ f_{i , j} \}_{j=1}^{m_i}$ reales tales que
$$M_i \cap U_i = \{ y \in W \mid f_{i,1} (y) = \dots = f_{i , m_i} (y) = 0 \} .$$
Estas funciones las podemos obtener de la definici\'on de subvariedad. Sea $V = \bigcap\limits_{i=1}^s U_i$ y $n = m - m_1 - \dots - m_s>0$, sabemos que es un n\'umero positivo debido a que los espacios se encuentran en posici\'on general. El conjunto $\{ f_{i,j} \}_{(1 \leq i \leq s)(1 \leq j \leq m_i)}$ tiene cardinalidad $\sum\limits_{i=1}^s m_i$ y el subespacio de $T_x M$ que se anula en $(df_{i,j})_x$ para todas $i$ y $j$ no es nada m\'as que $\bigcap\limits_{i=1}^s T_x M_i$ cuya dimensi\'on es $\sum\limits_{i=1}^s m_i$, ya que los espacios se encuentran en posici\'on general. Como nada m\'as tenemos este n\'umero de $f_{i,j}$ entonces \'estas deben ser linealmente independientes en $(T_x M)^\ast$.

Elegimos funciones $\{g_j \}_{j=1}^n$ reales definidas en $V$ tales que $ \{ (dg_j)_x \}_{j=1}^n$ sean una base de $(T_x M)^\ast$ con las $(df_{i,j})_x$. Sea $\phi: V \to \mathbb{R}^m$ dada por $\phi(y) = (f_{1 , 1} (y) , \dots , g_1 (y) , \dots , g_n (y))$. Por como construimos esta funci\'on, es un difeomorfismo local en $x$ para alguna vecindad $U$ de $x$ en $M$, elegimos a esta $U$ como la vecindad deseada y nuestra $\phi$ cumple lo deseado por construcci\'on. $\blacksquare$

\begin{theorem}
Sea $f:M \to N$ inmersi\'on. Si $f$ tiene cruces normales entonces $f$ es infinitesimalmente estable.
\end{theorem}

\underline{Demostraci\'on:} Sea $ y \in N$ y $\{x_i \}_{i=1}^s = f^{-1} (q)$. Demostraremos que existen vecindades $W$ de $y$ en $N$ y vecindades $\{U_i \}_{i=1}^s$ vecindades de cada $x_i$ respectivamente tales que cumplan:
\begin{enumerate}
\item $U_i \cap U_j = \emptyset$ para $i \neq j$,
\item $f \vert_{U_i}$ es una inmersi\'on propia 1 a 1,
\item $f(U_i) \subset W$ y
\item $f^{-1} (W) = \bigcup\limits_{i=1}^s U_i$.
\end{enumerate}

Primero elegimos vecindades $ \{V_i \}_{i=1}^s$ vecindades de cada $x_i$ respectivamente que cumplan 1 y 2, adem\'as existe una vecindad $W$ de $y$ tal que $f^{-1} (W) \subset \bigcup\limits_{i=1}^s V_i$. Sea $U_i = V_i \cap f^{-1} (W)$ para toda $i$. Sea $N_i = f(U_i)$ , por 2, tenemos que $\{N_i \}_{i=1}^s$ es una familia de subvariedades de $N$. Entonces por el lema anterior $\{N_i \}_{i=1}^s$ est\'an en posici\'on general en $y$. Elegimos a $W$ como la $U$ del lema anterior.

Para cada $y \in N$ tenemos una vecindad $W_y$ en $N$ que cumple los incisos anteriores, entonces $\{ W_y \}_{y \in N}$ es una cubierta abierta de $N$, y como $f$ es continua $\{ f^{-1} (W_y) \}_{y \in N}$ es una cubierta abierta de $X$, como $X$ es compacta podemos substraer una subcubierta finita $\{ f^{-1} (W_i) \}_{i=1}^r$, elegimos una partici\'on de la unidad $\{ \rho_i \}_{i=1}^r$ subordinada a esta cubierta. Sea $\omega$ un campo vectorial a lo largo de $f$, tenemos que $\omega= \sum\limits_{i=1}^r \rho_i \omega$, de aqu\'i obtenemos que nuestro problema lo podemos resolver nada m\'as en una $W_i$ para luego pegar las soluciones de todos.

Entonces $W$ es uni\'on de las subvariedades $ \{ f(U_i)\}_{i=1}^s$. Definimos un campo vectorial en $Y_i$ en $f(U_i)$ como $Y_i = \omega \circ (f\vert_{U_i} )^{-1}$. Entonces existe un campo vectorial $Y$ definido en $W$ tal que $Y \vert_{f(U_i)} - Y_i$ es tangente a $f(U_i)$. Extendemos $Y$ a todo $N$ de modo que $Y \equiv 0$ fuera de $W$, adem\'as $Y$ tiene soporte compacto ya que las $Y_i$ lo tienen. Sea $\omega^\prime = \omega - Y \circ f$, el cual para todo $x \in U_i$ se tiene que $\omega^\prime (x) = \omega(x) - Y (f(x))$ es tangente a $f(U_i)$. Por lo tanto, existe $X_i$ campo vectorial definido en $U_i$ tal que $(df) (X_i) = \omega^\prime$ con soporte compacto, entonces existe $X = X_i$ en $U_i$ para toda $i$ con $X\equiv 0$ fuera de $f^{-1} (W)$ y que cumple lo deseado. $\blacksquare$

Por el teorema de Thom, tenemos que las inmersiones que tienen cruces normales son estables. Ahora, cuando $\dim(N) = 2 \cdot \dim(M)$ las inmersiones son densas en $C^\infty (M,N)$, entonces si un mapeo es estable, este necesariamente tiene que ser una inmersi\'on, entonces bajo estas hip\'otesis tenemos el siguiente teorema.

\begin{theorem}
Supongamos que $2 \dim M \leq \dim N  $. Entonces $f:M \to N$ es estable si y s\'olo si $f$ es una inmersi\'on con cruces normales.
\end{theorem}

Para finalizar este ejemplo notemos que si $f: \mathbb{R} \to \mathbb{R}^2$ es una inmersi\'on con cruces normales, para todo $y \in N$ tenemos que este conjunto no puede tener cardinalidad mayor a 2, ya que, si $x_1, x_2 , x_3 \in M$ son tales que $f(x_1) = f(x_2) = (f_3) =y$, entonces $(df)_{x_1}(T_{x_1} \mathbb{R}) , (df)_{x_2} (T_{x_2} \mathbb{R}) , (df)_{x_3} (T_{x_3} \mathbb{R})$ est\'an en posici\'on general en $T_y \mathbb{R}^2$. Como $f$ es inmersi\'on las dimensiones de estos espacios es $1$ lo cual es equivalente a que su codimensi\'on sea $1$. Entonces la suma de las codimensiones es $3$ y la codimensi\'on de la intersecci\'on a lo m\'as es $2$. Entonces,  de manera m\'as general tenemos el siguiente teorema.
\begin{theorem}
Sea $f:M^m \to N^n$ una inmersi\'on estable y $y \in N$, entonces $f^{-1} (y)$ tiene a lo m\'as $n / (n-m) $ elementos.
\end{theorem}



\subsection{Sumersiones con dobleces}

Cuando $\dim(M) \geq \dim(N)$ demostramos que las sumersiones son estables, ahora debilitaremos un poco nuestras hip\'otesis para poder saber exactamente cu\'ales son todos los mapeos estables, entonces en esta secci\'on asumiremos que la dimensi\'on del dominio es mayor o igual a la del codominio.

\begin{defi}
Sea $f: M \to N$ tal que $j^1 f \transv S_1(M,N)$. Llamaremos a $x \in S_1(M,N)$ \textit{punto de doblez} de $f$ si $T_x (S_1 (f)) + \ker (df)_x = T_x M$, donde $S_1 (f) = (j^1 f)^{-1} (S_1(M,N))$.
\end{defi}

Esta definici\'on tiene sentido ya que $\cod (S_1(f)) = \cod (S_1(M,N)) = m - n +1$ y $\dim (\ker(df)_x) = m - n +1$, por lo tanto, si es posible que existan los puntos de doblez y adem\'as la suma ser\'a directa, es decir, los subespacios $T_x(S_1(f))$ y $\ker (df)_x$ se intersecan en el $0$ de $T_x M$ nada m\'as.

\begin{defi}
\begin{enumerate}

\item Diremos que $f:M \to N$ es una \textit{sumersi\'on con dobleces} si sus \'unicos puntos cr\'iticos son puntos de doblez.
\item Si $f:M \to N$ es una sumersi\'on con dobleces a $S_1(f)$ lo llamaremos el \textit{lugar de doblez}.
\end{enumerate}
\end{defi}

\begin{lem}
Sea $f:M \to N$ sumersi\'on con dobleces, entonces $f$ restringida a $S_1(f)$ es una inmersi\'on.
\end{lem}

La demostraci\'on es clara de la definici\'on de punto de doblez. Notemos que si $f:M \to N$ es una sumersi\'on, entonces es una sumersi\'on con dobleces y si $N = \mathbb{R}$ las sumersiones con dobleces son las funciones de Morse. 

Supongamos que $f:M \to N$ es una sumersi\'on con dobleces, entonces $f$ en $(M - S_1(f))$ tenemos que si $\omega \in C^\infty (f^\ast TN)$ entonces existen $X $ campo vectorial definido en $(M - S_1(f))$ y $Y$ campo vectorial en $N$ tal que $\omega = (df)(X) + f^\ast (Y)$, entonces $f$ es infinitesimalmente estable en esta parte de su dominio, si suponemos que $f \vert_{S_1(f)}$ adem\'as de ser una inmersi\'on, es una inmersi\'on con cruces normales, entonces $f \vert_{S_1(f)}$ tambi\'en es infinitesimalmente estable, por lo tanto, si $f:M \to N$ es una sumersi\'on con dobleces tal que $f_{S_1(f)}$ tiene cruces normales entonces es estable. Antes de enunciar este resultado como se debe demostraremos un lema para demostrar la suficiencia de esto.

\begin{lem}
Sea $f:M \to N$ sumersi\'on con dobleces $x \in S_1(f)$. Entonces existen coordenadas alrededor de $x$ y de $f(x)$, tal que $f$ bajo estas coordenadas es de la siguiente forma:
$$(x_1, \dots , x_m) \to (x_1 , \dots , x_{n-1} , \pm x_n^2 , \dots , \pm x_m^2).$$
\end{lem}

\underline{Demostraci\'on:} Como este es un resultado local podemos suponer que $M = \mathbb{R}^m$ y $N= \mathbb{R}^n$, y que $f \vert_{S_1 (f)}$ es la inclusi\'on can\'onica, $S_1 (f)$ tiene codimensi\'on $m - n +1$, por lo cual tiene dimensi\'on $n-1$, entonces estamos pensando a $S_1 (f)$ como $\mathbb{R}^{n-1} \times \{ 0 \}$ en $\mathbb{R}^m$. En estas coordenadas $f$ es de la forma $$(x_1 , \dots , x_m) \to (x_1 , \dots , x_{n-1} , g(x)).$$ Si $y \in S_1 (f)$ entonces $f(y) = (y_1 , \dots , y_{n-1} , 0)$. Adem\'as tenemos que $\partial g / \partial x_i $ cuando $n \leq i \leq m$. Entonces por el lema de Hadamard tenemos que 
$$g(y) = \sum\limits_{i, j \geq n} h_{i,j}(y) x_i x_j,$$
donde la matriz $ \{ h_{i,j} \}_{i, j \geq n}$ es no singular, ya que si lo fuera entonces elegimos coordenadas tales que $\{h_{i,j} \}_{i , j \geq n}$ tiene entradas $ \pm 1$ y por lo menos un $0$. Entonces $f$ ser\'ia de la forma $$(x_1 , \dots , x_m) \to (x_1 , \dots , x_{n-1} , \pm x_n^2 \pm \dots \pm x_s )$$ donde $s < m$. Pero si esto pasara tendr\'iamos que $S_1(f)$ ser\'ia una subvariedad de codimensi\'on $s - n + 1 < m - n +1$ lo cual no es posible por el teorema de la dimensi\'on. Por lo tanto, la matriz $\{ h_{i,j} \}_{i, j \leq n}$ es un invertible, por lo tanto existe un cambio de coordenadas donde la matriz nada m\'as tiene $\pm 1$ en la diagonal y $0$ en lo dem\'as. $\blacksquare$

\begin{lem}
Sea $x \in M$ punto de doblez de $f$ y $\omega$ campo vectorial a lo largo de $f$ en una vecindad $U$ de $x$ tal que $\omega \vert_{S_1 (f) \cap U} = 0$. Entonces existe $X$ campo vectorial en $U$ tal que $\omega = (df)(X)$ en $U$.
\end{lem}

\underline{Demostraci\'on:} Elegimos coordenadas como en el lema anterior. En estas coordenadas $\omega = \sum\limits_{i=1}^n \omega_i (\partial / \partial y_i )$, elegimos $X$ campo vectorial en $U$ que en sus primeras $n-1$ entradas $X_i = \omega_i$ y que las dem\'as entradas valgan $0$. Es claro del lema anterior que definida as\'i $X$ cumple la igualdad. $\blacksquare$

\begin{theorem}
Sea $f:M \to N$ sumersi\'on con dobleces. Entonces $f$ es estable si y s\'olo si $f \vert_{S_1(f)}$ tiene cruces normales.
\end{theorem}

\underline{Demostraci\'on:} Necesidad. Nada m\'as es necesario demostrar que $f \vert_{S_1(f)}: S_1 (f) \to N$ es infinitesimalmente estable. Sea $\omega$ campo vectorial a lo largo de $f \vert_{S_1 (f)}$. Extendemos a $\omega$ a todo $M$, entonces existen campos vectoriales $X$ en $M$ y $Y$ en $N$ tal que $\omega = (df)(X) + f^\ast (Y)$. Como $TM \vert_{S_1 (f)} = T S_1 (f) \bigoplus \ker(df)$, sea $X^\prime = \pi (X)$ donde $\pi: TM \to T S_1 (f)$ es la proyecci\'on can\'onica. Entonces $\omega = (d f \vert_{S_1 (f)}) (X^\prime) + f^\ast Y$.

Suficiencia. Sea $\omega$ un campo vectorial a lo largo de $f$. Existen campos vectoriales $X^\prime$ en $S_1 (f)$ y $Y^\prime$ en $N$ tales que $\omega = (d f \vert_{S_1 (f)}) (X^\prime) + f^\ast (Y^\prime)$. Extendemos $X^\prime$ a todo $M$, a este campo vectorial lo denotaremos $X$. Sea $\omega^\prime = \omega - (df)(X) - f^\ast (Y^\prime)$. 

Definido de esta manera $\omega^\prime \vert_{S_1(f)} = 0$. Aplicamos el lema anterior en cada punto de $x \in M$ para luego pegar todo con particiones de la unidad. $\blacksquare$


\chapter{Clasificaci\'on de singularidades}

Recordemos que $S_r (M,N) \subset J^1 (M,N)$ es una subvariedad de $M$, si $f: M \to N$ es tal que $j^k f \transv S_r(M,N)$ tenemos que $(j^1 f)^{-1}(S_r(M,N))= S_r(f)$ es una subvariedad de $M$, entonces $f \vert_{S_r(f)}: S_r (f) \to N$ es un mapeo suave entre variedades, en este cap\'itulo estudiaremos esta clase de mapeos. Seguiremos asumiendo que la variedad del dominio es compacta.
\begin{defi}
Sea $f:M \to N$, decimos que $f$ es 1-gen\'erico si $j^1 f \transv S_r$ para toda $r \in \mathbb{N}$.
\end{defi}

\section{Teorema de Whitney para mapeos $1$-gen\'ericos entre $2$-variedades}

En esta secci\'on asumiremos que todas las variedades tienen dimensi\'on 2. Sea $f:M \to N$ un mapeo 1-gen\'erico. Recordemos que $S_r (M,N)$ es una subvariedad de $J^1 (M,N)$ de codimensi\'on $r^2$. Entonces cuando $S_1 (f)$ es una subvariedad de $M$ de dimensi\'on $1$ y $S_2 (f)$ tiene codimensi\'on $4$, lo cual no tiene sentido. Entonces, si $x \in S_1 (f)$ pueden pasar dos cosas por el teorema de la dimensi\'on:
\begin{enumerate}
\item $T_x S_1 (f) \bigoplus \ker (df)_x = T_x M$ y
\item $T_x S_1 (f) = \ker(df)_x$.
\end{enumerate}

El caso 1 es el caso de un punto de doblez; en esta secci\'on estudiaremos el caso 2. Sea $X$ un campo vectorial a lo largo de $f$ tal que para todo punto de $x \in S_1(f)$, $X(x) \in \ker(df)_x$, es decir, $X$ es tangente a $S_1 (f)$. Sea $\rho: M \to \mathbb{R}$ suave tal que $\rho (S_1 (f)) = 0$, entonces la funci\'on $(df) (\rho)$ tiene un cero en $x$.

\begin{defi}
Decimos que $x \in M$ es una \textit{c\'uspide sumple} si este cero es un cero simple.
\end{defi}

Ahora enunciamos el teorema principal de esta secci\'on, el cual tambi\'en es el t\'itulo de esta secci\'on.

\begin{theorem}{Teorema de Whitney para mapeos $1$-gen\'ericos entre $2$-variedades}
Sea $f:M \to N$ 1-gen\'erico y $x \in S_1 (f)$. Entonces 
\begin{enumerate}[a.]
\item Si 1 pasa, existen coordenadas alrededor de $x$ tal que $f$ es de la forma $(x_1 , x_2 ) \to (x_1 , x_2^2) .$
\item Si 2 pasa, existen coordenadas alrededor de $x$ tal que $f$ es de la forma $(x_1, x_2) \to (x_1 , x_1 x_2 + x_2^3)$.
\end{enumerate}
\end{theorem}
\underline{Demostraci\'on:} El primer inciso es el caso de sumersiones con dobleces, que ya fue analizado en el cap\'itulo anterior nada m\'as tenemos que resolver el caso 2. Como este es un resultado local podemos asumir que $M = N = \mathbb{R}^2$ y que $x= 0$. Podemos elegir coordenadas alrededor de $0$ tal que $f$ sea de la forma $(x_1, x_2) \to (x_1, g(x_1, x_2))$ ya que $(df)$ tiene rango $1$ en $0$. Adem\'as en estas coordenadas tenemos que 
$$ (df)_0 =\begin{bmatrix}
1 & 0 \\
0 & 0
\end{bmatrix} .$$
Entonces $\partial g / \partial x_1 (0) = \partial g / \partial x_2 (0) = 0$, adem\'as $d( \partial g / \partial x_2 (0)) \neq 0$, ya que si no lo cumpliera tendr\'iamos que $$\frac{\partial}{\partial x_1} \left( \frac{\partial g}{\partial x_2}\right)_0 = \frac{\partial }{\partial x_2} \left( \frac{\partial g}{\partial x_1}\right)_0 = 0 .$$

Sea $\gamma = 1 / 2 (\partial^2    g /\partial x_1^2 )(0)$, consideremos el mapeo $(x_1 , x_2 ) \to (x_1, \gamma x_1^2)$, el cual tiene el mismo 2-jet en $0$ que $f$ por como definimos $\gamma$, pero esta \'ultima funci\'on siempre es de rango $1$.

El conjunto $S_1(f)$ son los ceros de $\partial g / \partial x_2$, entonces en cada punto de $S_1 (f)$ el n\'ucleo de $(df)$ est\'a generado por $\partial / \partial x_2$, entonces como tenemos una c\'uspide simple en $0$ tenemos que
$$h(0)=\frac{\partial g}{\partial x_2} = \frac{\partial^2 g}{\partial x_2^2} = 0 \quad \text{ y } \quad \frac{\partial h}{\partial x_2^3} \neq 0 .$$

Por el teorema de Malgrange podemos escribir a $x_2^3$ de la siguiente manera;
$$x_2^3 = 3 h_2(x_1, g) x_2^2 + h_1 (x_1, h) x_2 + h_0(x_1,h) ,$$
donde $h_1, h_2, h_0$ son funciones definidas en una vecindad de $g(0)$ que se anulan en $g(0)$. Despejando la ecuaci\'on anterior obtenemos 
$$(x_2 - h_2)^3 + h_3(x_2 - h_4) = h_5 .$$
Donde $h_3$, $h_4$ y $h_5$ son algunas funciones adecuadas. Evaluamos en $x_1 = 0$ y obtenemos que el lado derecho es de la forma $x_2^3 + O(\vert x \vert^4)$, y como $h(0,x_2) =x_2^4 + O ( \vert x_2 \vert^4)$ la igualdad sigue siendo igualdad si y s\'olo si $\partial h_5 / \partial y_2 (0) \neq 0$. El primer t\'ermino de la serie de Taylor de $g$ es un m\'ultiplo de $x_1 x_2$, comparando los dos lados de la igualdad obtenemos que $\partial h_5 / \partial y_1 (0) = \partial h_3 / \partial y_1(0) \neq 0$. Entonces tenemos los siguientes cambios de coordenadas $$(x_1 ,x_2 ) \to (h_3 (x_1, g) , x_2 - h_2(x_1, g)) \quad \text{ y } \quad (y_1 , y_2 ) \to (h_3(y_1 , y_2) , h_5(y_1,y_2)).$$

Juntando estos dos cambios de coordenadas obtenemos lo deseado. $\blacksquare$

\begin{theorem}
Existe un conjunto residual en $C^\infty (M,N)$ tal que si $f$ pertenece a este conjunto entonces sus \'unicas singularidades son puntos cr\'iticos y c\'uspides.
\end{theorem}
Demostraremos este teorema en un contexto m\'as general en el futuro, pero para demostrar esto de una manera m\'as f\'acil necesitaremos introducir una nueva clase de variedades para luego usar el teorema de Thom.

\section{Derivada intr\'inseca} 

Sean $V$ y $W$ espacios vectoriales sobre $\mathbb{R}$, recordemos que las transformaciones lineales de $V$ a $W$ de corango $r$ es una subvariedad de $\hom(V,W)$, este conjunto era una subvariedad de $\hom(V,W)$. Sea $A \in L_r (V,W)$, $K_A = \ker(A)$ y $L_A = \cok (A)$. Sea $N_A: T_A \hom(V,W)/ T_A L_r (V,W)$, recordemos que $T_A \hom(V,W) \sim \hom(V,W)$, entonces definimos el mapeo $\phi= \pi \circ i: T_A \hom (V,W) \to \hom (K_A, L_A)$ donde $i: \hom(V,W) \to \hom (K_A , W)$ es la restricci\'on y $\pi: \hom(K_a, W) \to \hom(K_A, L_A)$ es la proyecci\'on.

\begin{lem}
El n\'ucleo de $\phi$ es el espacio tangente a $A$ en $L_r (V,W)$.
\end{lem}

\underline{Demostraci\'on:} Sea $n$ el rango de $A$, entonces tom\'andonos una base adecuada $A$ es de la forma 
$$\begin{bmatrix}
I_n & 0\\
0 & 0
\end{bmatrix} .$$
Si vemos a $L_r(V,W)$ como la imagen inversa del $0$ del mapeo 
$$\begin{bmatrix}
S & T \\
U & Z
\end{bmatrix} \to Z - U S^{-1} T, $$
donde $S$ es una matriz de $n \times n$, $Z$ de $r \times r$, $T $ de $n \times r$ y $U$ de $r \times n$, entonces $T= U= 0$ en $A$, por lo cual la derivada de este mapeo en $A$ es $\phi$ por como elegimos nuestra base para $A$. $\blacksquare$

\begin{coro}
$\phi$ induce un isomorfismo entre $N_A $ y $\hom(K_A, L_A)$.
\end{coro}

Sean $E$ y $F$ haces vectoriales sobre $M$, y $\rho: E \to F$ morfismo de haces. Podemos pensar a $\rho:M \to \hom(E,F)$, sea $x \in M$. La derivada intr\'inseca es la siguiente composici\'on;
$$T_x M \overset{(d \rho)_x}{\to} T_{\rho(x)} \hom(E,F) \to \hom(K_{\rho(x)} , L_{\rho(x)} \cong N_{\rho(x)}$$
donde $K_{\rho(x)}$ y $L_{\rho(x)}$ denotan al n\'ucleo y al con\'ucleo respectivamente, y $N_{\rho(x)}$ al espacio normal, siendo el \'ultimo mapeo la proyecci\'on can\'onica.


\section{Singularidades $S_{r,s}$}

Sea $f:M \to N$ 1-gen\'erico. Entonces tenemos que $f \vert_{S_r (f)}: S_r (f) \to N$ es un mapeo entre variedades, la pregunta m\'as natural que podemos hacernos es acerca de los puntos cr\'iticos de este mapeo. Denotamos al conjunto $S_{r,s} (f)$ al conjunto de puntos donde la derivada de $f \vert_{S_r (f)}$ disminuye de rango $s$. Entonces $x \in S_{r,s} (f)$ si y s\'olo si $(df)_x$ interseca a $T_x S_r (f)$ en un espacio de dimensi\'on $s$. En el caso de 2 variedades $S_{1,0}(f)$ corresponde a los puntos de doblez y $S_{1,1} (f)$ a las c\'uspides.

Recordemos que $S_r \sim L_r (TM, TN)$. Dado $\sigma \in S_r (M,N)$ con $\alpha(\sigma) = x$ y $\beta(\sigma) = y$, denotamos $K = \ker(\sigma) $ y $L= \cok (\sigma)$, definimos haces vectoriales en $S_r$ denotados por $K$ y $L$, donde $K$ es el haz que a cada $\sigma \in S_r$ le adjunta $K$, de manera an\'aloga definimos $L$. Entonces el haz normal a $S_r$ en $J^1(M,N)$ es  $\hom(K,L)$. 

Sean $V$ y $W$ espacios vectoriales, denotamos al producto tensorial de $V$ y $W$ como $V \otimes W$, si $\{ v_i \}_{i=1}^m \subset V$ y $\{ w_j \}_{j=1}^n \subset W$ son bases de $V$ y $W$ respectivamente entonces, $\{ v_i \otimes w_j \}_{(1 \leq i \leq m)(1 \leq j \leq n)}$ es base de $V \otimes W$. En el caso que $V = W$ al subespacio de $V \otimes V$ generado por $\{ v_i \otimes v_i \}_{i=1}^m$ lo denotamos por $V \circ V$. Al espacio complementario a \'este \'ultimo lo denotaremos por $V \wedge V$. Entonces $V \otimes V = (V \circ V) \bigoplus (V \wedge V)$, a las proyecciones can\'onicas las denotaremos como $\pi_\circ :V \otimes V \to V \circ V$ y $\pi_\wedge : V \otimes V \to V \wedge V$. Entonces $\pi_\circ$ nos induce un morfismo $\pi_\circ^\ast: \hom (V \circ V , W) \to \hom (V \otimes V , W)$. Definimos el mapeo $\xi: \hom(V \otimes V , W) \to \hom(V , \hom (V,W))$ dado por $\xi (f(v \otimes w)) = (f(v))(w)$. Sea $\hom (V \circ V, W)_s  = \xi \circ \pi_\circ^\ast (L_s(V,\hom (V,W))^{-1}$.

Dado un haz vectorial $E$ sobre $M$ cuya fibra es $V$ podemos definir el haz vectorial $E \otimes E$ cuyas fibras son $V \otimes V$, adem\'as con la ayuda de $\pi_\circ$ tambi\'en tenemos al haz vectorial $E \circ E$ cuyas fibras son $V \circ V$. Sea $S_r^{(2)} \subset J^2(M,N)$ la preimagen de $S_1$ bajo la proyecci\'on can\'onica. Entonces tenemos el siguiente diagrama conmutativo $$ \xymatrix{
 \ar[d] S_r^{(2)} \ar[dr]\\\
S_r  \ar[r] & \hom(K \circ K , L) .}$$
Donde la flecha $S_r^{(2)} \to \hom (K \circ K,L)$ es el mapeo inducido por la derivada intr\'inseca, y los otros mapeos son las proyecciones. Ocupando la misma notaci\'on que hemos ocupado tenemos la siguiente proposici\'on;

\begin{prop}
$\hom (V \circ V , W)_s$ es una subvariedad de $\hom (V \circ V , W)$ de codimensi\'on
$$\frac{1}{2} m (m+1) -\frac{n}{2} (m-s) (m-s+1) - s(m-s).$$
\end{prop}
\underline{Demostraci\'on:} Sea $E$ el haz can\'onico de $G(s,V)$, sea $F$ el haz vectorial de $G(s,V)$ cuya fibra en $x \in G(s,V)$ es $V/ E_x= F_x$ donde $E_x$ es la fibra de $x$ en $E$. El conjunto $\hom (F_x \circ F_x , W)_0$ es un subconjunto abierto de $\hom (F_x \circ F_x  ,W)$, entonces $\hom (F \circ F, W)_0 = \bigcup\limits_{x \in G (s,V)} \hom (F_x \circ F_x , W)_0$ es un subhaz fibrado de $\hom ( F \circ F, W)$.

La proyecci\'on can\'onica $\pi_x: V \to F_x$ induce un mapeo $\pi_x \otimes \pi_x: V \otimes V \to F_x \otimes F_x$, adem\'as el mapeo $\pi_x: V \circ V \to F_x \circ F_x$ es suprayectivo. Entonces la imagen de $\pi^\ast: \hom (F \circ F, W) \to \hom (V \circ V,W)$ es $\bigcup\limits_{t \geq s} \hom (V \circ V , W)_t$. M\'as a\'un la restricci\'on de este mapeo a  $\hom (F \circ F ,W) $ es una biyecci\'on con $\hom (V \circ V , W)_s$. Para probar que $\hom (V \circ V , W)$ es una subvariedad mencionamos la siguiente proposici\'on:
\begin{prop}
El mapeo $\pi^\ast: \hom (F \circ F, W)_0 \to \hom (V \circ V, W) - \bigcup\limits_{t > s}  \hom (V \circ V, W)_s$ es una inmersi\'on propia 1 a 1.
\end{prop}

Denotamos a la preimagen de $\hom (K \circ K ,L)_s$ en $S_r^{(2)}$ por $S_{r,s}$, el cual es una subvariedad de la misma codimensi\'on. $\blacksquare$

\begin{theorem}
Sean $f:M \to N$ 1 gen\'erico y $x \in M$. $x \in S_{r,s} (f) $ si y s\'olo si $j^2 f(x) \in S_{r,s}$
\end{theorem}

\underline{Demostraci\'on:} Sea $j^1 f(x) = \sigma \in S_r$. El espacio normal a $S_r$ en $J_1 (M,N)$ en $\sigma$ es $\hom (K_\sigma, L_\sigma)$, entonces la derivada de $j^1 f$ en $x$ induce un mapeo $$T_x M \to \hom (K_\sigma, L_\sigma)$$ dado por la derivada intr\'inseca. Este mapeo es suprayectivo ya que $j^1 f \transv S_r$ en $x$ y su n\'ucleo es el espacio tangente a $S_r(f)$ en $x$. Si $x \in S_{r,s} (f)$ el n\'ucleo de la derivada intr\'inseca interseca al n\'ucleo de $(df)_x$ en un subespacio de dimensi\'on $s$, entonces la restricci\'on de la derivada intr\'inseca a $K_\sigma$ tiene n\'ucleo de dimensi\'on $s$, lo cual es equivalente a que $j^2 f(x) \in \hom(K_\sigma \circ K_\sigma , L_\sigma)_s$. La suficiencia es una consecuencia de la demostraci\'on anterior.$\blacksquare$

El teorema de transversalidad de Thom nos asegura que el conjunto de funciones que satisfacen $j^2 f \transv S_{r,s}$ es un conjunto residual. Un mapeo que interseca transversalmente a $S_{r,s}$ para toda $r,s \in \mathbb{N}$ se llama 2-gen\'erico.

\section{Estratificaci\'on de Thom-Boardman}

De la misma manera que definimos $S_{i,j}$ podemos definir $S_{i,j,k}$, si $f:M \to N$ es 2-gen\'erico definimos a $S_{i,j,k}$ como los puntos en $S_{i,j}$ tales que $f \vert_{S_{i,j}}$ tiene corango $k$. Si sucediera que $S_{i,j,k}$ fuera una subvariedad de $M$, podr\'iamos repetir este proceso para definir $S_{i,j,k,l}$, y as\'i sucesivamente.

\begin{theorem}
Sea $r_1 \geq r_2 \geq \dots \geq r_k \geq 0$ sucesi\'on de n\'umeros enteros, podemos definir $S_{r_1, \dots , r_k}$ subhaz fibrado de $J^k (M,N)$ tal que si $j^1 f$ es transversal a $S_{r_1, \dots, r_l}$ para $l <k$, entonces $S_{r_1, \dots , r_k} (f)$ est\'a bien definido y adem\'as $$x \in S_{r_1, \dots , r_k} (f) \iff j^k f(x) \in S_{r_1, \dots , r_k} .$$
\end{theorem}

Si este teorema fuera cierto, llamaremos a $f:M \to N$ \textit{mapeo de Boardman} si $j^k f$ interseca transversalmente a $S_{r_1, \dots , r_k}$ para toda $k$, entonces por el teorema de transversalidad tenemos que los mapeos de Boardman forman un conjunto residual, por lo tanto, si una funci\'on es estable entonces necesariamente ser\'a un mapeo de Boardman, lamentablemente no demostraremos este teorema. Nos gustar\'ia que los mapeos de Boardman fueran estables. Por el momento lo mejor que podemos hacer es el siguiente teorema.

\begin{theorem}
El conjunto de mapeos de Boardman que satisfacen que para cualquier conjunto de mult\'indices $\{ I_j \}_{j=1}^s$, $\{ x_j \}_{j=1}^s \subset M$ distintos con $x_j \in S_{I_j}$ respectivamente tal que $f(x_1) = \dots = f(x_s) =y$ entonces $\{ (df)_{x_j} (T_{x_j} S_{I_j} (f) \}_{j=1}^s$ est\'an en posici\'on general en $T_y N$, es un conjunto residual.
\end{theorem}

\underline{Demostraci\'on:} El mapeo $\beta: S_{I_1} \times \dots \times S_{I_s} \to N \times \dots \times N$ es una sumersi\'on, entonces $\beta^{-1} (\Delta N)$ es una subvariedad de $S_{I_1} \times \dots \times S_{I_s}$. Por el teorema de multitransversalidad existe un conjunto residual en $C^\infty(M,N)$ tal que $j^k f \transv \beta^{-1} (\Delta N)$ si $f$ se encuentra en este conjunto. Adem\'as es claro que si $j^k f$ es transversal a $S_{I_j}$ entonces $j^k_s f$ es transversal a $S_{I_1} \times \dots \times S_{I_s}$. Entonces la preimagen de $S_{I_1} \times S_{I_s}$ es $S_{I_1} (f) \times \dots \times S_{I_s}$ sin $M^{(s)}$, denotaremos a esta subvariedad como $L$. Entonces $j^k_s f:L \to S_{I_1} \times \dots \times S_{I_s}$ es transversal a $\beta^{-1} (\Delta N)$, entonces $f: L \times \dots \times L \to N \times \dots \times N$ es transversal a $\Delta N$. $\blacksquare$

Lamentablemente los mapeos de Boardman que satisfacen la condici\'on de este teorema no necesariamente son estables para ciertas dimensiones de nuestras variedades.

\section{Fin}

Como vimos en el teorema de Malgrange, las dimensiones de nuestras variedades juegan un papel importante en la teor\'ia de transformaciones estables. En los casos donde las transformaciones eran densas ten\'iamos dimensiones muy particulares. Ahora, dada cierta condici\'on en las dimensiones estudiaremos cuando los mapeos estables no son densos.

\begin{theorem}
Sean $M$ y $N$ variedades de dimensi\'on $n^2$. Entonces existe $f:M \to N$ 1-gen\'erico tal que $S_n (f)$ es no vac\'io.
\end{theorem}

\underline{Demostraci\'on:} Como la transversalidad es un problema local nada m\'as tenemos que construir $f$ alrededor de alg\'un punto $x \in M$. En coordenadas elegimos $f$ de modo que su polinomio de Taylor de grado 2, tenga todos sus coeficientes distintos de $0$, luego por un argumento de particiones de la unidad la podemos extender a todo $M$ de modo que fuera de esta vecindad no tenga puntos cr\'iticos deseados. $\blacksquare$

\begin{lem}
Sean $M$ y $N$ variedades, $W \subset C^\infty (M,N)$ abierto. Entonces $A_W = \{ \sigma \in J^k (M,N) \mid \exists f \in W \text{ y } x \in M \text{ con } \sigma = j^k f(x) \}$ es abierto en $J^k (M,N)$.
\end{lem}
\underline{Demostraci\'on:} Sea $\sigma = j^k f(x) \in A_W$. Elegimos cartas coordenadas $U$ alrededor de $x$ y $V$ alrededor de $f(x)$ tales que $f(\overbar{U} ) \in V$. Sea $\rho: M \to \mathbb{R}$ cuyo soporte est\'e dentro de $U$ y que valga $1$ en $x$. Sea 
$$g_h = \begin{cases}
f + \rho h & \text{ en } U\\
f & \text{ en } M - U,
\end{cases}$$
donde $h$ es un polinomio de grado menor o igual a $k$. Para $h$ cercana al polinomio $0$ tenemos que $g_h \in W$. Entonces $j^k g_h (x)$ cuando $h$ es cercana al polinomio $0$ nos da una vecindad de $\sigma$ en $A_W$. $\blacksquare$

\begin{prop}
Sean $M$ y $N$ variedades de dimensi\'on $n^2$. Sea $f: M \to N$ 1-gen\'erico. Si $S_n (f)$ es no vac\'io y $n > 2$ entonces $f$ no es estable.
\end{prop}
\underline{Demostraci\'on:} Sabemos que $\dim (S_n(f)) = 0$, entonces $S_n (f)$ es un conjunto finito de puntos $\{ x_i \}_{i=1}^s \subset M$. Sea $\omega_i = j^2 f(x_i)$. Si suponemos que $f$ es estable existe una vecindad $W_f$ de $f$ en $C^\infty (M,N)$ tal que todas las funciones en esta vecindad son equivalentes a $f$. Entonces $A_{W_f}$ definido como en el lema anterior es abierto en $J^2 (M,N)$. Si $\alpha \in A_{W_f}$ y tambi\'en est\'a en $S_n^{(2)}$, debe ser conjugada a una de nuestras $\omega_i$.

Entonces $A_{W_f} \cap S_n^{(2)}$ est\'a cubierto por un conjunto numerable de \'orbitas de las $\omega_i$ bajo la acci\'on de $\dif (M) \times \dif (N)$. Sea $\tau = j^1 f(x_1)$, $K_\tau = \ker (\tau)$ y $L_\tau = \cok (\tau)$. Entonces existe un abierto en $\hom (K_\sigma \circ K_\sigma , L_\sigma)$ cubierto por un conjunto numerable de \'orbitas de $\gl (K_\sigma) \times \gl (L_\sigma)$. Si las \'orbitas tuvieran menor dimensi\'on que $\hom (K_\sigma \circ K_\sigma , L_\sigma)$, tendr\'ian medida cero por ser subvariedades inmersas. Para concluir esta proposici\'on demostraremos el siguiente lema para ver que una de las \'orbitas tiene que ser abierta.

\begin{lem}
Sean $V$ y $W$ espacios vectoriales de dimensi\'on $n$. Sea $G= \gl (V) \times \gl (W)$. Entonces $G$ tiene una \'orbita abierta en $\hom (V \circ V, W)$ s\'olo cuando $n >3$.
\end{lem}
\underline{Demostraci\'on:} La dimensi\'on de $\hom (V \circ V , W)$ es $n^2 (n+1) / 2$ y la de $G$ es $2n^2$. Entonces la \'orbita de $G$ tiene dimensi\'on menor o igual a la de $G$. Tenemos que $n^2 (n+1) / 2  \leq 2n^2$ cuando $n \leq 3$, entonces se cumple lo deseado. Ahora, si $n=3$, las dimensiones son iguales, pero $G$ contiene un subgrupo que act\'ua trivialmente en $\hom (V \circ V , W)$, dado por $\langle c Id_V , c^2 Id_W\rangle$ donde $c \in \mathbb{R}$. Por lo tanto no se puede dar la igualdad. $\blacksquare$

Para finalizar mencionamos que Mather demostr\'o que los mapeos estables entre dos variedades $M^m$ y $N^n$ son densos cuando $n$ y $m$ cumplen cualquiera de las siguientes condiciones:
\begin{enumerate}

\item $n < 7(n - m) + 8$ cuando $n-m \geq 4$,
\item $n < 7 (n -m) + 9$ cuando $3 \geq n - m \geq 0$,
\item $n < 8$ cuando $n - m = -1$,
\item $n < 6$ cuando $n - m = -2$ y
\item $n < 7$ cuando $ n - m \leq  - 3$.
\end{enumerate}

\begin{thebibliography}{9}
\bibitem{Differential Topology} 
V. Guillemin, A. Pollack. 
\textit{Differential Topology}. 
Prentice-Hall, Englewood Cliffs, New Jersey, 1974.
 
\bibitem{Introduction to Differential Topology} 
 Th. Bröcker, K. Jänich.
\textit{Introduction to Differential Topology}.
Cambridge University Press, Cambridge, 1982.
 
\bibitem{Differential Topology 2} 
M. Golubitsky, V. Guillemin.
\textit{Stable Mappings and Their Singularities}. 
Springer-Verlag, New York, 1973.
 
\bibitem{Singularities of Differentiable Maps} 
V. Arnold, S. Gussein-Zade, A. Varchenko
\textit{Singularities of Differentiable Maps}. 
Birkhaüser, Boston, 1985.
\end{thebibliography}

\end{document}






